\documentclass[a4paper,
fontsize=11pt,
%headings=small,
oneside,
numbers=noperiodatend,
parskip=half-,
bibliography=totoc,
final
]{scrartcl}

\usepackage[babel]{csquotes}
\usepackage{synttree}
\usepackage{graphicx}
\setkeys{Gin}{width=.4\textwidth} %default pics size

\graphicspath{{./plots/}}
\usepackage[ngerman]{babel}
\usepackage[T1]{fontenc}
%\usepackage{amsmath}
\usepackage[utf8x]{inputenc}
\usepackage [hyphens]{url}
\usepackage{booktabs} 
\usepackage[left=2.4cm,right=2.4cm,top=2.3cm,bottom=2cm,includeheadfoot]{geometry}
\usepackage{eurosym}
\usepackage{multirow}
\usepackage[ngerman]{varioref}
\setcapindent{1em}
\renewcommand{\labelitemi}{--}
\usepackage{paralist}
\usepackage{pdfpages}
\usepackage{lscape}
\usepackage{float}
\usepackage{acronym}
\usepackage{eurosym}
\usepackage{longtable,lscape}
\usepackage{mathpazo}
\usepackage[normalem]{ulem} %emphasize weiterhin kursiv
\usepackage[flushmargin,ragged]{footmisc} % left align footnote
\usepackage{ccicons} 
\setcapindent{0pt} % no indentation in captions

%%%% fancy LIBREAS URL color 
\usepackage{xcolor}
\definecolor{libreas}{RGB}{112,0,0}

\usepackage{listings}

\urlstyle{same}  % don't use monospace font for urls

\usepackage[fleqn]{amsmath}

%adjust fontsize for part

\usepackage{sectsty}
\partfont{\large}

%Das BibTeX-Zeichen mit \BibTeX setzen:
\def\symbol#1{\char #1\relax}
\def\bsl{{\tt\symbol{'134}}}
\def\BibTeX{{\rm B\kern-.05em{\sc i\kern-.025em b}\kern-.08em
    T\kern-.1667em\lower.7ex\hbox{E}\kern-.125emX}}

\usepackage{fancyhdr}
\fancyhf{}
\pagestyle{fancyplain}
\fancyhead[R]{\thepage}

% make sure bookmarks are created eventough sections are not numbered!
% uncommend if sections are numbered (bookmarks created by default)
\makeatletter
\renewcommand\@seccntformat[1]{}
\makeatother

% typo setup
\clubpenalty = 10000
\widowpenalty = 10000
\displaywidowpenalty = 10000

\usepackage{hyperxmp}
\usepackage[colorlinks, linkcolor=black,citecolor=black, urlcolor=libreas,
breaklinks= true,bookmarks=true,bookmarksopen=true]{hyperref}
\usepackage{breakurl}

%meta
%meta

\fancyhead[L]{K. Schuldt\\ %author
LIBREAS. Library Ideas, 39 (2021). % journal, issue, volume.
\href{https://doi.org/10.18452/23449}{\color{black}https://doi.org/10.18452/23449}
{}} % doi 
\fancyhead[R]{\thepage} %page number
\fancyfoot[L] {\ccLogo \ccAttribution\ \href{https://creativecommons.org/licenses/by/4.0/}{\color{black}Creative Commons BY 4.0}}  %licence
\fancyfoot[R] {ISSN: 1860-7950}

\title{\LARGE{Wie von der Automatisierung in Bibliotheken erzählen?}}% title
\author{Karsten Schuldt} % author

\setcounter{page}{1}

\hypersetup{%
      pdftitle={Wie von der Automatisierung in Bibliotheken erzählen?},
      pdfauthor={Karsten Schuldt},
      pdfcopyright={CC BY 4.0 International},
      pdfsubject={LIBREAS. Library Ideas, 39 (2021).},
      pdfkeywords={Bibliotheken, Computer, Automatisierung, Schweiz},
      pdflicenseurl={https://creativecommons.org/licenses/by/4.0/},
      pdfcontacturl={http://libreas.eu},
      baseurl={https://doi.org/10.18452/23449},
      pdflang={de},
      pdfmetalang={de}
     }



\date{}
\begin{document}

\maketitle
\thispagestyle{fancyplain} 

%abstracts
\begin{abstract}
\noindent
\textbf{Kurzfassung:} Ein Buch zur Einführung und Verbreitung von Computern in Bibliotheken im Kanton Genf wird als Anlass genommen zu diskutieren, wie eine Geschichte von Technik in Bibliotheken erzählt werden und was aus ihr gelernt werden kann. Einer der Autoren des Buches hat im Laufe seiner Karriere weitere Werke vorgelegt, welche die Entwicklung von Computern praxisorientiert begleiteten. Dies bietet eine gute Möglichkeit, die Veränderungen seines Blickwinkels zu beschreiben. Dabei geht es im Text aber nicht um eine Kritik dieser Werke, sondern darum zu diskutieren, wie eine Geschichtsschreibung aufgebaut sein kann, welche möglichst viele Fragen stellt, von denen die heutige Praxis profitieren könnte.
\end{abstract}


%body
Dieser Text sollte eine kurze Rezension eines aktuellen Buches werden,
welches thematisch zum Schwerpunkt dieser Ausgabe passt und -- auch weil
es mit fast 400 Seite sehr umfangreich ist -- viel versprach:
\enquote{Histoire d'une (r)évolution: L'informatisation des
bibliothèques genevoises 1963-2018}. Alain Jacquesson und Gabrielle von
Roten (Jacquesson \& von Roten 2019) versprechen mit dieser Geschichte
der \enquote{Informatisation} -- der Ausstattung mit Computer und
ähnlichen Maschinen sowie den damit einhergehenden Umstellungen -- der
Bibliotheken im Kanton Genf seit 1963 Themen anzugehen, die im Call for
Paper für diese Ausgabe der Libreas angesprochen wurden: Den Einfluss
von Technik und Automatisierung bei der Entwicklung von Bibliotheken.
Das klingt relevant, aber leider erfüllt das Buch die Erwartungen, die
es weckt, nicht.

Beide Autor*innen haben das Bibliothekswesen in Genf und der
(französischsprachigen) Westschweiz lange begleitet. Einer der beiden,
Alain Jacquesson, hat über die Jahre weitere Monographien verfasst, die
sich mit der Informatisation von Bibliotheken befassen. Diese sind, im
Gegensatz zum aktuellsten Buch, nicht als historisch Arbeit, sondern zum Beispiel
als Lehrbücher (Jacquesson 1995; Jacquesson \& Rivier 2005) oder
Debattenbeiträge (Jacquesson 2010) konzipiert. Damit liegt ein
interessanter Fall vor: Für das aktuelle Buch wurden Entscheidungen
darüber getroffen, was erzählt und was nicht erzählt wird. Geschichte
wurde hier -- wie immer, wenn über Geschichte geschrieben wird -- mit
einem spezifischen Blick interpretiert. In der Zusammenschau mit den
anderen Büchern, in denen der Autor zeigt, welche Themen ihn zu den
jeweiligen Zeitpunkten beschäftigt haben, lässt sich also fragen, was
ausgewählt, was nicht ausgewählt und was neu interpretiert wurde. Anhand
dieses Beispiels lässt sich -- was hier im Weiteren getan werden soll --
diskutieren, wie man über Automatisierung und Computerisierung von
Bibliotheken sprechen kann und was sich aus dieser Geschichte lernen
lässt.

\hypertarget{zur-einfuxfchrung-der-computer-in-den-genfer-bibliotheken}{%
\section{Zur Einführung der Computer in den Genfer
Bibliotheken}\label{zur-einfuxfchrung-der-computer-in-den-genfer-bibliotheken}}

Beiden Autor*innen, welche das Werk zur Informatisation der Bibliotheken
in Genf vorlegten, haben ihr Berufsleben in diesem Umfeld und der
beschriebenen Zeit verbracht. Gabrielle von Roten koordinierte zum
Beispiel lange Zeit die Bibliotheken der Universität Genf und einige
Zeit auch den westschweizerischen Bibliotheksverbund Rero, war in der
föderalen Kommission für wissenschaftliche Information sowie der
Kommission der Schweizerischen Nationalbibliothek aktiv und trieb die
Gründung des Konsortiums wissenschaftlicher Bibliotheken der Schweiz
voran. Alain Jacquesson leitete zu unterschiedlichen Zeiten die
Bibliotheksschule in Genf, die Öffentlichen Bibliotheken des Kantons und
die Kantonsbibliothek. Er lehrte an der genannten Schule, auch als sie
zur Fachhochschule ausgebaut wurde. Ausserdem beriet er andere
Bibliotheken bei der Einführung von Informationstechnik. Die Anzahl der
schweizerischen Bibliotheken ist überschaubar, die in der
französischsprachigen Schweiz noch mehr. Insoweit kann man sagen, dass
beide wohl alle anderen Aktiven und alle wichtigen Projekte entweder
kannten oder -- wie sich im Buch immer wieder zeigt -- selbst an den
Projekten beteiligt waren. (Wer sich heute in diesen Kreisen bewegt,
wird im Buch deshalb auch immer wieder auf Personen stossen, mit denen
persönliche Bekanntschaften bestehen.)

Die Publikation ist von diesem Hintergrund geprägt. Sie vermittelt den
Eindruck einer möglichst umfassenden Chronologie. Jeder im Kanton
vertretende Bibliothekstyp -- Wissenschaftliche und Öffentliche
Bibliotheken, Schulbibliotheken, Bibliotheken internationaler
Organisationen und Spezialbibliotheken -- werden in einzelnen Kapiteln
behandelt, teilweise hinunter bis zu einzelnen Zweig- oder
Gemeindebibliotheken. Ebenso werden Themen wie die Entwicklung der
Datenbanken, Mikrofiche und CDs, Retrokonversion und Ausbildung in
einzelnen Kapiteln behandelt. Nicht immer liegt der Fokus auf dem Kanton
Genf, vielmehr werden auch Entwicklungen in anderen Kantonen (vor allem
dem Kanton Vaud/Waadt) und Ländern geschildert. Teilweise werden diese
Exkurse an die Entwicklungen in Genf zurückgebunden, beispielsweise wenn
von der Universitätsbibliothek in Genf erst das an der
Universitätsbibliothek Lausanne, später dann das an der Bibliothek der
ETH Zürich entwickelte Bibliothekssystem übernommen wird. Aber oft ist
nicht ganz ersichtlich, warum eine bestimmte Entwicklung erwähnt wird
und eine andere nicht. Auffällig ist auch ein sehr westschweizerischer
Blick: Projekte und Entwicklungen in der Romandie werden intensiv
geschildert, einige Projekte der Nationalbibliothek in Bern und an der
ETH Zürich werden angesprochen. Was ausserhalb dessen in den
Bibliotheken der restlichen Sprachräume der Schweiz passierte, scheint
hingegen die Autor*innen überhaupt nicht zu interessieren. Sehr sichtbar
wird dies am Kapitel über die Ausbildung für Genfer Bibliothekar*innen.
(Jacquesson \& von Roten 2018: 253--278) Selbstverständlich stehen
(angehenden) Bibliothekar*innen in der Schweiz mehrere Möglichkeiten der
Aus- und Weiterbildung offen.\footnote{Der Autor dieses Textes ist
  lehrend an der Fachhochschule in Chur tätig und kann deshalb aus
  Erfahrung sagen, dass immer auch Studierende aus der Romandie über
  die Sprachgrenze hinausgehen und zum Beispiel in Chur studieren.} In
diesem Kapitel werden aber nur, dafür ausführlich, die Entwicklungen des
Curriculums an der Bibliotheksschule, später Fachhochschule in Genf,
Weiterbildungen an den Universitäten Genf, Lausanne, Fribourg und Bern
sowie Kurse für Öffentliche Bibliotheken erwähnt. Weiter wird der Bogen
nicht gespannt -- oder anders gesagt: Über den \enquote{Röstigraben} wird nicht
gegangen.\footnote{Auch keine Aus- und Weiterbildungsgänge in
  Frankreich, obwohl diese von vielen Bibliothekar*innen der Romandie
  gewählt werden und so deren Ausbildungsinhalte Einfluss darauf haben,
  wie in den Bibliotheken in Genf gearbeitet wird.}

Das ganze Buch scheint davon zu leben, an welchen Projekten die beiden
Autor*innen im Laufe ihres Lebens beteiligt waren und zu welchen ihnen
Unterlagen vorlagen. Bei vielen wird sehr tief in die jeweilige
Entwicklung gegangen. Es werden Punkte angeführt, die aus internen
Protokollen zitiert worden sein müssen.\footnote{Bei den
  Quellennachweisen wird nicht nur in diesem Buch, sondern auch in den
  anderen hier angeführten sehr reduziert vorgegangen. Es finden sich
  praktisch keine direkten Nachweise, dafür werden am Ende jedes
  Kapitels die verwendeten Quellen -- geordnet nach Erscheinungsjahr --
  aufgeführt. Deshalb ist es oft nicht nachvollziehbar, woher einzelne
  Angaben oder Zitate im Text stammen.} Teilweise werden bis auf die
letzte Zahl genaue Angaben zu Ergebnissen von Projekten geliefert,
beispielsweise wie viele Medien digitalisiert wurden oder wie hoch der
Etat war (allerdings ohne einzuordnen, was die Geldsummen im jeweiligen
Kontext bedeuten). Andere Projekte hingegen werden nur kurz, teilweise
mit Schätzungen über ihre Ergebnisse, angesprochen. Vorgegangen wird
dabei chronologisch. Es wird zum Beispiel kurz eine Bibliothek
vorgestellt und dann aufgezählt, was in dieser zwischen 1963 und 2018 im
Bezug auf die Informatisation passierte. Manchmal werden beteiligte
Personen vorgestellt, oft geht es nur um den Ablauf von Projekten und
die Entwicklung neuer Technologien. In einigen Fällen führt das dazu,
dass der Text fast nur aus kurzen Absätzen besteht, die immer gleich
anfangen: \enquote{En 1999...}, \enquote{En 2001...}, \enquote{En
2002...}, \enquote{En 2006} (Jacquesson \& von Roten 2019: 197).

Es ist eine Chronologie, die einerseits den Eindruck einer möglichst
umfassenden Sammlung, andererseits grosser Beliebigkeit vermittelt. Man
erfährt vieles nicht, unter anderem, warum in Bibliotheken bestimmte
Entscheidungen getroffen wurden oder wie sich die Auswahl eines
technischen Systems auf die jeweiligen Bibliotheken auswirkte.

\hypertarget{exkurs-technik-determiniert-sozial-determiniert-ant}{%
\section{Exkurs: Technik determiniert, Sozial determiniert,
ANT}\label{exkurs-technik-determiniert-sozial-determiniert-ant}}

Geschichte, die mit Technologie zu tun hat, kann immer auf verschiedene
Weise untersucht und dargestellt werden. Eine radikale Möglichkeit ist,
Technik und ihre Entwicklung als quasi-natürlichen Prozess zu verstehen
und andere Entwicklungen an diesen orientiert zu denken. Eine solche
Geschichtsschreibung stellt die Entwicklung von Technik, ihre
Durchsetzung und Anwendung, oft als logische Abfolge dar, die praktisch
folgerichtig zu dem Zustand führt, welcher am Ende der jeweiligen
Geschichte steht. Es gibt in solchen Erzählungen kaum Abweichungen oder
Alternativen. Wenn die Durchsetzung länger dauert als erwartet oder sich
eine Technologie doch nicht durchsetzt, wird dies zumeist auf andere
Umstände als die Technik selber zurückgeführt, zum Beispiel auf
Weigerungen von Personen, sich mit einer Technik auseinanderzusetzen.
Die Entwicklung von Gesellschaften oder Institutionen wird dann meist
als Ergebnis der Technikentwicklung dargestellt: Weil Computer vernetzt
werden können, werden sie vernetzt und dann von Menschen als Netzwerk
genutzt.

Die diesem Dispositiv radikal entgegenstehende Möglichkeit ist,
gesellschaftliche Entwicklungen in den Mittelpunkt zu stellen. Bei
diesem Ansatz wird gefragt, warum Technik in welche Richtung entwickelt
wurde oder auch, warum und von wem bestimmte Dinge zu Problemen erklärt
wurden, welche mittels Technologien gelöst werden müssen. Technologien
werden somit als Werkzeuge in gesellschaftlichen Entwicklungen und
Auseinandersetzungen verstanden, die nicht neutral sind, sondern denen
Ziele, Denkweisen und Geschichte eingeschrieben sind. Zum Beispiel
könnte gefragt werden, warum Computer überhaupt so entwickelt wurden,
dass sie in Netzwerken zusammengeschlossen werden können. Es würde auch
gefragt werden, welche Veränderungen durch diese technischen
Entwicklungen dann in der Gesellschaft oder in spezifischen
Einrichtungen stattfanden.

Zwischen diesen beiden Polen lassen sich zahlreiche Zwischenstufen
verorten, um die geschichtliche Entwicklung von Technik zu verstehen.
Eine der oft angeführten Möglichkeiten ist die Actor-Network-Theory
(ANT),\footnote{Unter anderem von Bruno Latour ausgearbeitet, siehe
  \emph{On Actor-network Theory. A few Clarifications} in: \emph{Soziale
  Welt} 47, 1996, Heft 4, S. 369--382. oder \emph{Eine neue Soziologie
  für eine neue Gesellschaft. Einführung in die
  Akteur-Netzwerk-Theorie}. Frankfurt am Main: Suhrkamp 2007} welche --
sehr verkürzt gesagt -- die Sichtweise vertritt, dass die Welt und die
Gesellschaft als Netzwerk gestaltet ist, in dem auch Objekten eine
Agency innewohnt. Nicht in dem Sinne, dass Objekte denken würden, aber
so, dass sie durch ihr Vorhandensein Handlungen von Personen motivieren,
Fragen ermöglichen oder gerade verhindern. Ausgearbeitet wurde die ANT
unter anderem am Beispiel von Forschungslaboren, wo die Ansammlung von
Geräten und Wissen dazu führte, dass bestimmte Dinge untersuchbar und
damit zum Teil des vorhandenen Wissens wurden. Bezogen auf Computer
lässt sich mit der ANT fragen, warum sie überhaupt so konstruiert
wurden, dass sie vernetzt werden konnten. Aber dann auch, was sie, als
es sie dann gab, überhaupt ermöglichten, antrieben und als Alternativen
vorstellbar machten, als sie vernetzt wurden.

\hypertarget{die-nutzung-dieser-dispositive-bei-jacquesson}{%
\section{Die Nutzung dieser Dispositive bei
Jacquesson}\label{die-nutzung-dieser-dispositive-bei-jacquesson}}

Das Buch von Jacquesson und von Roten (2019) folgt implizit der ersten
Vorstellung, der Kontinuität der technischen Entwicklung, muss diese
aber immer wieder zurücknehmen, da die Entwicklungen in der Realität
doch immer wieder stocken. Auf der einen Seite beschreiben die
Autor*innen, wie immer wieder neue Technologie entwickelt wurden --
beispielsweise immer leistungsfähigere Rechner, von Grossrechenanlagen
hin zu Laptops und vernetzten Systemen, aber auch von Software für
einzelne bibliothekarische Aufgaben über \enquote{schlüsselfertige}
Bibliothekssysteme hin zu Cloud-basierten Bibliothekssystemen. Auf der
anderen Seite müssen sie immer wieder berichten, wie die Umsetzung von
Projekten gestoppt, verlangsamt oder mit anderen Ergebnissen als
erwartet beendet wurden. Aber es wird nie die Frage gestellt, warum
bestimmte Systeme überhaupt weiterentwickelt wurden, beispielsweise
welche Probleme oder Möglichkeiten neu aufgetaucht sind, die mit neuen
Systemen angegangen werden mussten. Auch die Frage, warum die Umsetzung
in der Realität oft nicht wie anfänglich gedacht stattfand, wird nicht
gestellt. In einigen Fällen finden sich Wertungen, nach denen dies auf
Unzulänglichkeiten einzelner Bibliotheken zurückgeführt wird.\footnote{Zum
  Beispiel wird einmal der Schweizerischen Nationalbibliothek
  vorgeworfen, eine Entscheidung unter anderem aus Gründen der
  Prokrastination nicht getroffen zu haben. (Jacquesson \& von Roten
  2018: 112)} Aber meistens wird einfach das Ergebnis erwähnt und dann
zum nächsten Projekt weitergegangen.

Erstaunlich ist diese Perspektive, weil zumindest Jacquesson im Laufen
seines Berufslebens auch anders argumentiert hat. In seinem Buch
\enquote{L'informatisation des bibliothèques} (Jacquesson 1995)
schildert er, neben einer Geschichte der Computer in Bibliotheken bis
zum damaligen Zeitpunkt, auch, aus welchen Gründen Bibliotheken Technik
auswählen oder sich gegen sie entscheiden sollen. Er führt explizit
durch eine \enquote{Systemanalyse} zur Einführung von Hard- und
Software (Jacquesson 1995: 29--62) hindurch und diskutiert neben möglichen
Vorteilen auch immer wieder potentielle Nachteile und Veränderungen von
Technologie. Zwar trifft er auch Aussagen über wahrscheinliche
Entwicklungen (beispielsweise, dass es tendenziell immer weniger
Bibliotheksverbünde gegeben wird, in diesen Bibliotheken aber ihre
Autonomie grösstenteils erhalten werden können, Jacquession 1995:
149--210), das aber mit Vorsicht.

In dem mit Alexis Rivier gemeinsam publizierten \enquote{Bibliothèques
et documents numériques} (Jacquesson \& Rivier 2005) wird zuerst
ausführlich in Fragen der Digitalisierung eingeführt. Zum Beispiel wird
geschildert, wie Texte und Bilder auf technischer Ebene elektronisch
dargestellt werden. (Jacquesson \& Rivier 2005: 71--94) Ebenso werden
Technologien wie Buchscanner nicht nur mit ihren Funktionen, sondern
auch den dafür notwendigen technischen Grundlagen vorgestellt.
(Jacquesson \& Rivier 2005: 127--167) Am Ende dieses Buches wird
festgehalten, dass die Entwicklung von Technik in Bibliotheken offen
ist. (Jacquesson \& Rivier 2005: 509--544) In diesen beiden Werken werden
zwar Technologien und ihre Entwicklung geschildert, aber den
Bibliotheken auch eine grosse Agency zugestanden. Sie werden ermahnt,
sich aktiv mit Möglichkeiten und Problemen auseinanderzusetzen. Es wird
zwar nicht die Frage gestellt, warum sich bestimmte Technologien in
bestimmte Richtungen entwickeln, aber es wird auch nicht davon
ausgegangen, dass sie umstandslos eine Entwicklung erzwingen würden, die
dann in Bibliotheken umgesetzt werden müssten.

In seinem Buch \enquote{Google Livres et le futur des bibliothèques
numériques} (Jacquesson 2010) geht Jacquesson davon aus, dass Google
Books einen Einfluss auf die Entwicklung von Bibliotheken und der
Nutzung von Medien haben wird. Es ist ein Diskussionsbeitrag zu einem
damals stark diskutierten Thema. In ihm wird betont, dass Google mit
diesem Projekt Digitale Bibliotheken nicht erfunden hat, aber eine
gewisse Form von \enquote{Industrialisierung} eingeführt hätte.
(Jacquesson 2010: 193--214) In Zukunft wäre es nicht mehr möglich, trotz
aller Kritik an dem Projekt selber, über digitale Bibliotheken zu reden,
ohne Google Books zu erwähnen. (Jacquesson 2010: 165) Neun Jahre später,
in der Geschichte der Informatisation der Bibliotheken in Genf, wird es
nicht angesprochen. Auch in der am Ende eingefügten Tabelle (Jacquesson
\& von Roten 2019: 379--389) mit den wichtigsten Entwicklungen im Bezug
auf das Thema in Genf, der Schweiz und weltweit, taucht es nicht
auf.\footnote{Bibliotheken in Genf gehörten nicht zu den Partnern des
  Projektes, aber die Universitäts- und Kantonsbibliothek in Lausanne,
  zu der von einigen Bibliotheken in Genf enge Kontakte bestehen, schon.}

Vorhersagen treffen also auch nicht ein, Projekte scheitern,
Technologien verschwinden wieder. Alain Jaquesson hat dies nicht nur
beim Thema Google Books selbst erlebt, er schildert auch in seinen
anderen Büchern, wie Technologien wieder abgelöst werden (beispielsweise
Mainframe-Computer, Jacquesson 1995: 245--265) oder unerwartete
Gleichzeitigkeiten auftreten (zum Beispiel, dass sich Schreibmaschinen
in Bibliotheken durchsetzten, als die ersten Computer aufkamen,
Jacquesson \& Rivier 2005: 47--69). Aus seinen eigenen Publikationen geht
hervor, dass es keine einfach fortschreitende Geschichte gibt. Und
dennoch versucht er in seinem aktuellen Buch, eine solche zu schreiben.

\hypertarget{muxf6gliche-fragen}{%
\section{Mögliche Fragen}\label{muxf6gliche-fragen}}

Dabei würde die Geschichte, über die von Jacquesson und von Roten (2019)
geschrieben wird, interessante Frage ermöglichen:

\begin{itemize}
\item
  Die Autor*innen schreiben immer so, als wäre die Entwicklung, zu der
  es dann kam, quasi automatisch vorgegeben gewesen. Insbesondere bei
  den Wissenschaftlichen Bibliotheken scheint es so, als würde die
  Entwicklung hin zum gesamtschweizerischen Netz SLSP, welches Ende 2020
  tatsächlich umgesetzt wurde, der logische Endpunkt einer Entwicklung
  zu immer mehr Zusammenarbeit von Bibliotheken sein. Aber gleichzeitig
  scheint sich die Realität regelmässig dagegen gewehrt zu haben: Immer
  wieder kam es zum Abbruch von Projekten, zum Zusammenbruch von
  Kooperationen, zu Problemen bei der Umsetzung von technischen
  Veränderungen. Diese Friktionen zu untersuchen, würde vieles dazu
  klären können, wie in Bibliotheken Entscheidungen getroffen werden.
  Die Bibliotheken in Genf würden sich als Untersuchungsgegenstand dafür
  sogar gut eignen: Es sind sehr viele auf sehr kleinem Raum, mit
  starker Vernetzung. Es stehen für sie eigentlich auch immer einige
  Geldmittel bereit, so dass Entscheidungen nicht einfach aufgrund von
  finanziellen Zwängen getroffen werden, sondern es immer andere Gründe
  geben muss.\footnote{Zu Beginn des Kapitels zu Bibliotheken
    internationaler Organisationen (die es in Genf fraglos gibt) wird
    auch postuliert, dass diese durch ihre Kontakte, beispielsweise
    durch Personal aus anderen bibliothekarischen
    Ausbildungstraditionen, das kantonale Bibliothekswesen bereichern
    würden. (Jacquesson, von Roten 2019: 27) Diese These wird dann nicht
    mehr weiterverfolgt, dabei wäre sie folgenreich.} Zudem sind diese
  Bibliotheken nicht unbedingt immer die, welche in der Schweiz die
  Vorreiterrolle übernehmen, sondern solche, die reagieren, also nicht
  einfach Technologien einführen oder Arbeitsgänge verändern, weil sie
  damit die ersten wären.
\item
  Was die Chronologie deutlich macht, ist, dass sich eigentlich immer in
  Bibliotheken damit auseinandergesetzt wird, wie und welche Technologie
  eingesetzt werden könnte. Was treibt sie dazu? Was erhoffen sie sich
  davon? Warum scheitern viele dieser Überlegungen dennoch? Das wird
  leider nirgends angesprochen.
\item
  Das Buch schildert vor allem die Einführung von Technologien in
  Bibliotheken. Teilweise werden, wie gesagt, auch ganz konkrete
  Projektergebnisse aufgeführt. Aber es wird nicht diskutiert, was diese
  Technologien dann in den Bibliotheken verändert haben: Haben sie zu
  mehr oder weniger Arbeit geführt? Haben sie die Arbeit erleichtert?
  Verändert? Was haben sie automatisiert? Haben die Bibliothekar*innen
  die jeweilige Technologie in ihren Arbeitsalltag integriert und wenn
  ja, wie? Was haben sie gelernt? Was hat sich verändert und was ist
  gleich geblieben? Die chronologische Aufzählung vermittelt den
  Eindruck, als würde einfach die eine Technik auf die nächste folgen
  und als würde der eine Standard natürlicherweise vom nächsten Standard
  abgelöst werden. Aber das ist ja nicht, was in der Realität passiert.
\item
  Ein Thema, dass sich durch das ganze Buch zieht, ist die immer
  stärkere Zusammenarbeit von Bibliotheken, die in gewisser Weise als
  Ergebnis von Projekten zur Einführung von Technologie geschildert
  wird. Beispielsweise waren die Bibliotheken der Universität Genf erst
  selbstständig, so sehr, dass noch nicht einmal bekannt war, wie viele
  es gab. Dann wurden Computer eingeführt und die Bibliotheken der
  Universität -- unter anderem von den beiden Autor*innen --
  koordiniert. Später dann wurden sie an das Bibliothekssystem der
  Universität Lausanne angeschlossen. Endpunkt dieser Entwicklung -- die
  noch über einige weitere Schritte ging -- sei der Zusammenschluss
  aller Wissenschaftlichen Bibliotheken in SLSP (der bei der Publikation
  des Buches noch bevorstand). Aber wie haben sich diese Kooperationen
  gestaltet? Wie wurden sie angegangen und warum? Wer hat sie
  vorangetrieben und was stand ihnen entgegen? Warum, zum Beispiel,
  waren die Bibliotheken an der Universität Genf in den 1970ern
  überhaupt so dezentral organisiert? Warum wurde nicht schon früher ein
  schweizweiter Verbund angestrebt? Hat die Technologie diese
  Zusammenarbeit vorangetrieben oder war sie Mittel zum Zweck? Was hat
  sich in den Bibliotheken konkret verändert? All das wird nicht
  untersucht, obwohl es viel über die Funktion der Bibliotheken und der
  Wirkung von Technologie zeigen würde.
\end{itemize}

Was dem Buch an sich fehlt, ist eine klare Fragestellung und eine sich
daraus ergebende Systematik. Was sollte untersucht werden? Wozu und für
wen ist es geschrieben worden? Was wollten die Autor*innen erzählen oder
untersuchen? Abschliessen hier einige Hinweise dazu, wie dies bei
zukünftigen Auseinandersetzungen mit der Geschichte von Technologie in
Bibliotheken anders gemacht werden könnte.

\begin{itemize}
\item
  An sich wäre es anhand des Exempels der Genfer Bibliotheken gut
  möglich, eine Geschichte von Erwartungen an Technologien,
  Automatisierung und Veränderung der Aufgaben von Bibliotheken zu
  schreiben. Man könnte lernen, wie diese Erwartungen entstehen, dann zu
  Projekten führen -- die ja auch immer Ressourcen binden -- und wie
  damit umgegangen wird, wenn diese Erwartungen nicht eintreten. Bezogen
  auf die Arbeiten von Jacquesson würden sich, neben Google Books, auch
  CD-Roms als Thema anbieten. Er beschreibt diese Technik sowohl in
  seinem Buch von 1995 (Jacquesson 1995) als auch 2005 (\enquote{La
  révolution CD-Rom}, Jacquesson \& Rivier 2005: 56) ausführlich. Im
  Buch von 2018 gibt es eine ganz kurzes Kapitel zur \enquote{Abandon
  des CD-Rom} (Jacquesson \& von Roten 2018: 178). Aber warum haben sich
  zu einem Zeitpunkt Erwartungen an dieses Medium ergeben? Welche
  Probleme sollten mit ihm gelöst werden? Ist die Entwicklung ein
  Beispiel für die Geschichte anderer neuer Medien? Bei einer solchen
  Geschichte könnte auch berücksichtigt werden, dass eine bestimmte
  Anzahl von Vorhersagen tatsächlich eintreffen. Hier ist zum Beispiel
  zu nennen, dass Jacquesson am Ende seines Buches von 1995 postuliert,
  dass eine zukünftige Frage für Bibliotheken sein wird, ob sie Medien
  besitzen oder Zugang zu ihnen schaffen werden. (Jacquesson 1995: 340)
  Das ist eingetroffen. Wie ist das zu erklären, dass diese Vorhersage
  richtig war, andere aber nicht? Kann man aus diesen Beispielen
  vielleicht Modelle bilden, die bessere Voraussagen ermöglichen?
\item
  Das Buch von Jacquesson und von Roten (2018) ist an sich ein gutes
  Beispiel dafür, warum das Dispositiv der Technik-Determination nicht
  haltbar ist: Die Entwicklung von Technik in Bibliotheken ist nicht als
  lineare Geschichte zu schreiben. Dafür wird sie einfach zu oft
  unterbrochen, nimmt Abzweigungen und Entwicklungen werden auch wieder
  vergessen. Aber welches Dispositiv würde sich sonst eignen? Einfach
  von sozialer Determiniertheit auszugehen, ist wohl auch nicht möglich.
  Bibliotheken reagieren ja oft tatsächlich auf technische
  Entwicklungen, die sie nicht selber mitbestimmen. (Die Entwicklung der
  CD-Roms, die eben angesprochen wurde, ist auch dafür ein gutes
  Beispiel.) Auffällig an der Geschichte, wie sie Jacquesson und von
  Roten (2018) erzählen, ist, dass zwar immer wieder die Namen von
  beteiligten Personen genannt, aber nicht auf ihre Rolle für bestimmte
  Entscheidungen eingegangen wird. So wird eine Ebene hinter der Technik
  angedeutet, die offenbar Relevanz hat. Wie lässt sich diese erfassen?
  Was aus dieser lernen? Wie das Zusammenspiel von Technikentwicklung,
  Reaktion der Bibliotheken und des Personals sowie Entwicklung von
  Technik in und für Bibliotheken untersuchen? Zu lernen wäre hierbei
  zum Beispiel, wie gross der Einfluss der Bibliotheken selbst darauf
  ist, wie sich Technologie in der Bibliothek entwickelt. Gleichzeitig
  ist es kein Ausweg, einfach ANT oder ähnliche Dispositive zu wählen.
  Auch diese sind voraussetzungsvoll und zwingen zu Entscheidungen. Wie
  viel Einfluss konkreten Technologien zugestanden wird, ist zum
  Beispiel immer wieder neu zu klären.
\item
  Geschichte an sich wird nicht ohne Grund erzählt. Wie ausgeführt,
  lässt das Buch von Jacquesson und von Roten (2018) offen, was dieser
  Grund ist. Implizit scheint es aber, als wollten sie zeigen, dass die
  Technikentwicklung in Bibliotheken diese dazu treibt, immer enger
  zusammenzuarbeiten. Sinnvoll wäre aber, die eigentlichen Fragen und
  den Grund, warum eine Geschichte interessant genug ist, um sich mit
  ihr zu befassen, vorab und transparent zu klären. Im Fall der Genfer
  Bibliotheken wären zum Beispiel Fragen von Vernetzung und Autonomie,
  von gesellschaftlichen und politischen Einflüssen oder der Veränderung
  von Bibliotheksarbeit durch Technik naheliegend. Grundsätzlich ist
  Geschichtsschreibung, auch wenn sie so ein spezifisches Thema wie die
  Informatisation von Bibliotheken eines Kantons hat, immer dann
  sinnvoll, wenn aus ihr etwas für die Zukunft zu lernen ist,
  beispielsweise für die weitere Entwicklung. Dies sollte bei ähnlichen
  bibliotheksgeschichtlichen Vorhaben berücksichtigt werden.
\end{itemize}

\hypertarget{literatur}{%
\section{Literatur}\label{literatur}}

Jacquesson, Alain (2010). \emph{Google Livres et le futur des
bibliothèques numériques}. (Collections Bibliothèques) Paris: Édition du
Cercle de la Librairie, 2010

Jacquesson, Alain (1995). \emph{L'informatisation des bibliothèques:
Historique, stratégie et perspectives.} (Collections Bibliothèques) --
Nouvelle édition -- Paris: Édition du Cercle de la Librairie, 1995

Jacquesson, Alain ; Rivier Alexis (2005). \emph{Bibliothèques et
documents numériques: Concepts, composantes, techniques et enjeux.}
(Collections Bibliothèques) -- Nouvelle édition -- Paris: Édition du
Cercle de la Librairie, 2015

Jacquesson, Alain ; von Roten, Gabrielle (2019). \emph{Histoire d'une
(r)évolution: L'informatisation des bibliothèques genevoises 1963-2018}.
Genève: L'Esprit de la Lettre Éditions, 2019

%autor
\begin{center}\rule{0.5\linewidth}{0.5pt}\end{center}

\textbf{Karsten Schuldt} ist Wissenschaftlicher Mitarbeiter am
Schweizerischen Institut für Informationswissenschaft, FH Graubünden und
Redakteur der LIBREAS. Library Ideas.

\end{document}
