\textbf{Kurzfassung}: Soziale Roboter sind fähig, mit Menschen zu
kommunizieren oder zu interagieren und erfreuen sich zunehmender
Beliebtheit in Berufs- und Alltagssituationen. Deshalb ist in der FHNW
Campus Brugg-Windisch Bibliothek in Zusammenarbeit mit dem FHNW Robo-Lab
seit Dezember 2019 ein Roboter des Modells Pepper im Einsatz. Der
Roboter beantwortet häufige Fragen, erklärt die Registrierung für die
Bibliotheksbenutzung, gibt Tipps zu Verpflegungsmöglichkeiten und bittet
die Anwesenden kurz vor der abendlichen Schliessung, die Bibliothek zu
verlassen. In einem partizipativen Prozess wird der Roboter durch
Studierende laufend weiterentwickelt und die bestehenden Szenarien
werden mit der Hochschule für Angewandte Psychologie evaluiert. In
diesem Artikel werden die Erkenntnisse aus einer Evaluation nach dem
Human Centered Design Prozess vorgestellt, in welcher Szenarien anhand
von Usability Tests mit Bibliotheksbenutzenden untersucht wurden.

\begin{center}\rule{0.5\linewidth}{0.5pt}\end{center}

\textbf{Abstract}: Social robots are able to communicate or interact
with humans. They are becoming increasingly popular in professional and
everyday situations. For this reason, a robot has been in use at the
FHNW Campus Brugg-Windisch library in collaboration with the FHNW
Robo-Lab since December 2019. The robot answers frequently asked
questions, explains how to register to use the library, gives hints on
catering options and asks the students to leave the library shortly
before closing time in the evening. In a participatory process, the
robot is continuously developed by students and the existing scenarios
are evaluated with the School of Applied Psychology. This article
presents the findings from an evaluation based on the human-centered
design process, in which scenarios were examined using usability tests
with library users.
