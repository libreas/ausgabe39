\textbf{Kurzfassung}: Die Digitalisierung und Onlinepräsentation von
Buchbeständen ist in der Welt der Bibliotheken schon lange kein Novum
mehr. Der Artikel beschreibt am Beispiel der Projekte der Bibliothek des
Deutschen Museums die verschiedenen Formen von Bestandsdigitalisierung
in chronologischer Reihenfolge. Dabei wird nicht nur die Genese eines
konkreten Digitalisierungskonzeptes verständlich, sondern auch generell
Kriterien bibliothekarischer Digitalisierung und ihrer Vorzüge
nachvollziehbar.

\begin{center}\rule{0.5\linewidth}{0.5pt}\end{center}

\textbf{Abstract}: The digitization and online presentation of book
collections has long since ceased to be a novelty in the world of
libraries. Using the projects of the library of the Deutsches Museum as
an example, the article describes the various forms of collection
digitization in chronological order. In doing so, not only the genesis
of a concrete digitization concept becomes clear, but also general
criteria of library digitization and its advantages.
