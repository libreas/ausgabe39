\documentclass[a4paper,
fontsize=11pt,
%headings=small,
oneside,
numbers=noperiodatend,
parskip=half-,
bibliography=totoc,
final
]{scrartcl}

\usepackage[babel]{csquotes}
\usepackage{synttree}
\usepackage{graphicx}
\setkeys{Gin}{width=.4\textwidth} %default pics size

\graphicspath{{./plots/}}
\usepackage[ngerman]{babel}
\usepackage[T1]{fontenc}
%\usepackage{amsmath}
\usepackage[utf8x]{inputenc}
\usepackage [hyphens]{url}
\usepackage{booktabs} 
\usepackage[left=2.4cm,right=2.4cm,top=2.3cm,bottom=2cm,includeheadfoot]{geometry}
\usepackage{eurosym}
\usepackage{multirow}
\usepackage[ngerman]{varioref}
\setcapindent{1em}
\renewcommand{\labelitemi}{--}
\usepackage{paralist}
\usepackage{pdfpages}
\usepackage{lscape}
\usepackage{float}
\usepackage{acronym}
\usepackage{eurosym}
\usepackage{longtable,lscape}
\usepackage{mathpazo}
\usepackage[normalem]{ulem} %emphasize weiterhin kursiv
\usepackage[flushmargin,ragged]{footmisc} % left align footnote
\usepackage{ccicons} 
\setcapindent{0pt} % no indentation in captions

%%%% fancy LIBREAS URL color 
\usepackage{xcolor}
\definecolor{libreas}{RGB}{112,0,0}

\usepackage{listings}

\urlstyle{same}  % don't use monospace font for urls

\usepackage[fleqn]{amsmath}

%adjust fontsize for part

\usepackage{sectsty}
\partfont{\large}

%Das BibTeX-Zeichen mit \BibTeX setzen:
\def\symbol#1{\char #1\relax}
\def\bsl{{\tt\symbol{'134}}}
\def\BibTeX{{\rm B\kern-.05em{\sc i\kern-.025em b}\kern-.08em
    T\kern-.1667em\lower.7ex\hbox{E}\kern-.125emX}}

\usepackage{fancyhdr}
\fancyhf{}
\pagestyle{fancyplain}
\fancyhead[R]{\thepage}

% make sure bookmarks are created eventough sections are not numbered!
% uncommend if sections are numbered (bookmarks created by default)
\makeatletter
\renewcommand\@seccntformat[1]{}
\makeatother

% typo setup
\clubpenalty = 10000
\widowpenalty = 10000
\displaywidowpenalty = 10000

\usepackage{hyperxmp}
\usepackage[colorlinks, linkcolor=black,citecolor=black, urlcolor=libreas,
breaklinks= true,bookmarks=true,bookmarksopen=true]{hyperref}
\usepackage{breakurl}

%meta
\expandafter\def\expandafter\UrlBreaks\expandafter{\UrlBreaks%  save the current one
  \do\a\do\b\do\c\do\d\do\e\do\f\do\g\do\h\do\i\do\j%
  \do\k\do\l\do\m\do\n\do\o\do\p\do\q\do\r\do\s\do\t%
  \do\u\do\v\do\w\do\x\do\y\do\z\do\A\do\B\do\C\do\D%
  \do\E\do\F\do\G\do\H\do\I\do\J\do\K\do\L\do\M\do\N%
  \do\O\do\P\do\Q\do\R\do\S\do\T\do\U\do\V\do\W\do\X%
  \do\Y\do\Z}
%meta

\fancyhead[L]{Redaktion LIBREAS\\ %author
LIBREAS. Library Ideas, 39 (2021). % journal, issue, volume.
{}} % doi 
\fancyhead[R]{\thepage} %page number
\fancyfoot[L] {\ccLogo \ccAttribution\ \href{https://creativecommons.org/licenses/by/4.0/}{\color{black}Creative Commons BY 4.0}}  %licence
\fancyfoot[R] {ISSN: 1860-7950}

\title{\LARGE{Das liest die LIBREAS, Nummer \#8 (Frühling / Sommer 2021)}}% title
\author{Redaktion LIBREAS} % author

\setcounter{page}{1}

\hypersetup{%
      pdftitle={Das liest die LIBREAS, Nummer \#8 (Frühling / Sommer 2021)},
      pdfauthor={Redaktion LIBREAS},
      pdfcopyright={CC BY 4.0 International},
      pdfsubject={LIBREAS. Library Ideas, 39 (2021)},
      pdfkeywords={Literaturübersicht, Bibliothekswissenschaft, Informationswissenschaft, Bibliothekswesen, Rezension,literature overview, library science, information science, library sector, review},
      pdflicenseurl={https://creativecommons.org/licenses/by/4.0/},
      pdfcontacturl={http://libreas.eu},
      baseurl={},
      pdflang={de},
      pdfmetalang={de}
     }



\date{}
\begin{document}

\maketitle
\thispagestyle{fancyplain} 

%abstracts

%body
Beiträge von Eva Bunge (eb), Janine Laura Bromby (jlb), Nadine Ebert
(ne), Dominic Göhring (dg), Lina Feller (lf), Katharina Förster-Kuntze
(kfk), Karsten Schuldt (ks), Pauline Frenzel (pf), Karolina Magdalena
Galek (kmg), Fatima Jonitz (fj), Sara Juen (sj), Sophie Kobialka (sk),
Amber Kok (ak), Monika Kolano (mk), Michaela Voigt (mv), Georg Schelle
(gs), Vivian Schlosser (vs), Valentina de Toledo (vt), Viola Voß (vv)

\hypertarget{zur-kolumne}{%
\section{1. Zur Kolumne}\label{zur-kolumne}}

Ziel dieser Kolumne ist es, eine Übersicht über die in der letzten Zeit
publizierte bibliothekarische, informations- und
bibliothekswissenschaftliche sowie für diesen Bereich interessante
Literatur zu geben. Enthalten sind Beiträge, die der LIBREAS-Redaktion
oder anderen Beitragenden als relevant erscheinen.

Themenvielfalt sowie ein Nebeneinander von wissenschaftlichen und
nicht-wissenschaftlichen Ansätzen wird angestrebt und auch in der Form
sollen Printmedien und elektronische Publikationen ebenso erwähnt werden
wie zum Beispiel Blogbeiträge, Videos oder TV-Beiträge.

Gerne gesehen sind Hinweise auf erschienene Literatur oder Beiträge in
anderen Formaten. Diese bitte an die Redaktion richten. (Siehe
\href{http://libreas.eu/about/}{Impressum}, Mailkontakt für diese
Kolumne ist
\href{mailto:zeitschriftenschau@libreas.eu}{\nolinkurl{zeitschriftenschau@libreas.eu}}.)
Die Koordination der Kolumne liegt bei Karsten Schuldt, verantwortlich
für die Inhalte sind die jeweiligen Beitragenden. Die Kolumne
unterstützt den Vereinszweck des LIBREAS-Vereins zur Förderung der
bibliotheks- und informationswissenschaftlichen Kommunikation.

LIBREAS liest gern und viel Open-Access-Veröffentlichungen. Wenn sich
Beiträge dennoch hinter einer Bezahlschranke verbergen, werden diese
durch \enquote{{[}Paywall{]}} gekennzeichnet. Zwar macht das Plugin
\href{http://unpaywall.org/}{Unpaywall} das Finden von legalen
Open-Access-Versionen sehr viel einfacher. Als Service an den
Leser*innen verlinken wir OA-Versionen, die wir vorab finden konnten,
jedoch auch direkt. Für alle Beiträge, die dann immer noch nicht frei
zugänglich sind, empfiehlt die Redaktion Werkzeuge wie den
\href{https://openaccessbutton.org/}{Open Access Button} oder
\href{https://core.ac.uk/services/discovery/}{CORE} zu nutzen oder auf
Twitter mit
\href{https://twitter.com/hashtag/icanhazpdf?src=hash}{\#icanhazpdf} um
Hilfe bei der legalen Dokumentenbeschaffung zu bitten.

\hypertarget{artikel-und-zeitschriftenausgaben}{%
\section{2. Artikel und
Zeitschriftenausgaben}\label{artikel-und-zeitschriftenausgaben}}

\hypertarget{vermischte-themen}{%
\subsection{2.1 Vermischte Themen}\label{vermischte-themen}}

Fraser-Arnott, Melissa (2020). \emph{Library orientation practices in
special libraries.} In: Reference Services Review 48 (2020) 4: 525--536.
\url{https://doi.org/10.1108/RSR-03-2020-0017} {[}Paywall{]}
{[}OA-Version: \url{https://scholarworks.sjsu.edu/slis_pub/179/}{]}

Die Bedeutung dieses Artikels liegt vor allem darin, auf ein sonst in
der bibliothekarischen Literatur kaum beachtetes Thema hinzuweisen: Die
konkrete Organisation der Ersteinführung von Nutzer*innen in einer
Spezialbibliothek (oder allgemein einer Bibliothek). Diese lässt sich
aktiv gestalten und planen. Im Artikel werden die Ergebnisse einer
Umfrage unter US-amerikanischen Firmenbibliotheken dazu präsentiert, wie
diese diese Einführungen angehen. Das ist nicht uninteressant -- so
warten einige explizit ein paar Wochen, damit die neu eingestellten
Mitarbeiter*innen sich erst in der jeweiligen Firma zurechtfinden und
ihre Informationsbedürfnisse erkennen können, andere verweisen darauf,
dass die Einführungen tatsächlich dazu führen, dass Mitarbeiter*innen
später aktive Nutzer*innen werden. Aber die Hauptnachricht ist, dass es
sich lohnt -- sowohl in einer konkreten Bibliothek als auch allgemein --
über diese Veranstaltungen und die Erfahrungen damit offen nachzudenken
und zu diskutieren. (ks)

\begin{center}\rule{0.5\linewidth}{0.5pt}\end{center}

McGinnis, Robbin; Kinder, Larry Sean (2021). \emph{The library as a
liminal space: Finding a seat of one's own}. In: The Journal of Academic
Librarianship 47 (2021) 1: 102263,
\url{https://doi.org/10.1016/j.acalib.2020.102263}

Dieser Artikel berichtet über den aktuellen Umbau der Bibliothek der
Western Kentucky University. Wie so oft gibt es die Vorstellung, diese
Bibliothek zu einem \enquote{Hub} für unterschiedliche Funktionen
umzubauen: Lernen, Socializing, andere Funktionen. Was den Text und das
Projekt aus der Masse solcher Projektberichte heraushebt, ist eine
Umfrage, die im Laufe des Umbaus zu den Präferenzen der Nutzer*innen
durchgeführt wurde.

Die Ergebnisse wurden genutzt, um den Umbau zu gestalten, sind aber eine
Erinnerung daran, dass die Vorstellung, Universitätsbibliotheken müssten
Hubs werden, nicht unbedingt von den Nutzer*innen selber kommt. Befragt
nach ihren Gründen für die Benutzung der Bibliothek, gaben die meisten
an, studieren zu wollen. Niemand (!) wählte \enquote{socializing} als
Antwort. Die meisten Nutzer*innen suchten Plätze am Fenster, auch wenn
es solche gab, die Orte mit anderen Qualitäten bevorzugten. Ebenso
zeigten sich viele Nutzer*innen nicht von speziellen Sitzgelegenheiten
beeindruckt: Mehr als die Hälfte bevorzugte einfache Tische und Stühle,
erst dann bequemere Stühle oder individuelle Quarrels. Couches
bevorzugten nur 5\,\%, Sitzsäcke fielen ganz durch. Oder anders: Die
Nutzer*innen wollen (auch) in der Bibliothek vor allem klassische
Arbeitsplätze und die Bibliothek hauptsächlich zum Lernen nutzen. (ks)

\begin{center}\rule{0.5\linewidth}{0.5pt}\end{center}

Macleod, Malcolm R.; Michie, Susan; Roberts, Ian; Dirnagl, Ulrich;
Chalmers, Iain; Ioannidis, John P. A. et al.~(2014): \emph{Biomedical
research: increasing value, reducing waste}. In: The Lancet 383 (9912),
S. 101--104. \url{https://doi.org/10.1016/S0140-6736(13)62329-6}.
{[}Paywall{]}

McLeod et al.~analysieren am Beispiel der Biomedizin die Gründe für die
Entstehung von repetitiver, wenig aussagekräftiger Forschung,
sogenanntem \enquote{research waste}. Dies sind zum einen die
kommerziellen Motive der Gesundheitsindustrie und die Profitabilität
beim Verlegen wissenschaftlicher Artikel, sowie zum anderen die sozialen
und politischen Motivationen der ForscherInnen, die unter
Publikationsdruck stehen und zum Teil nicht ausreichend handwerklich
geschult sind. Daraus resultiert eine kurzlebige Forschungskultur, in
der schnelle und einfache Ergebnisse gegenüber nachhaltigen
Erkenntnissen bevorzugt werden. Um die Qualität medizinischer Forschung
zu erhöhen, ist es nötig, die bestehenden Strukturen zu transformieren
und neue Wertmaßstäbe an den wissenschaftlichen Publikationsprozess zu
stellen. Durch die besonderen fachlichen Spezifikationen der Biomedizin,
wie zum Beispiel die starke Abhängigkeit von privaten Drittmittelgebern
und das besondere öffentliche Interesse, können sich in diesem Bereich
die Schwierigkeiten deutlicher herauskristallisieren als in anderen
Bereichen. Die zugrunde liegende Dynamik der verschiedenen Akteure
wissenschaftlicher Forschung lässt sich für andere
Wissenschaftsdisziplinen jedoch übertragen. (vs)

\begin{center}\rule{0.5\linewidth}{0.5pt}\end{center}

Lynch, Renee; Young, Jason C.; Jowaisas, Chris; Boakye-Achampong,
Stanley; Sam, Joel (2020). \emph{African Libraries in Development:
Perceptions and Possibilities}. In: International Information \& Library
Review {[}Latest Articles{]},
\url{https://doi.org/10.1080/10572317.2020.1840002}

Durch Interviews mit Entwicklungshelfer*innen versuchten die Autor*innen
zu bestimmen, ob (und wenn ja, wie) Bibliotheken in Afrika für die
Entwicklungszusammenarbeit genutzt werden können. Über lange Strecken
liest sich das wie eine Klage, dass -- was sich allerdings in den
Interviews auch zeigte -- die Entwicklungshelfer*innen kein richtiges
Bild von Bibliotheken und deren Möglichkeiten hätten. Oft sei nicht
klar, dass es sie überhaupt in Afrika gibt. Wenn, dann würden sie vor
allem mit dem Lesen verbunden, nicht mit anderen Themen wie
Informationsverbreitung oder Communities.

Am Ende des Artikels wird aber über diese Darstellung hinausgegangen. Es
wird diskutiert, dass Bibliotheken selbst dafür sorgen müssen, bei den
Organisationen der Entwicklungszusammenarbeit gesehen zu werden.
Beispielsweise beschreiben die befragten Entwicklungshelfer*innen, dass
sie \enquote{überzeugende Daten} (wobei auch nicht ganz klar, welche
Daten dies genau sein sollten) bräuchten, um zu verstehen, wie
Bibliotheken funktionieren (können). Daraus wird im Artikel geschlossen,
dass Bibliotheken diese Daten liefern müssen. Personen, die sich dafür
interessieren, Bibliotheken in Afrika zu unterstützen, wird deshalb
geraten, diese zu befähigen, solche Daten zu produzieren und bei den
Entwicklungshilfeorganisationen unterzubringen. (ks)

\begin{center}\rule{0.5\linewidth}{0.5pt}\end{center}

Kenyon, Jeremy; Attebury, Ramirose Ilene; Doney, Jylisa;
Seiferle-Valencia, Marco; Martinez, Jessica; Godfrey, Bruce (2020).
\emph{Help-Seeking Behaviors in Research Data Management}. In: Issues in
Science and Technology Librarianship, (2020) 96,
\url{https://doi.org/10.29173/istl2568}

Die Autor*innen -- allesamt an der Bibliothek der University of Iowa
arbeitend -- interviewten 18 Forschende an ihrer Universität dazu, wie
sie mit Forschungsdaten umgehen und vor allem, wo und wann sie sich dazu
Hilfe holen. Grundsätzlich ging es darum, einen Weg für die Bibliothek
zu finden, in diesem Bereich Dienstleistungen aufzubauen. Im Ergebnis
zeigt sich, dass auch Forschende -- und nicht nur Studierende -- nicht
eindeutig wissen, wie mit Daten umzugehen ist und wo sich Hilfe besorgt
werden kann. Interessant ist, dass sie dennoch produktiv sind und
Forschung betreiben. Im Normalfall versuchen sie, wenn es notwendig ist,
sich im eigenen Umfeld Hilfe zu holen, also bei anderen Forschenden oder
Projektmitarbeiter*innen, oder aber selbstständig eine Lösung zu finden.
Einige nutzen auch Services, die an der Hochschule angeboten werden,
inklusive der Bibliothek. Aber hauptsächlich geht es ihnen nicht darum,
den Umgang mit Forschungsdaten vollständig zu verstehen. Sie sind
zufrieden, wenn sie die Aufgaben, die sich aus dem
Forschungsdatenmanagement ergeben, an jemand anderen auslagern können.
Zudem: Auch wenn sie wissen, dass bestimmte Lösungen nicht perfekt sind
und es wohl jeweils bessere gibt, reicht es ihnen, wenn sie soweit mit
den nötigen Daten umgehen können, dass es für ihre jeweilige Forschung
ausreicht. (ks)

\begin{center}\rule{0.5\linewidth}{0.5pt}\end{center}

Mamtora, Jayshree, Ovaska, Claire, and Mathiesen, Bronwyn (2021).
\emph{Reconciliation in Australia: the role of the academic library in
empowering the Indigenous community}. In: IFLA Journal {[}Online
First{]}, 2021, \url{https://doi.org/10.1177/0340035220987578}
{[}Paywall{]}, \url{https://researchonline.jcu.edu.au/65807/}
{[}OA-Version{]}

Der Artikel gibt einen Überblick darüber, wie die Bibliothek an der
James Cook University in Queensland, Australien, vorgeht, um den
\enquote{Reconciliation} genannten Dekolonisierungsprozess in Bezug auf
Australien mitzugestalten. Er zeigt, dass dieser Prozess in Australien
ernsthafter angegangen wird als im DACH-Raum, beispielsweise mit
expliziten Reconciliation Action Plans, und bespricht die
unterschiedlichen Themengebiete, die in der Bibliothek angegangen werden
(unter anderem Personal, Bestand, Veranstaltungen, interkulturelle
Kompetenzen). Gleichzeitig liest sich der Artikel wie ein an manchen
Stellen zu positiv verfasster Erfolgsbericht. (ks)

\begin{center}\rule{0.5\linewidth}{0.5pt}\end{center}

Frick, Claudia (2020). \emph{Peer-Review im Rampenlicht. Ein prominentes
Fallbeispiel.} In: Informationspraxis 6 (2), S. 1--18,
\url{https://doi.org/10.11588/ip.2020.2.74406}

In dem Paper geht es um Open-Peer-Review-Verfahren zur Zeit der Covid-19
Pandemie und den damit einhergehenden Veränderungen in der internen
Wissenschaftskommunikation. Die Autorin betrachtet die Preprint-Kultur
an einem Beispiel des während der Pandemie medial sehr präsenten
Professor Christian Drosten, der als Leiter des Instituts für Virologie
an der Charité in Berlin arbeitet. Sie zeichnet daran den Ablauf eines
Open-Peer-Review-Verfahrens nach. Dabei stellt sie fest, dass die erste
Veröffentlichung nicht auf einem Preprint-Server stattfand, sondern
direkt auf der Webseite der Charité. Dies führte unter anderem zu
schlechterer Auffindbarkeit und geringerer Nachvollziehbarkeit des
Review-Verfahrens. Die Open-Peer-Reviews waren demnach nicht unmittelbar
aufzufinden, es gab keine gesammelte Stelle der Gutachten, keine
Dokumentation nicht-öffentlicher Kommentare zu den Preprints und keine
einheitliche Verknüpfung von Preprint(s) und Reviews. Im Unterschied zum
klassischen Publikationsprozess kommentierten hier GutachterInnen auch
disziplinär übergreifend und wurden später teils sogar zu
Co-Autor*innen. Auch die kurze Zeitspanne zwischen Erstveröffentlichung
des Preprints und ersten Reviews sowie deren unüblich hohe Anzahl werden
von Frick hervorgehoben. Frick stellt die Frage, wie \enquote{open} ein
Open-Peer-Review-Verfahren sein kann, wenn Kriterien wie
Nachvollziehbarkeit, Auffindbarkeit und auch Transparenz kaum gegeben
sind. Sie formuliert daher den Appell, dass die Infrastruktur für
Open-Peer-Review-Verfahren besser organisiert sein müsse. (sk)

\begin{center}\rule{0.5\linewidth}{0.5pt}\end{center}

Martin, Jennifer M. (2021). \emph{Records, Responsibility, and Power: An
Overview of Cataloging Ethics}. In: Cataloging \& Classification
Quarterly 59 (2021) 2--3,
\url{https://doi.org/10.1080/01639374.2020.1871458} {[}Paywall{]},
\url{http://hdl.handle.net/11603/20612} {[}OA-Version{]}

In diesem Text werden Themen zusammengefasst, die offenbar vor allem in
einer Arbeitsgruppe von Katalogier*innen aus dem englischsprachigen Raum
zu ethischen Fragen der Katalogisierung und der Community um diese herum
besprochen werden. Es ist eine gute Übersicht, welche die einzelnen
Themen und jeweils die verschiedenen Positionen (allerdings beschränkt
auf Bibliothekswesen in englischsprachigen Ländern des globalen Nordens)
zu ihnen darstellt. Eine eigene Wertung nimmt die Autorin nicht vor.
Einzig die Überzeugung, dass Katalogisierung eigene ethische Fragen
aufwirft, die in den herkömmlichen Bibliotheksethiken nicht umfassend
geklärt sind und der Impetus, dass der Hauptzweck des Katalogs die
Nutzung des Bestandes ist, vertritt sie selbst. (ks)

\begin{center}\rule{0.5\linewidth}{0.5pt}\end{center}

Vosberg, Dana; Lütjen, Andreas (2021). \emph{Bestandscontrolling bei
elektronischen Ressourcen: Entscheidungshilfen für die Lizenzierung}.
In: o-bib 8 (2021) 1, \url{https://doi.org/10.5282/o-bib/5672}

Eine Umfrage unter Universitäts- und Hochschulbibliotheken in
Deutschland über deren Vorgehen bei den Entscheidungen über Verlängerung
oder Verhandlung von Lizenzen zeigte ein recht unterschiedliches
Vorgehen, das in vielen Fällen auf Erfahrungswerten und Nutzungszahlen
von Verlagen basiert. Viele Auswertungen dieser Zahlen erfolgen
händisch, Entscheidungen werden nicht einfach direkt getroffen, sondern
in komplexeren Zusammenhängen. Der Artikel berichtet über die
Rückmeldungen aus der Umfrage, geht aber darüber kaum hinaus. Dabei
drängt sich eigentlich auf, über die Gründe dieser Situation weiter
nachzudenken und zugleich nach Möglichkeiten, wie diese Entscheidungen
professioneller gestaltet werden können, zu fragen. (ks)

\pagebreak

\hypertarget{covid-19-und-die-bibliotheken-zweite-welle}{%
\subsection{2.2 COVID-19 und die Bibliotheken, Zweite
Welle}\label{covid-19-und-die-bibliotheken-zweite-welle}}

Murphy, Julie A. (edit.) ; Newport, Joshua (2021). \emph{Reflecting on
Pandemics and Technology in Libraries}. In: Serials Review 47 (2021) 1:
37-42, \url{https://doi.org/10.1080/00987913.2021.1879622} {[}Paywall{]}

In dieser Kolumne denkt die Autorin darüber nach, was Bibliotheken aus
der COVID-19 Pandemie lernen konnten und vor allem, was an den Lösungen
für die Zukunft und für zukünftige Gesundheitskrisen dieser Art bleiben
wird. Als Basis nimmt sie vor allem ihre eigenen Erfahrungen als
Bibliothekarin an der Illinois State University. Es sind keine neuen
oder erstaunlichen Aussagen, zu der sie kommt (mehr Hygiene, mehr
Online-Aktivitäten, mehr Lösungen für die Lieferung von Medien).
Interessant ist, dass sie davon ausgeht, dass Bibliotheken (und
Gesellschaften) längerfristig durch die Pandemie verändert wurden und
nicht einfach dahin zurückkehren werden, wo sie Februar 2020 waren. (ks)

\begin{center}\rule{0.5\linewidth}{0.5pt}\end{center}

Alajmi, Bibi M. ; Albudaiwi, Dalal (2020). \emph{Response to COVID-19
Pandemic: Where Do Public Libraries Stand?}. In: Public Library
Quarterly {[}Latest Articles{]},
\url{https://doi.org/10.1080/01616846.2020.1827618}

Alajmi und Albudaiwi analysieren in dieser Studie die Tweets, welche von
Öffentlichen Bibliotheken in New York City während der ersten Monate
2020 (Januar bis April) veröffentlicht wurden. (Anzumerken ist, dass nur
38 der 222 Bibliotheken einen aktiven Twitter-Account haben. Andere
nutzen wohl andere Kanäle.) Die Frage war, wie sich die COVID-19
Pandemie in diesen Tweets widerspiegelt. Sie tut das zum Teil. Der
grösste Teil der Tweets (85,5\,\%) hatte Informationen zum Betrieb der
Bibliotheken zum Thema (zum Beispiel Schliessungen, Lieferdienste für
Medien, Literaturempfehlungen, Veranstaltungshinweise). Die anderen
14,5\,\% bezogen sich direkt auf die Pandemie. Hier wurden vor allem
Hinweise auf Hilfsdienstleistungen in der Community, aufmunternde
Nachrichten oder Hinweise auf weitergehende Quellen zur Pandemie und
Verhaltenshinweisen gepostet. Laut den Autorinnen übernahmen die
Bibliotheken mit ihren Tweets eine wichtige Funktion für die jeweilige
Community, indem sie deren Resilienz stärkten und gleichzeitig für eine
gewisse Normalität standen. (ks)

\begin{center}\rule{0.5\linewidth}{0.5pt}\end{center}

Weeks, Aidy; Houk, Kathryn M.; Nugent, Ruby L.; Corn, Mayra; Lackey,
Mellanye (2020). \emph{UNLV Health Sciences Library's Initial Response
to the COVID-19 Pandemic: How a Versatile Environment, Online
Technologies, and Liaison Expertise Prepared Library Faculty in
Supporting Its User Communities}. In: Medical Reference Services
Quarterly, 39 (2020) 4, 344--358,
\url{https://doi.org/10.1080/02763869.2020.1826197}

Grundsätzlich ist dies einer der vielen Texte, in denen Bibliotheken
vorstellen, wie sie ihre Arbeit während der COVID-19-Pandemie
organisiert haben, hier die Health Library der University of Nevada.
Heraus sticht diese vielleicht, weil sie offenbar den Übergang zur
Online-Arbeit sehr gut gemeistert hat und zudem ihre Kompetenz als
Medizinbibliothek einbrachte, um nicht nur die Universität selbst,
sondern auch die Community in Las Vegas und Clark County mit
Informationen zu versorgen.

Hervorzuheben ist ein kurzer Abschnitt, in dem die Auswirkung der
Pandemie auf das Bestandsmanagement dargestellt wird: Hier wurde, in
Erwartung von zukünftigen Etat-Kürzungen, darauf geachtet, Rechnungen
für längerfristige Abonnements schon 2020 zu zahlen. (ks)

\begin{center}\rule{0.5\linewidth}{0.5pt}\end{center}

Anderson, Raeda; Fisher, Katherine; Walker, Jeremy (2021). \emph{Library
consultations and a global pandemic: An analysis of consultation
difficulty during COVID-19 across multiple factors}. In: The Journal of
Academic Librarianship 47 (2021) 1: 102273,
\url{https://doi.org/10.1016/j.acalib.2020.102273}

Die Autor*innen werten Daten über die wahrgenommene Schwierigkeit bei
der Nutzer*innen\-beratung in Wissenschaftlichen Bibliotheken in den USA
aus. Genutzt wurden Daten, die auf der Basis der in den USA verbreiteten
Reference Effort Assessment Data (READ) Scale genutzt, bei der die
Bibliothekar*innen im Anschluss an eine Beratung die Schwierigkeit
derselben bewerten. Grundsätzlich zeigt sich, dass -- auch bei
unterschiedlichen Voraussetzungen, beispielsweise Zweigbibliotheken mit
spezifischen Klientel -- mit der Umstellung auf die Online-Beratung in
der COVID-19 Pandemie diese als schwieriger und somit fordernder
wahrgenommen wurde. (ks)

\begin{center}\rule{0.5\linewidth}{0.5pt}\end{center}

Yap, Joseph ; Manabat, April (2020). \emph{Managing a sustainable
work-from-home scheme: Library resilency in times of pandemic}. In:
International Journal of Librarianship 5 (2020) 2, 61--77,
\url{https://doi.org/10.23974/ijol.2020.vol5.2.168}

In diesem Artikel wird, leider nur auf der Basis von sechs Interviews,
einerseits über die Umstellung der Arbeit von Bibliotheken während der
Covid-19 Pandemie in Kasachstan berichtet, andererseits über die
Herausforderungen, die sich durch die Arbeit im Homeoffice für die
befragten Bibliothekar*innen ergab. Der erste Teil ist wenig
interessant: Die Bibliotheken taten das, was andere Bibliotheken in
dieser Situation auch taten, wobei die Probleme durch die
wirtschaftliche Lage der Bibliotheken, die wenige elektronische Medien
anbieten konnten, und die nicht vorhandene Katastrophenplanung verstärkt
wurde.

Interessanter ist der zweite Teil. Er zeigt, dass es kein eindeutiges
Bild gibt: Einige Kolleg*innen fanden das Arbeiten daheim positiv,
einige negativ. Bei vielen gab es Probleme durch das Zusammenleben mit
der eigenen Familie -- viele richteten eigene \enquote{ruhige Ecken}
ein, um arbeiten zu können --, aber nicht bei allen. Einige hielten ihre
tägliche Arbeitsroutine aufrecht, andere veränderten sie. Einige fanden
sich in ihrer Produktivität eingeschränkt, die meisten nicht. Hingegen
fanden viele, dass sie während der Pandemie mehr Aufgaben hätten als
zuvor. Die meisten plädierten dafür, nach der Pandemie einen Mix aus
Homeoffice und Arbeit vor Ort zu etablieren. Wünschenswert ist, dass
dieser Teil der Studie inhaltlich in anderen Ländern wiederholt wird.
(ks)

\begin{center}\rule{0.5\linewidth}{0.5pt}\end{center}

Hendal, Batool A. (2020). \emph{Kuwait University faculty's use of
electronic resources during the COVID-19 pandemic}. In: Digital Library
Perspectives 36 (2020) 4, 429--439,
\url{https://doi.org/10.1108/DLP-04-2020-0023} {[}Paywall{]}

Das interessante Ergebnis der Umfrage, das in diesem Artikel präsentiert
wird, ist, dass 60\,\% der befragten Universitätsangehörigen in Kuwait die
elektronischen Medien der Bibliothek gar nicht genutzt haben. In den
Schilderungen vieler Bibliotheken, was sich während der
COVID-19-Pandemie bei ihnen verändert hat, stehen diese Medien und der
Zugang zu diesen, welcher in vielen Einrichtungen vereinfacht wurde, oft
im Fokus. Wenn sich die Ergebnisse der relativ einfachen Umfrage, welche
die Autorin hier präsentiert, in anderen Studien bestätigen lassen, hat
dies aber nur einen (grossen) Teil der möglichen Nutzer*innen erreicht.
Insoweit ist vielleicht die Bedeutung der elektronischen Medien, die im
Mittelpunkt bibliothekarischer Bemühungen während der Pandemie standen,
weiterhin nicht so gross wie gedacht. (ks)

\begin{center}\rule{0.5\linewidth}{0.5pt}\end{center}

Koos, Jessica A. ; Scheinfeld, Laurel ; Larson, Christopher (2021).
\emph{Pandemic-Proofing Your Library: Disaster Response and Lessons
Learned from COVID-19}. In: Medical Reference Services Quarterly 40
(2021) 1: 67--78, \url{https://doi.org/10.1080/02763869.2021.1873624}
{[}Paywall{]}

Im längsten Teil dieses Artikels wird geschildert, wie eine
Medizinbibliothek der State University of New York während der COVID-19
Pandemie agierte, inklusive der ersten Öffnungsschritte nach der ersten
Welle. Dies unterschied sich wenig von anderen Bibliotheken. Interessant
ist, dass sich am Ende Gedanken dazu gemacht werden, was aus dieser
Erfahrung gelernt werden kann: Neben der Verstärkung von
Online-Aktivitäten, einer besseren Planung des Einsatzes von Personal
vor Ort und einigen Umbauten im Raum, betonen die Autor*innen auch, dass
es einer besseren Katastrophenplanung bedarf. (ks)

\begin{center}\rule{0.5\linewidth}{0.5pt}\end{center}

Mayer, Adelheid (2020). \emph{Stress und Flexibilität. Befragung der
Mitarbeiter*innen der Universitätsbibliothek und des Universitätsarchivs
Wien zu den Auswirkungen des ersten Lockdowns auf deren
Arbeitssituation}. In: Mitteilungen der Vereinigung Österreichischer
Bibliothekarinnen und Bibliothekare, 73 (2020) 3--4,
\url{https://doi.org/10.31263/voebm.v73i3-4.5337}

Wie der Titel verkündet, geht es in diesem Artikel um eine Umfrage unter
Mitarbeiter*innen von Bibliothek und Archiv der Universität Wien. Es
werden verbatim, also ohne weitere Analyse oder Bewertung, die
Ergebnisse berichtet. Diese sind recht positiv: Die meisten Kolleg*innen
kamen mit der Umstellung im ersten Lockdown Anfang 2020 gut klar,
bewerten die Arbeit im Homeoffice positiv und wollen sie beibehalten.
Obwohl verbesserungswürdig, funktionierte auch die Kommunikation recht
gut. Abgefragt wurde auch Wissen über die (vorhandenen) Notfallpläne,
die nicht ausreichend bekannt waren. Dennoch gab es Probleme bei einer
Anzahl von Kolleg*innen, insbesondere mit Herausforderungen im
Homeoffice (Einsamkeit, Isolation, aber auch schlechtes Internet und
technische Ausstattung).

Würden sich solche Ergebnisse auch in anderen Bibliotheken zeigen, wären
sie ein Plädoyer für die Erweiterung von Homeoffice-Möglichkeiten, der
besseren Kommunikation von Katastrophenplänen und Betreuung von Personal
in Krisensituationen. (ks)

% \begin{center}\rule{0.5\linewidth}{0.5pt}\end{center}

Suchenwirth, Leonhard (2020). \emph{Sacherschließung in Zeiten von
Corona -- neue Herausforderungen und Chancen}. In: Mitteilungen der
Vereinigung Österreichischer Bibliothekarinnen und Bibliothekare
73.3--4. \url{https://doi.org/10.31263/voebm.v73i3-4.5332}

Ein kurzer Überblick zur Sacherschließung im Besonderen und zum
Bibliotheksbetrieb im Allgemeinen in Pandemie-Zeiten -- der sich
vermutlich 1:1 auf viele (nicht nur Hochschul-)Bibliotheken übertragen
lässt. (vv)

\hypertarget{informationskompetenz}{%
\subsection{2.3 Informationskompetenz}\label{informationskompetenz}}

Hicks, Alison; Lloyd, Annemaree (2020). \emph{Deconstructing information
literacy discourse: Peeling back the layers in higher education}. In:
Journal of Librarianship and Information Science {[}Online First{]},
\url{https://doi.org/10.1177/0961000620966027}

In dieser Studie wird eine Diskursanalyse der englischsprachigen
bibliothekarischen Literatur zur Informationskompetenz im
Hochschulbereich durchgeführt. Diskurs, so erinnern die Autorinnen
richtig, ist nicht nur, wie über ein Thema geredet wird, sondern hat
auch materielle Folgen: Er bestimmt, wie Veranstaltungen geplant,
Ressourcen genutzt, Ziele bestimmt und gemessen werden. Im Fall der
Informationskompetenz bestimmt er auch, wie in Bibliotheken über
Studierende gedacht wird.

Die Studie zeigt abstrakt, dass Diskursanalyse hilfreich ist, um
bibliothekarische Praxis zu verstehen (und dann vielleicht auch zu
verbessern). Konkret werden im Bereich Informationskompetenz zwei
unterschiedliche Denkweisen aufgezeigt. Auf der einen Seite gibt es eine
Tradition, die Praxis der \enquote{Informationskompetenz-Arbeit} zu
kritisieren und darauf zu drängen, konstruktivistische Ansätze zu
stärken. Es geht hierbei darum, davon auszugehen, was Studierende (und
andere) tatsächlich tun und mit Beratungen, Lernangeboten und so weiter
daran anzuschliessen. Demgegenüber steht auf der anderen Seite eine
Praxis, die eher nach innen gerichtet ist und Bibliotheken als
Lehranstalten versteht, die Studierenden, deren Fähigkeiten als
defizitär angesehen werden, etwas beibringen. In dieser Sicht werden
Studierende und deren Praktiken eher abgewertet, die der Bibliotheken
aufgewertet. (ks)

\begin{center}\rule{0.5\linewidth}{0.5pt}\end{center}

Bennedbaeka, D.; Clarka, S.; George, D. (2020). \emph{The impact of
librarian-student contact on students' information literacy competence
in small colleges and universities}. In: College \& Undergraduate
Libraries {[}Latest Articles{]},
\url{https://doi.org/10.1080/10691316.2020.1830907} {[}Paywall{]}

Die Autor*innen dieser Studie postulieren, auf der Basis schon
vorhandener Studien, dass der direkte Kontakt zwischen
Bibliothekar*innen und Studierenden dazu führt, dass letztere die
Kompetenzen ersterer überhaupt wahrnehmen und sich dann erst bei Fragen
bezüglich Recherche (und anderer Themen) an diese wenden würden. Dies
sei in kleinen Hochschuleinrichtungen einfacher als in grossen. Dort, wo
die Person, welche in Einführungsveranstaltungen die Bibliothek und
deren Angebote vorstellt, auch die Person ist, die dann in der
BIbliothek zu finden ist, sei die Wahrscheinlichkeit grösser, dass sie
angesprochen wird. Gleichzeitig, so die Autor*innen weiter, würde sich
die meiste Forschung zu Informationskompetenz und die Arbeit von
Bibliotheken, die sich um dieses Thema gruppiert, gerade mit der
Situation in grossen Hochschulen beschäftigen.

In ihrer Studie erstellen und testen sie -- mithilfe einer Umfrage --
ein Modell, welches in einer kleinen Hochschule einen direkten
Zusammenhang zwischen gutem Kontakt von Bibliothekar*innen und
Studierenden auf der einen Seite und der Informationskompetenz der
Studierenden sowie der Nutzung von Bibliotheksressourcen postuliert.
Dieses Modell stellt sich, zumindest für diese Hochschule, als sehr
tragfähig heraus: Der direkte Kontakt führt dazu, dass die Studierenden
die Bibliothek mehr nutzen, die Bibliothekar*innen als kompetenter
wahrnehmen und schlussendlich eine höhere Informationskompetenz
ausprägen.

So ein Ergebnis hat, wenn es sich auch in anderen Zusammenhängen
bestätigt, eine hohe praktische Bedeutung für Bibliotheken und den
Einsatz von Bibliothekspersonal: Studierende sollten die Personen immer
wieder sehen, zu denen sie schon Kontakt hatten. {[}Siehe auch den Text
von Shin (2020), der hier weiter unten besprochen wird.{]} (ks)

\begin{center}\rule{0.5\linewidth}{0.5pt}\end{center}

Revez, Jorge ; Corujo, Luís (2021). \emph{Librarians against fake news:
A systematic literature review of library practices (Jan.~2018--Sept.
2020)}. In: The Journal of Academic Librarianship 47 (2021) 2: 102304,
\url{https://doi.org/10.1016/j.acalib.2020.102304} {[}OA-Version:
\url{http://hdl.handle.net/10451/45706}{]}

Wie im Titel angegeben, wurde für diese Studie eine systematische
Literaturrecherche durchgeführt, um zu zeigen, welche Strategien
Bibliotheken \enquote{gegen fake news} anwenden. Dabei wurde, wie oft
bei solchen Recherchen, auf Datenbanken zurückgegriffen, die einen Bias
hin zu anglo-amerikanischen Publikationen haben. Insoweit sind die
gefundenen und ausgewerteten Texte vor allem aus anglo-amerikanischen
Ländern. Was dennoch gezeigt wird, sind zwei Dinge: Zum einen schliessen
die meisten dieser Strategien an ihre schon vorhandenen Angebote im
Bereich Informationskompetenz an {[}was die Frage aufwirft, ob dies
wirklich sinnvoll ist oder einfach eine Erweiterung dessen, was
Bibliotheken schon tun{]} und zum anderen sind die Erfolge dieser
Angebote bislang praktisch nicht untersucht. Es werden in der
bibliothekarischen Literatur vor allem Projekte geschildert, aber nicht
deren Effekte. (ks)

\begin{center}\rule{0.5\linewidth}{0.5pt}\end{center}

Perry, Heather Brodie: \emph{Is Access Enough? Interrogating the
Influence of Money and Power in Shaping Information}. In: Open
Information Science 4 (1): 29--38,
\url{https://doi.org/10.1515/opis-2020-0003}

Open Access ist eine wertvolle Errungenschaft für den freien Zugang zu
Information. Dennoch bliebe auch bei einem weitreichenden freien
gesamtgesellschaftlichen Zugang das Problem ungleicher
Nutzungsmöglichkeiten. Denn es bleibt die Herausforderung der
Informationskompetenz und damit verbunden die Notwendigkeit, abschätzen
zu können, ob eine Quelle seriös oder unseriös ist: \enquote{Information
consumers may not possess the competence required to navigate the
complex information ecosystem to find the accurate, high-quality,
resources required to meet their need.} Bibliotheken besitzen auch hier
die Aufgabe, die Menschen so zu schulen, dass diese mit den verfügbaren
Informationen bestmöglich umgehen. Neben dem Zugang zur Information, ist
die Vermittlung des Wissens um die Qualität und Gültigkeit einer
Information essentiell. (lf)

\hypertarget{open-access}{%
\subsection{2.4 Open Access}\label{open-access}}

Kirsch, Mona Alina (2020). \emph{Plan S in der Diskussion -- Reaktionen
aus der Wissenschaft auf die internationale Open-Access-Initiative}. In:
Perspektive Bibliothek 9 (2020) 1,
\url{https://doi.org/10.11588/pb.2020.1.77850}

Die Autorin gibt einen Überblick über die Entwicklung und Rezeption von
Plan S. Dabei geht sie auf Reaktionen von Wissenschaftlerinnen und
Wissenschaftlern, aber auch Fachgesellschaften, Forschungs- und
Förderorganisationen sowie politischen Akteuren ein. So wrd der Einfluss
aufgezeichnet, den diese unterschiedlichen Akteure auf die Ausarbeitung
von Plan S genommen haben und der öffentliche Diskurs um das Vorhaben
wird nachvollzogen. Der Artikel bietet somit einen guten Einstieg, um
sich mit dem Thema vertraut zu machen. Insbesondere der ausführliche
Literaturapparat bietet viele Möglichkeiten, sich mit ausgewählten
Themen vertiefend zu befassen. (eb)

\begin{center}\rule{0.5\linewidth}{0.5pt}\end{center}

Ball, Joanna ; Stone, Graham ; Thompson, Sarah (2021). \emph{Opening up
the Library: Transforming our Policies, Practices and Structures}. In:
Liber Quarterly 31 (2021): 1--16,
\url{https://www.liberquarterly.eu/articles/10.18352/lq.10360/}

In diesem Text werden Diskussionen aus Workshops dazu vorgestellt, wie
Bibliotheken auf den von den Autor*innen wahrgenommenen Trend zur
Förderung von Open Access für wissenschaftliche Monographien reagieren
sollten. Als Hauptkritik wird geäussert, dass oft die
Erwerbungsabteilungen und die Teams, welche in der gleichen Bibliothek
im Bereich Open Access tätig sind, unterschiedliche Ziele und
Vorstellungen hätten. Die Strukturen der Erwerbungsabteilungen seien
sehr darauf ausgerichtet, Monographien zu kaufen, was der Transformation
hin zu Open Access im Weg stünde. Im Text wird argumentiert, dass die
Bibliotheken sich strukturell ändern müssten, um die Transformation
voranzutreiben. Er ist als Diskussionsanstoss zu lesen, bei denen an
Open Access interessierte Kolleg*innen ihre Argumente präsentieren, aber
andere Personen nicht zu Wort kommen.

Zudem verweist der Text erstaunlich oft darauf, dass sich durch die
Pandemie viel verändert hätte, so als würden die Autor*innen hoffen, die
Krise hätte im Ergebnis die Veränderungen, die sie sich erwünschen,
hinterrücks eingeführt. (ks)

\begin{center}\rule{0.5\linewidth}{0.5pt}\end{center}

Lange, Jessica / Hanson, Carrie (2020). \emph{\enquote{You Need to Make
it as Easy as Possible for Me}: Creating Scholarly Communication Reports
for Liaison Librarians.} In: Journal of Librarianship and Scholarly
Communication 8.1:p.eP2329, \url{http://doi.org/10.7710/2162-3309.2329}

\enquote{Scholarly Communication (SC) {[}verstanden v.a. als
(Open-Access-)Publizieren, V.V.{]} is becoming a core function of
liaison librarians' work}, haben die beiden Autorinnen von der
kanadischen McGill-Universität festgestellt: \enquote{The typical
trifecta of liaison librarian positions (collections, reference, and
teaching) is evolving, and the role now demands greater integration into
the research life cycle at the university. Scholarly communications is a
notable example of an area where liaison librarians can expand their
capabilities to assist students and researchers.} (S. 2)

Damit stellt sich die Frage, wie man Liaisons in diesem neuen Gebiet
verbessern kann, und das möglichst kompakt, da sie oft unter Zeitmangel
leiden (wie ja eigentlich jede:r im Bibliothekswesen). In einem
Pilotprojekt mit einer LIS-Masterstudentin wurden daher für zwei
Fachgebiete \enquote{Scholarly Communications Reports} entwickelt, die
Informationen zu Publikationstrends in den Fachgebieten allgemein und
zum Publikationsverhalten der jeweiligen Institute der
McGill-Universität zusammenstellen.

Am Ende des Projekts hatte die Studentin viel in Sachen Publizieren und
Open Access gelernt, und die Liaisons hatten einen handlichen Überblick,
der in Gesprächen mit \enquote{ihren} Wissenschaftler:innen hilfreich
sein könnte. Die Erstellung so umfangreicher Reports, noch ergänzt um
die nach Projektabschluss für zukünftige Berichte notierten weiteren
Daten, ist \enquote{neben dem Fachreferats-Alltagsgeschäft} wohl in der
Tat kaum zu leisten. Vielleicht wäre das auch in deutschsprachigen
Bibliotheken ein Thema für Praktikanten-/Ausbildungs-Projekte;
vielleicht pickt man sich aber auch als Fachreferent:in gezielt einige
der Informationen, die gesammelt wurden, heraus, die als
Hintergrundwissen für die eigene alltägliche Beratungsarbeit hilfreich
sein könnten, und recherchiert sie bei Gelegenheit selbst. Die
methodologischen Notizen im Anhang können dabei eventuell nützlich sein.
(vv)

\hypertarget{kommunikation-von-bibliotheken-mit-forschenden-liaison-librarians}{%
\subsection{2.5 Kommunikation von Bibliotheken mit Forschenden / Liaison
Librarians}\label{kommunikation-von-bibliotheken-mit-forschenden-liaison-librarians}}

Safin, Kelly / Kiner, Renee (2020). \emph{Campus Engagement: Faculty
Recognition and the Library's Role}. In: Journal of Library Outreach \&
Engagement (JLOE) 1.1:1--5.
\url{https://doi.org/10.21900/j.jloe.v1i1.444}

Wie kann man \enquote{als Bibliothek} eine engere Bindung zu den eigenen
Hochschulinstituten aufbauen? Die Millstein Library der Universität
Pittsburgh (\url{https://library.pitt.edu/greensburg}) hat es mit
\enquote{faculty recognition events} versucht: Ausstellungen und
Präsentationen des \enquote{Forschungsoutputs} (Veröffentlichungen,
erhaltene Grants, Preise) ihrer Hochschule. Die ersten beiden
Veranstaltungen waren erfolgreich: \enquote{Faculty seemed genuinely
happy that the library hosted this informal networking event. They were
able to view their colleagues' work while answering questions about
their own achievements. {[}\ldots{]} The library was open during this
event, and students were able to view the posters, browse publications,
and talk with their instructors. Students also stopped and looked at the
displays of faculty work after the event.} Die Reihe soll daher
fortgesetzt werden; 2020 fand die \enquote{Celebration of Faculty
Scholarship and Service} Corona-bedingt online statt:
\url{https://pitt.libguides.com/pgfacultycelebration}. Vielleicht eine
Anregung auch für hiesige Bibliotheken?

Übrigens ist dies der erste Artikel des neugegründeten \enquote{Journal
of Library Outreach \& Engagement (JLOE)},
\url{https://iopn.library.illinois.edu/journals/jloe/about}. (vv)

\begin{center}\rule{0.5\linewidth}{0.5pt}\end{center}

Schoonover, Dan / Kinsley, Kirsten / Colvin, Gloria (2018).
\emph{Reconceptualizing Liaisons: A Model for Assessing and Developing
Liaison Competencies to Guide Professional Development}. In: Library
Leadership \& Management (LL\&M) 34.4.
\url{https://doi.org/10.5860/llm.v32i4.7275} {[}Die DOI ist leider noch
nicht registriert; alternative URL:
\url{https://journals.tdl.org/llm/index.php/llm/article/view/7275}.{]}

Die Frage \enquote{Was macht \enquote*{das Fachreferat} in
wissenschaftlichen Bibliotheken heute eigentlich noch/nicht
mehr/zukünftig?} treibt viele Fachreferent:innen und auch
Bibliotheksleitungen um. Dieser Bericht aus der Florida State University
kann Material zum weiteren Nachdenken bieten. An der Bibliothek wurde
ein \enquote{set of core values and competencies} für Liaison Librarians
entwickelt. Die Werte: Engagement, Advocacy, Collaboration; die
Kompetenzen: Research Services, Scholarly Communication, Use of Digital
Tools, Teaching, Collection Development and Access. Nach einer Erhebung
zur Selbsteinschätzung unter den Liaisons der Bibliothek wurde ein
Trainingsprogramm erarbeitet, um die Bereiche, in denen
Fortbildungsbedarf gesehen wurde, zu unterstützen. (vv)

\begin{center}\rule{0.5\linewidth}{0.5pt}\end{center}

Shin, Eun-Ja (2020). \emph{Embedded librarians as research partners in
South Korea}. In: Journal of Librarianship and Information Science,
{[}Latest Articles{]} \url{https://doi.org/10.1177/0961000620962550}
{[}Paywall{]}

Die Autorin postuliert, dass immer mehr Bibliothekar*innen als
Co-Autor*innen auf wissenschaftlichen Veröffentlichungen auftauchen
würden, weil sie als Embedded Librarians direkt an der Arbeit an
Artikeln beteiligt seien. Sie möchte für Südkorea herausarbeiten, was
genau diese Bibliothekar*innen machen. In einem ersten Teil ihrer
Analyse, die auf einer Auswertung aller Artikel in Scopus von 2010 bis
2019 beruht, welche diesem Rahmen entsprechen, zeigt sie, dass diese
Aussage vor allem für die Medizin und angrenzende Gebiete --
beziehungsweise vor allem Medizinbibliotheken -- gilt, jedoch fast nicht
für andere Bereiche.

Im zweiten Teil berichtet sie von drei Interviews mit
Bibliothekar*innen, die als Co-Autor*innen auf wissenschaftlichen Papern
auftauchen. Das ist interessant, weil es zeigt, dass diese Arbeit vor
allem über persönliche Kontakte funktioniert: Forschende lernen entweder
durch direkte Anfragen an Bibliotheken (aufgrund von Problemen mit
Datenbanken et cetera) oder in Weiterbildungen der Bibliotheken deren
Angebote (und die betreffenden Bibliothekar*innen) kennen und
realisieren dann, dass diese wertvolle Zuarbeit für die Forschungsarbeit
leisten können. Dann erst werden Bibliothekar*innen eingebunden,
beispielsweise als Expert*innen bei der Recherche in Datenbanken, der
systematischen Analyse von Literatur oder bei der Arbeit an Artikeln.
Erst durch diese Tätigkeit entsteht ein Vertrauensverhältnis zwischen
Forschenden und Bibliothekar*innen, die es möglich macht, dass die
geleistete Arbeit durch Co-Autor*innenschaft gewürdigt wird.

Wie gesagt ist das bislang vor allem im medizin-bibliothekarischen Feld
zu beobachten. Aber das heisst ja nicht, dass sich diese Praktik in
Zukunft nicht auch in anderen Feldern etablieren könnte. (ks)

\begin{center}\rule{0.5\linewidth}{0.5pt}\end{center}

Darch, Peter T.; Sands, Ashley E.; Borgman, Christine L.; Golshan,
Milena S. (2020). \emph{Library cultures of data curation: Adventures in
astronomy}. In: JASIST 71 (2020) 12: 1470--1483,
\url{https://doi.org/10.1002/asi.24345} {[}Paywall{]},
\url{https://escholarship.org/uc/item/5r80d66g} {[}OA-Version{]}

Man könnte vermuten, zum Thema Datenmanagement in Bibliotheken sei schon
fast alles gesagt worden, aber diese Studie beleuchtet tatsächlich eine
neue Seite: Die der Bibliotheken und ihrer Kulturen. Hauptaussage ist,
dass Bibliotheken bei gleichen Aufgaben durch ihre jeweils spezifischen,
lokalen Kulturen zu unterschiedlichen Lösungen gelangen.

Untersucht wurden zwei Bibliotheken die am gleichen Projekt arbeiteten
mit Interviews und Feldbesuchen: Ein grosses astronomisches Projekt
hatte über Jahre Daten gesammelt. Beim Projektabschluss wurden von
Forschungsgruppen Möglichkeiten gesucht, diese Daten langfristig
aufzubewahren und weiter zu nutzen. Beiden untersuchte Bibliotheken sind
an Universitäten angesiedelt, welche Forschungsgruppen, die am Projekt
beteiligt waren, beherbergten. Beide gingen -- obwohl es sich um die
gleichen Daten handelt -- unterschiedlich vor: Eine gründete eine
Arbeitsgruppe, die sich aus Kolleg*innen verschiedener Teilbibliotheken
zusammensetzte, die andere hat eine Innovationsabteilung, die sich dem
Projekt annahm. Beide mussten lernen, dass Datenmanagement nicht trivial
ist und die Forschenden unterschiedliche Anforderungen stellen, in
diesem Fall beispielsweise nicht nur Daten vorlagen, sondern auch Daten
über die Veränderung dieser Daten, die nicht als strukturierte
Metadaten, sondern in Mails gespeichert waren.

Neben dem Ergebnis, dass unterschiedliche lokale Kulturen in
Bibliotheken zu unterschiedlichen Lösungen führen, kommt der Artikel zu
dem Schluss, dass Bibliotheken nicht die einzigen Institutionen sind,
welche Datenmanagement übernehmen können: Nachdem die Forschungsgruppen
andere Finanzierungswege gefunden hatten, führten sie das Management
selbst weiter. Die Lösung, die langfristige Archivierung und Pflege von
Daten Bibliotheken zu übergeben, ist nicht alternativlos. (ks)

\begin{center}\rule{0.5\linewidth}{0.5pt}\end{center}

Fenlon, Katrina Simone (2020). \emph{Sustaining Digital Humanities
Collections: Challenges and Community-Centred Strategies}. In:
International Journal of Digital Curation 15 (2020) 1,
\url{https://doi.org/10.2218/ijdc.v15i1.725}

In Digital Humanities-Projekten und in Bibliotheken bestehen
unterschiedliche Vorstellungen davon, was Nachhaltigkeit heisst, wenn es
um Sammlungen geht. Das ist relevant, weil Nachhaltigkeit oft
hergestellt werden soll, indem am Ende von Projekten die erstellten
Sammlungen an Bibliotheken übergeben und von diesen langfristig
unterhalten werden.

Die Autorin des Konferenzbeitrags interviewte Forschende in Digital
Humanities Center und ähnlichen Strukturen und kam zu dem Schluss, dass
bei diesen eine Sammlung immer als lebendig verstanden wird. Eine
Sammlung besteht demnach nicht einfach aus Dokumenten, sondern aus einer
Community, welche die Sammlung durch ihre Arbeit ständig ergänzt und
verändert. Oft sind die Kontakte innerhalb dieser Community wichtiger
als die eigentlichen Dokumente. In Bibliotheken hingegen werden
Sammlungen eher statisch verstanden. Sie können ergänzt werden, aber
innerhalb vorgegebener Sammlungsstrukturen. Diese beiden Verständnisse
schliessen sich nicht unbedingt aus, aber sie erschweren die
Zusammenarbeit. Die Autorin vermerkt auch, dass es nur wenig konkrete
Zusammenarbeit zwischen Digital Humanities-Projekten und Bibliotheken
gibt. Der Grossteil dieser Arbeit an den Sammlungen findet \enquote{am
Rand} bibliothekarischer Arbeit statt, teilweise auf Eigeninitiative
einzelner Bibliothekar*innen, aber kaum strukturiert. Deshalb kommt es
auch immer wieder zu Problemen, wenn es zu einer \enquote{Übergabe} von
Sammlungen kommen soll. Diese ändern, wenn der Prozess erfolgreich ist,
den Status von \enquote{niemals fertigen} Sammlungen hin zu festen und
tendenziell vollständigen Sammlungen. Die Autorin schlägt vor -- aber
eher für die Seite der Digital Humanities -- wie diese Unstimmigkeit
angegangen werden könnte. (ks)

\begin{center}\rule{0.5\linewidth}{0.5pt}\end{center}

McLean, Jaclyn; Dawson, Diane ; Sorensen, Charlene (2021).
\emph{Communicating Collections Cancellations to Campus: A Qualitative
Study}. In: College \& Research Libraries 82 (2021) 1,
\url{https://doi.org/10.5860/crl.82.1.19}

Untersucht wurde, mittels Dokumentenanalyse und halb-strukturierten
Interviews, wie in kanadischen Universitätsbibliotheken die Kündigung
von Big Deals kommuniziert wurde und welchen Erfolg dies hatte. Die
Autor*innen konnten auf eine erstaunlich breite Wissensbasis
zurückgreifen, weil schon viele dieser Bibliotheken Big Deals kündigen
mussten. Grundsätzlich zeigt sich, dass die Kommunikation dieser
Entscheidung an die Universität und die Forschenden möglich ist, aber
dass es sich lohnt, schon weit vorher eine regelmässige Kommunikation
aufzubauen. Bibliotheken, die schon zuvor mit der Universitätsleitung
und den Departements regelmässige Beziehungen hatten, konnten auch die
Entscheidung zum Canceln solcher Deals mit weniger Problemen umsetzen.

Gleichzeitig fiel den Autor*innen auf, dass nur einige Bibliotheken die
Kommunikation über diese Entscheidungen, das heisst die Information über
die grundsätzlichen Probleme des wissenschaftlichen
Kommunikationssystems, mit der Förderung von Alternativen verbanden. Nur
eine Bibliothek (die der Université de Montréal) bezog eine eindeutige
Position dazu, Alternativen zu fördern. In vielen Bibliotheken scheinen
die Angebote im Bereich OA und die Entscheidungen über den Bestand so
weit voneinander getrennt zu sein, dass sie nicht einmal in solchen
\enquote{Krisen} zusammengebracht werden. Dabei würde sich das -- so die
Autor*innen -- in solchen Situationen geradezu anbieten, um den
Forschenden zu zeigen, dass es sich nicht um
\enquote{Bibliotheksprobleme} handelt, sondern um strukturelle Probleme
der Wissenschaftskommunikation. (ks)

\hypertarget{uxf6ffentliche-bibliotheken}{%
\subsection{2.6 Öffentliche
Bibliotheken}\label{uxf6ffentliche-bibliotheken}}

Hebert, Holly S.; Huwieler, Cara (2020). \emph{Exploring Adult Large
Print User Preferences at a Suburban Public Library}. In: Public Library
Quarterly {[}Latest Articles{]},
\url{https://doi.org/10.1080/01616846.2020.1825589} {[}Paywall{]}

In der Public Library einer kleineren US-amerikanischen Stadt wurde eine
Umfrage unter Nutzer*innen von Grossdruckbüchern durchgeführt. (Es wird
auch erwähnt, dass sich diese Bestände in fast allen Public Libraries
finden würden, aber gleichzeitig wenig darüber bekannt sei, wie sie
genutzt werden. Studien gäbe es selten.) Interessant an den Ergebnissen
sind drei Punkte: Zum einen zeigt die Umfrage -- wie auch andere zuvor
--, dass nicht nur Menschen mit Sehbeeinträchtigungen die
Grossdruckbücher nutzen, sondern dass es andere Gründe geben kann, diese
zu lesen. Zweitens zeigte sich nur eine kleine Anzahl der Nutzer*innen
von Grossdruckbüchern (in dieser Bibliothek) an einem Lieferdienst für
Bücher interessiert. Drittens formulierten im Vorfeld der Studie
Bibliothekar*innen, dass sie relativ gut wüssten, wer die
Grossdruckbücher nutzt und was verändert werden müsste. In der Umfrage
zeigte sich jedoch etwas anderes: Die Bibliothekar*innen vermuteten,
dass die Aufstellung der Bücher eine andere sein müsste, weil sie zum
Beispiel schwer zu erreichen wären. Die Befragten waren mit der
Aufstellung aber vollkommen zufrieden. Die Autor*innen betonen, dass es
vielleicht notwendig wäre, die Vorstellungen der Bibliothekar*innen,
dass sie wissen, was die Nutzer*innen wünschen, zu hinterfragen. (ks)

\begin{center}\rule{0.5\linewidth}{0.5pt}\end{center}

DeRosa, Antonio P. , Jedlicka, Caroline ; Mages, Keith C. ; Stribling,
Judy Carol (2021). \emph{Crossing the Brooklyn Bridge: a health literacy
training partnership before and during COVID-19}. In: Journal of the
Medical Library Association 109 (2021) 1,
\url{https://doi.org/10.5195/jmla.2021.1014}

Was übernehmen Öffentliche Bibliotheken alles für Aufgaben? Ein Blick in
die US-amerikanische Literatur zu Public Libraries zeigt immer wieder
eine erstaunliche, aber auch beängstigende Vielfalt an Angeboten in
verschiedensten Bereichen, die oft als normaler Teil der
bibliothekarischen Arbeit dargestellt werden. Es geht dabei nicht darum,
Literatur zu allen möglichen Themen anzubieten, sondern konkrete
Angebote zu machen, weil diese die Aufgabe von Bibliotheken wäre. Es ist
eine Erinnerung daran, dass die Öffentlichen Bibliothekssysteme
verschiedener Länder, und damit auch die Vorstellungen davon, was zur
bibliothekarischen Arbeit gehört, tatsächlich unterschiedlich sind.
Dieser Text beschreibt, wie eine Medizinbibliothek eine Öffentliche
Bibliothek dabei unterstützte, Beratungen zu Gesundheitsthemen
aufzubauen. Die Initiative dazu ging von letzterer aus. Erstaunlich ist
an dem Texte aber, wie selbstverständlich es für beide Bibliotheken zu
sein schien, dass dies eine bibliothekarische Aufgabe wäre. (ks)

\begin{center}\rule{0.5\linewidth}{0.5pt}\end{center}

Sørensen, Kristian Møhler (2020). \emph{Where's the value? The worth of
public libraries: A systematic review of findings, methods and research
gaps.} In: Library \& Information Science Research 43 (2021) 1: 101067,
\url{https://doi.org/10.1016/j.lisr.2020.101067}

Es gibt eine Tradition, nach dem \enquote{Wert} von Öffentlichen
Bibliotheken zu fragen. Diese Studie zeigt, vielleicht ohne das direkt
anzustreben, dass dies schon sehr oft und für sehr verschiedene Werte
getan wurde. Der Autor strebte an, mit einer systematischen
Literaturrecherche zu zeigen, welches Wissen schon vorhanden ist und
welche Forschungsfragen weiter offen sind. Bei dieser Methode werden
möglichst vollständig die thematisch passenden Studien und Artikel
methodisch ausgewertet und zusammengeführt. Was so gezeigt wird, ist,
dass die Frage schon oft angegangen wurde. Der Autor schloss zum
Beispiel die Frage nach dem \enquote{ökonomischen Wert} aus, weil dazu
schon eine andere Metastudie existiert, welche die Ergebnisse anderer
Studien zusammen führte. Am Ende konnte er 42 Studien, die sehr harte
Kriterien erfüllen, auswerten und kam zu dem Ergebnis, dass mehrfach
gezeigt wurde, dass Öffentliche Bibliotheken dabei helfen, dass Menschen
soziales Kapital generieren und das die Gesellschaft funktioniert. Zudem
gäbe es viele Hinweise, aber keine konkreten Nachweise dafür, dass sie
Funktionen als demokratische Einrichtungen übernehmen.

Eher im Nebensatz kommt der Autor auf die interessantere Frage zu
sprechen: Er schlägt vor, es sollte mehr untersucht werden, wie
Bibliotheken diese Werte an die richtigen Stakeholder vermitteln. Das
nämlich ist die eigentliche Frage: So oft wurde untersucht, welchen Wert
Bibliotheken haben, teilweise mit expliziten Ergebnissen in Euros und
Cents, aber was bringt das alles? Reagiert die Politik, die
Öffentlichkeit oder irgendwer anders überhaupt darauf? Ist es das, was
die Wahrnehmung von Bibliotheken durch andere Einrichtungen prägt? (ks)

% \begin{center}\rule{0.5\linewidth}{0.5pt}\end{center}
\pagebreak

Cahill, Maria ; Ingram, Erin (2021). \emph{Instructional Asides in
Public Library Storytimes: Mixed-Methods Analyses with Implications for
Librarian Leadership.} In: Journal of Library Administration 61 (2021)
4, \url{https://doi.org/10.1080/01930826.2021.1906544} {[}Paywall{]}

In den USA -- und anderswo -- gibt es die Tendenz, die regelmässigen
\enquote{Story Time}-Veranstaltungen, bei denen für Kinder in der
Bibliothek vorgelesen wird, zu professionalisieren. Sie sollen zum
Beispiel als Veranstaltungen konzipiert werden, in denen Eltern und
andere Erziehungspersonen lernen sollen, wie sie selbst daheim das
aktive Lesen gestalten können. Dazu wurde unter anderem die Initiative
\enquote{Every Child Ready to Read} der American Library Association
aufgesetzt, bei der unter anderem ein Toolkit inklusive
\enquote{Anleitungen} für Erwachsenen erstellt wurde.

Die Studie untersucht, wie diese Aufgabe in Bibliotheken tatsächlich
umgesetzt wird. Dabei wurden Veranstaltungen beobachtet und
Bibliothekar*innen befragt. Es zeigt sich, dass der Einsatz solcher
Materialien und das Ansprechen von Erwachsenen sehr unterschiedlich
gehandhabt wird. Eine ganze Anzahl von Bibliothekar*innen verweigert
sich praktisch dieser Aufgabe, andere planen sie auch explizit als Teil
ihrer Veranstaltungen ein. Die Autor*innen plädieren für eine weitere
Professionalisierung dieser Aktivitäten, da offenbar das reine Erstellen
des Toolkits nicht ausreichte. Die Praxis in den Bibliotheken müsste
weiter entwickelt werden. Dieser Interpretation muss man nicht
zustimmen, interessant ist aber, dass es offensichtlich möglich ist,
auch eine solche Aktivität wie die \enquote{Story Time} professionell
und auf pädagogische Ziele hin zu planen. (ks)

\begin{center}\rule{0.5\linewidth}{0.5pt}\end{center}

Johnson, Sarah C. (2021). \emph{Innovative social work field placements
in public libraries}. In: Social Work Education {[}Latest Articles{]}
\url{https://doi.org/10.1080/02615479.2021.1908987} {[}Paywall{]}

Der Artikel richtet sich vor allem an Studierende in Sozialer Arbeit und
den Personen in Hochschulen, welche dafür zuständig sind, für deren
Praktika zu sorgen. Er wirbt dafür, diese Praktika in Öffentlichen
Bibliotheken durchzuführen -- ein wenig liest er sich deshalb auch nach
Marketing. Was allerdings von dem Artikel zu lernen ist, ist, dass der
Einsatz von Sozialarbeiter*innen in Öffentlichen Bibliotheken in den USA
und Kanada immer mehr zur Normalität geworden ist, so sehr, dass es
schon eine Social Worker Task Force innerhalb der ALA gibt. Bislang ist
der Trend im DACH-Raum nicht zu beobachten (obwohl es selbstverständlich
schon betreffende Projekte gab), aber vielleicht ist dies nur eine Frage
der Zeit. (ks)

\begin{center}\rule{0.5\linewidth}{0.5pt}\end{center}

Kretz, Chris (2021). \emph{Invited on the Air: Public Librarians at the
Beginning of Broadcast Radio}. In: Journal of Radio \& Audio Media
{[}Latest Articles{]}
\url{https://doi.org/10.1080/19376529.2021.1878183} {[}Paywall{]}

Ein recht unterhaltsamer Einblick in Sendungen von, beziehungsweise mit
Bibliothekarinnen, die in früher \enquote{Radiozeit} (1921-1922) in drei
Sendern in Sutter County (California), Toledo und Pittsburgh auftraten
und dort entweder die Bibliothek und ihre Angebote vorstellten oder für
Kinder vorlasen. Der Artikel baut auf den wenigen Quellen zu diesen
Programmen auf und verortet sie in der US-amerikanischen
Radiogeschichte. Kurz gesagt: In der \enquote{Ausprobierphase}, als
Radiosender noch sehr lokal operierten, galten die Öffentlichen
Bibliotheken offenbar als vertrauenswürdige Einrichtungen, die man
ansprechen konnte, um das Programm mitzugestalten. Diese Möglichkeiten
wurden von Bibliotheken aktiv genutzt. (ks)

\begin{center}\rule{0.5\linewidth}{0.5pt}\end{center}

Eberhard, Milena: \emph{Books for boys only!} In: BuB - Forum Bibliothek
und Information, Vol. 72 (11) 2020, S. 629--631.

Die Grundlage für diesen BuB-Artikel ist eine Studie, die die Autorin im
Rahmen ihrer Masterarbeit durchgeführt hat. In der Stadt- und
Regionalbibliothek Uster (Schweiz), in der die Autorin arbeitet, wurden
Kinder- und Jugendbücher teilweise mit geschlechtsspezifischen Sticker
(Mädchen/Jungen) versehen. Weiterhin gab es Regale mit den
entsprechenden Bezeichnungen. Milena Eberhard fragte sich, warum es
neben der durchaus sinnvollen thematischen Zuordnung auch eine
Aufteilung nach Geschlechtern gab. In ihrer Untersuchung wollte sie
herausfinden, ob sich das Ausleihverhalten der Nutzer*innen verändert,
wenn es keine geschlechtsspezifische Hinweise mehr gibt.

Es wurden knapp 100 neue Bücher beschafft, die keinen
geschlechtsbezogenen Sticker bekamen. Nach acht Monaten wurden die neuen
Daten mit bereits vorhanden Ausleihdaten der letzten fünf Jahre
verglichen. Die Ergebnisse waren eindeutig. Gab es während der
Jungs/Mädchen-Sticker-Ära nur 1--2\,\% \enquote{geschlechtsuntypische}
Ausleihen, waren es nach der Aufhebung der Geschlechter-Zuweisung 20\,\%.
Als Konsequenz wurden in der Bibliothek Uster die
geschlechtsspezifischen Aufkleber abgeschafft.

Auch in vielen anderen Öffentlichen Bibliotheken ist die
Geschlechterzuordnung von Büchern noch gängige Praxis. Sie findet sich
nicht nur bei Kindern- und Jugendbüchern, sondern auch bei Beständen für
Erwachsene. Es ist zu hoffen, dass dieser Artikel und andere
wissenschaftliche Arbeiten zu diesem Thema Bibliotheken anregen, diese
Praxis zu überdenken. (sj)

\begin{center}\rule{0.5\linewidth}{0.5pt}\end{center}

Deeg, Christoph (2020). \emph{Die Bibliothek als Ort der (Retro-)
Gaming-Kultur.} In: Bibliotheksdienst, Bd. 54, H. 5, S. 363--373,
\url{https://doi.org/10.1515/bd-2020-0046}

Die Geschichte des Kulturortes Bibliothek basiert auf dem Medium Buch,
das traditionell ihr zentrales Bezugsmedium war, heute jedoch nur ein
Medium unter vielen ist. Mittlerweile umfassen die Bestände öffentlicher
Bibliotheken eine große Vielfalt von Medienformen, zu denen auch
Computerspiele gehören. Dennoch wird dem Thema Gaming oft wenig
Beachtung geschenkt. Häufig wird es auf die Zielgruppe der Jugendlichen
reduziert. Wo es Spiele im Bibliotheksbestand gibt, sind es eher neuere
Veröffentlichungen. So genannte Pixelgames, also die Klassiker der
Computerspielkultur, finden sich dagegen kaum. Der Autor argumentiert,
dass wenn Bibliotheken gedruckte Literaturklassiker anbieten, auch das
Thema Gaming unter dem Aspekt seiner gesamten geschichtlichen
Entwicklung betrachtet werden müsste. Erstrebenswert wäre es, einen
Bestand für eine generationsübergreifende Gaming-Kultur mit
verschiedenen Zielgruppen anzubieten. Basierend auf diesen Beobachtungen
wird unter anderem diskutiert, warum das Thema Retro-Gaming eine Aufgabe
für Bibliotheken darstellt und ob es sich hierbei um einen Präsenz- oder
Leihbestand handeln sollte. Der Autor präsentiert ein Vier-Säulen-Modell
für die Nutzung von analogen und digitalen Spielen in Bibliotheken als
Orientierung für die Analyse und Weiterentwicklung der eigenen Angebote.
(mk)

\begin{center}\rule{0.5\linewidth}{0.5pt}\end{center}

Williment, Kent (2019). \emph{It Takes a Community to Create a Library.}
In: Public Library Quarterly, Bd. 39, H. 5, S. 410--420,
\url{https://doi.org/10.1080/01616846.2019.1590757} {[}Paywall{]}

Vier Jahre lang arbeitete die Gruppe \enquote{Working Together Community
Development Librarians} in den kanadischen Städten Vancouver, Regina,
Toronto und Halifax in verschiedenen Stadtvierteln und mit diversen
Communities, die traditionell als marginalisiert oder sozial
ausgeschlossen gelesen werden. Ziel des Projektes war ein neues
Planungsmodell für Bibliotheksdienste, welches community-based
konzipiert ist, also von der Gemeinschaft ausgeht. Das Modell sollte
unabhängig von Überzeugungen oder eventuellen Vorurteilen des
Bibliothekspersonals und der existierenden Fachliteratur geschehen. Bei
diesem Ansatz handelt sich um eine neue Methode, die das
Bibliothekspersonal mit den Community-Mitgliedern zusammenbringt, um die
konkreten Bedürfnisse der Community zu ermitteln und zu adressieren. Das
bedeutet, dass auch sozial benachteiligte und ausgegrenzte Personen an
jedem Schritt des Entwicklungsprozesses beteiligt sind. Der Artikel
bietet einen Einblick auch übergreifender Relevanz, da das Modell
flexibel gestaltet ist und in allen Bibliothekseinrichtungen, von allen
Bibliothekar*innen sowie auf alle Programm- und Serviceentwicklungen
angewendet werden kann. Es besitzt demzufolge ein großes Potential, die
Inklusivität von Bibliotheken insgesamt zu steigern. (mk)

\hypertarget{personalmanagement-und-probleme-des-arbeitsalltags-in-bibliotheken}{%
\subsection{2.7 Personalmanagement und Probleme des Arbeitsalltags in
Bibliotheken}\label{personalmanagement-und-probleme-des-arbeitsalltags-in-bibliotheken}}

Wilson, Daniel Earl (2020). \emph{Moving toward
democratic-transformational leadership in academic libraries}. In:
Library Management 41 (2020) 8/9, 731--744,
\url{https://doi.org/10.1108/LM-03-2020-0044} {[}Paywall{]}

Bei diesem Text ist das Thema interessanter als die berichteten
Ergebnisse. Wilson schliesst in ihm an seine Dissertation an, die sich
mit der Effektivität von Führungsstilen in Wissenschaftlichen
Bibliotheken beschäftigte. Darin zeigte er offenbar die Verbreitung
eines \enquote{demokratischen} Führungsstils auf, im Gegensatz -- oder
eher der Erweiterung -- des in der (US-amerikanischen)
bibliothekarischen Literatur oft vorgeschlagenen
\enquote{transformativen} Stils. Der Text berichtet von elf
strukturierten Interviews mit Leitenden von Bibliotheken, welche die
Ergebnisse der Dissertation nochmals untermauern.

Interessant ist einerseits, dass hier überhaupt Überlegungen dazu
angestellt werden, wie Bibliotheken effektiv geführt werden können. Das
ist im DACH-Raum kaum Thema von Diskussion und Forschung, insoweit
stehen viele Führungskräfte wohl für sich allein. Andererseits ist die
Beschreibung der beiden diskutierten Führungsstile bemerkenswert. Beide
reagieren darauf, dass sich die Arbeit in Bibliotheken kontinuierlich
entwickelt und deshalb immer Veränderung gesteuert werden muss. Der
\enquote{transformative} Stil zielt darauf ab, das Personal durch
gesteuerte Weiterbildung, Erläuterungen von Zielen und Kommunikation
dafür zu gewinnen, diese Arbeit zu leisten. Der \enquote{demokratische}
Stil ergänzt dies durch direkte Beteiligung an Entscheidungen und einer
Wertschätzung des gesamten Personals. Laut Wilson zeichnet sich letzters
durch forcierte Partizipation, den Aufbau von Beziehungen zwischen
Leitung und Personal, regelmässiger (am besten auch geplanter) und
ehrlicher Kommunikation, der Gleichbehandlung des Personals und an die
lokalen und individuellen Verhältnisse angepassten Führungspraktiken
aus. Zu bemerken ist, dass hinter beiden Führungsstilen auch jeweils ein
bestimmtes Verständnis davon steht, wie Leitungspersonen ihr Personal
sehen. Der Artikel ist hilfreich, um darüber nachzudenken, wie das
eigentlich in Bibliotheken im DACH-Raum funktioniert und wie es
vielleicht funktionieren sollte. (ks)

\begin{center}\rule{0.5\linewidth}{0.5pt}\end{center}

Colon-Aguirre, Monica ; Webb, Katy Kavanagh (2020). \emph{An exploratory
survey measuring burnout among academic librarians in the southeast of
the United States.} In: Library Management 41 (2020) 8/9, 703--715,
\url{https://doi.org/10.1108/LM-02-2020-0032} {[}Paywall{]}

Dieser Artikel ist deshalb hervorzuheben, weil er ein negatives Ergebnis
berichtet: Die formulierten Hypothesen stellten sich als falsch heraus.
Das ist gute wissenschaftliche Praxis, aber erstaunlich selten finden
sich in der bibliothekarischen Literatur solche Darstellungen.

Die Autorinnen wollten mit einem etablierten psychologischen Instrument
(MBI General Survey) herausfinden, ob -- wie sie vermuteten -- Burnout
unter akademischen Bibliothekar*innen im Südosten der USA verbreitet
ist. Zudem vermuteten sie, dass bestimmte Werte (Geschlecht, ethnic
status, sexuelle Identität) einen Einfluss auf die Verbreitung von
Burnout haben. Im Ergebnis zeigte sich aber, dass Burnout unter den
Befragten wenig verbreitet ist. Einige Faktoren (vor allem sexuelle
Identität) führten zu leicht höheren Werten. Aber im Grossen und Ganzen
scheint Burnout nicht das drängendste Problem bei den Befragten zu sein.
(ks)

\begin{center}\rule{0.5\linewidth}{0.5pt}\end{center}

Barr-Walker, Jill ; Hoffner, Courtney ; McMunn-Tetangco, Elizabeth ;
Mody, Nisha (2021). \emph{Sexual Harassment at University of California
Libraries: Understanding the Experiences of Library Staff Members}. In:
College \& Research Libraries 82 (2021) 2,
\url{https://doi.org/10.5860/crl.82.2.237}

Vollkommen erschreckende Ergebnisse über sexuelle Belästigungen von
Bibliothekar*innen am Arbeitsplatz liefert diese Umfrage an der
University of California. Befragt wurden 1610 Personen an den
Bibliotheken der Universität (mit 10 Standorten)- Von den 579
Antwortenden berichteten 54\,\% davon, sexuelle Belästigung erlebt zu
haben. Dies verteilte sich auf verschiedene Formen von Belästigung, traf
aber an allen Standorten zu. Schlimm sind auch die Rückmeldungen, dass
in vielen Fällen die Institution die Betroffenen nicht unterstützte oder
die Betroffenen selbst den Eindruck hatten, sich nicht hilfesuchend an
Personen wenden zu können. Es gab Lichtblicke, insbesondere ,dass viele
-- aber nicht alle -- Leitungen der Bibliotheken das Problem ernst
nahmen. Allerdings, so die Autorinnen, hat dies bislang nicht dazu
geführt -- wie auch regelmässige Fortbildungen zum Thema -- dass es
keine sexuelle Belästigung mehr gäbe.

Sicherlich ist das eine Untersuchung für ein US-amerikanisches
Bibliothekssystem (allerdings in einem als fortschrittlich
wahrgenommenen Bundesstaat und an Hochschulen, die einen ebenso
fortschrittlichen Ruf haben -- was für Bibliotheken in anderen
Bundesstaaten Schlimmes befürchten lässt). Aber es ist einfach nicht von
der Hand zu weisen, dass ähnliche Ergebnisse auch in anderen
Bibliotheken in anderen Staaten zu finden wären. (ks)

\hypertarget{automatisierung-und-kuxfcnstliche-intelligenz}{%
\subsection{2.8 Automatisierung und Künstliche
Intelligenz}\label{automatisierung-und-kuxfcnstliche-intelligenz}}

Hänßler, Boris: \emph{Service ohne Menschen: Heilsbringer oder
Heimsuchung: Über das widersprüchliche Verhältnis von Mensch und
Maschine.} In: BUB: Forum Bibliothek und Information 2--3 / 2018 S.
90--95

Automatisierung betrifft längst nicht mehr \enquote{nur} die
produzierende Industrie, sondern auch immer mehr den
Dienstleistungssektor, stellt der Autor Boris Hänßler in seinem Beitrag
fest. Und damit betrifft sie auch Bibliotheken. Im Zeitalter der vierten
Revolution (nach der Dampfkraft, Elektrizität und Digitalisierung)
übernehmen Maschinen immer komplexere Aufgaben und betreten durch
Künstliche Intelligenz weitere Bereiche der Arbeitswelt, deren
Ausführung bisher dem Menschen vorbehalten war. Maschinen sollen die
Menschen nicht mehr nur körperlich, sondern auch geistig entlasten.
Dadurch entsteht zweifelsohne auch ein konkurrierendes Verhältnis
zwischen Mensch und Maschine. Der Autor wirft die Frage auf, ob
Maschinen die Menschen aus der Arbeitswelt verdrängen werden, da sie
kostengünstiger, schneller und effizienter arbeiten, als dies ein Mensch
jemals in der Lage wäre.

Global agierende Unternehmen, die sich am Primat des Kapitalismus
orientieren, investieren jedenfalls immense Summen in die
(Weiter-)Entwicklung von Automatisierung in der Hoffnung auf Zeit- und
Kostenersparnis. Immer mit der Intention -- so jedenfalls die
öffentliche Stellungnahme der Unternehmen -- die Belegschaft zu
unterstützen und ihnen die Arbeit zu erleichtern, damit diese sich auf
komplexere Aufgaben konzentrieren kann.

Das McKinsley Global Institut hat eine Studie veröffentlicht, nach der
bis 2030 geschätzte 800 Millionen Menschen weltweit ihre Jobs an
Maschinen verlieren werden. Es gilt abzuwarten, ob sie mit dieser
Einschätzung Recht behalten. (dg)

 \begin{center}\rule{0.5\linewidth}{0.5pt}\end{center}

Vecera, Emanuel: \emph{Künstliche Intelligenz in Bibliotheken}. In:
Information - Wissenschaft \& Praxis, Band 71 Heft 1, 14.01.2020,
\url{https://doi.org/10.1515/iwp-2019-2053}

Der Artikel fragt: Wie kann Künstliche Intelligenz (KI) in Bibliotheken
eingesetzt werden? Wo gibt es Vorteile und wo Nachteile? Die
Einsatzbereiche von KI erstrecken sich vom Feld der automatischen
Indexierung bis hin zur Auskunft über Leihverhalten und Vorlieben der
Nutzenden. \enquote{Auch der Zugang zu Literatur kann durch KI
revolutioniert werden, wie beispielsweise durch den Aufbau einer
KI-gestützten Infrastruktur}. Die größte Chance beim Einsatz der KI wird
hier als die Steigerung der \enquote{Verfügbarkeit, Qualität, Quantität
und Schnelligkeit von Informationsdienstleistungen} beschrieben.
Dennoch, so betont der Autor, bleiben Bibliothekarin und Bibliothekar
auch zukünftig unersetzbar. (lf)

\begin{center}\rule{0.5\linewidth}{0.5pt}\end{center}

Vecera, Emanuel (2020): \emph{Künstliche Intelligenz in Bibliotheken}.
In: Information - Wissenschaft \& Praxis 71 (1), S. 49--52,
\url{https://doi.org/10.1515/iwp-2019-2053}

In seinem Artikel diskutiert Emanuel Vecera den Einsatz der künstlichen
Intelligenz (KI) in Bibliotheken. Hierbei stellt er fest, dass der
Einsatz von KI mehrere Anwendungsbereiche betrifft. Beispielsweise wird
KI in Expertensystemen verwendet, um Handlungsregeln und Problemlösungen
festzuhalten. Insbesondere wird hier der Fokus auf die Automatisierung
der Prozesse im Bereich der Katalogisierung gesetzt. Des Weiteren wird
KI auch dazu genutzt, um beispielsweise den Prozess der Indexierung
qualitativ zu verbessern und den Zugang zur Literatur durch eine neue
Art und Weise zu verändern. Auch diskutiert der Autor den bereits
anzutreffenden Einsatz von Robotern in Bibliotheken. Die Bibliothek des
Max Planck Institute Luxembourg for International, European and
Regulatory Procedural Law arbeitet beispielsweise gerade an der
Entwicklung eines Roboters, der zur Prüfung des Inventars verwendet
werden soll, um so den zeitaufwendigen Prozess zu automatisieren. Auch
wird der Einsatz von Robotern in anderen Bibliotheken diskutiert und
verglichen. Seit geraumer Zeit hat die Stadtbibliothek Köln einen
Roboter mit den Namen Nao im Einsatz. Dieser hilft dabei, den Ablageort
bestimmter Bücher zu identifizieren und beantwortet Fragen sowohl in
englischer als auch deutscher Sprache. (kmg)

\hypertarget{monographien-und-buchkapitel}{%
\section{3. Monographien und
Buchkapitel}\label{monographien-und-buchkapitel}}

\hypertarget{vermischte-themen-1}{%
\subsection{3.1 Vermischte Themen}\label{vermischte-themen-1}}

D'Ignazio, Catherine; Klein, Lauren F. (2020). \emph{Data Feminism.}
Cambridge, Massachusetts: The MIT Press (Ideas series).
\url{https://data-feminism.mitpress.mit.edu/}.

Die Autorinnen erklären in Data Feminism anhand zahlreicher Beispiele,
warum sich eine feministische Perspektive auf die Arbeit mit Daten
lohnt. In der Einleitung zeichnen sie die, oft erst in den letzten
Jahren bekannt gewordene, Geschichte der Pionierinnen der
Datenverarbeitung nach. Ihre Definition von Feminismus lautet:
\enquote{t{[}T{]}he term feminism as a shorthand for the diverse and
wide-ranging projects that name and challenge sexism and other forces of
oppression, as well as those which seek to create more just, equitable,
and livable futures} (D'Ignazio, Klein, 2020, S. 6). Die Autorinnen
zeigen auf, wie notwendig eine möglichst vielschichtige Perspektive ist,
um Forschung im Data-Bereich möglichst inklusiv zu gestalten, also so,
dass alle Beteiligten und möglichst alle Betroffenen abgebildet werden.
Das Buch besteht aus sieben Kapiteln und jedes Kapitel ist einem der
Prinzipien des Data Feminismus gewidmet. Dies wird von Beispielen aus
diversen Fachgebieten begleitet, wodurch die Breite des Spektrums für
die Anwendung von Data Feminismus sichtbar wird. Für die Bibliotheks-
und Informationswissenschaft ist dieses Thema insofern relevant, dass
Daten für die Disziplin eine wichtige Forschungsgrundlage darstellen.
Auch in der Bibliotheks- und Informationswissenschaft sollte man an
Vielfalt und Diversität der NutzerInnen beispielsweise von
Bibliothekskatalogen, Webseiten und natürlich Bibliotheken selbst
denken. Gleiches gilt für die Reflexion darüber, woher verwendete Daten
kommen, wer sie wie erhoben hat, wer befragt wurde und wer vielleicht
vergessen wurde. Ähnliche Fragen sollte man sich auch stellen, wenn man
die Daten selbst erhebt. Wie die Autorinnen betonen, sollten wir nie
vergessen, dass auch unsere eigene Perspektive zwangsläufig eine
eingeschränkte ist. (sj)

\begin{center}\rule{0.5\linewidth}{0.5pt}\end{center}
\pagebreak

Zweig, Katharina A. (2019). \emph{Ein Algorithmus hat kein Taktgefühl.
Wo künstliche Intelligenz sich irrt, warum uns das betrifft und was wir
dagegen tun können.} München: Heyne, 2019. {[}gedruckt{]}

Im Buch beschreibt Katharina Zweig, Informatikprofessorin an der TU
Kaiserslautern, anschaulich, wie maschinelles Lernen, Algorithmen und
Künstliche Intelligenz funktionieren. Sie zeigt dabei auf, wann und
warum es bei solchen Entwicklungen notwendig ist, nicht nur
Datenspezialist*innen zu beauftragen. Wichtig ist die Einbindung von
Expert*innen aus anderen Bereichen, die sich unter anderem ethischen
Fragen zuwenden und sicherstellen, dass das Endprodukt genau das tut,
was es soll, ohne diskriminierend zu sein. Außerdem führt die Autorin
aus, wo und wann in der Entwicklung neuer Systeme die Gesellschaft
mitbestimmen kann und sollte. Das Buch ist wichtig, um zu verstehen,
dass Algorithmen und auf maschinelles Lernen begründete Künstliche
Intelligenz nicht automatisch neutral und wertfrei sind. Es klärt
darüber auf, dass argloses Vertrauen in die Technik nicht nur falsch,
sondern auch gefährlich sein kann, zum Beispiel dann, wenn Maschinen
beginnen, Entscheidungen zu fällen, die Menschen betreffen. Damit
berührt es ein Thema, mit dem sich auch die Bibliotheks- und
Informationswissenschaft auseinandersetzen muss, da sich
Informationswissenschaftler*innen oft an der Schnittstelle von
Entwicklung und Nutzung derartiger Systeme befinden. (sj)

\begin{center}\rule{0.5\linewidth}{0.5pt}\end{center}

Schomberg, Jessica ; Highby, Wendy (2020). \emph{Beyond Accommodation:
Creating an Inclusive Workplace for Disabled Library Workers}.
Sacramento: Library Juice Press, 2020 {[}gedruckt{]}

Die beiden Autorinnen sind Bibliothekarinnen in den USA und leben mit
Behinderungen. Das Buch ist -- auf der Basis ihrer eigenen Erfahrungen
und derer von Kolleg*innen in ähnlicher Situation, die sie interviewt
haben -- in erster Linie für Personen in der gleichen Situation
geschrieben. Es diskutiert, was Behinderung bedeutet -- also dass sie
keine rein persönliche erlebte oder rein medizinisch zu erklärende,
sondern eine soziale Situation ist -- und versucht, Wissen zu
vermitteln, wie sich der Arbeitsalltag in einer solchen Situation
gestalten lässt. Dabei geht es zuerst um Ermächtigung der Betroffenen,
nicht zum Beispiel um ein Handbuch für Bibliotheken selber (wie das der
Titel andeutet). Dies alles vor US-amerikanischem Hintergrund,
beispielsweise der dortigen Gesetzeslage, aber auch solcher
gesellschaftlichen Absonderlichkeiten wie fehlender Krankenversicherung
bei Jobverlust oder beschränkten \enquote{sick days}. Für andere
Personen ist es ein Buch zum Zuhören -- anstatt gleich zu Lösungen zu
springen, lässt sich aus ihm viel über den Alltag und die
Herausforderungen beim Arbeiten von Personen mit Behinderung in
Bibliotheken erfahren. (ks)

\begin{center}\rule{0.5\linewidth}{0.5pt}\end{center}

Touitou, Cécile (dir.) (2020). \emph{Bibliothèques publiques
britanniques contemporaines: autopsie des années de crise.} {[}La
Numérique{]} Villeurbanne: Presses de l'enssib, 2020,
\url{https://doi.org/10.4000/books.pressesenssib.11527}

Die Öffentlichen Bibliotheken in Grossbritannien sind seit Jahren in
einer profunden Krise. Einst Vorbild für Bibliothekssysteme anderer
Länder müssen britische Bibliotheken heute oft konkret um ihr Überleben
fürchten. (Es liegt nahe, wird in diesem Buch aber nicht gemacht, dies
mit der aktuellen Krise Grossbritanniens in Verbindung zu setzen.) Das
Buch möchte die Gründe für diese Krise darlegen. Neben Texten, die neu
für diese Publikation verfasst wurden, wurden dafür Texte von britischen
Aktivist*innen für Bibliotheken übersetzt.

Grundsätzlich wird der Grund der Krise in der Regierung von
Konservativer und Liberal-Demo\-kratischer Partei (2010--2015) und
insbesondere im Regierungschef David Cameron (in dieser Position:
2010--2016) gesehen. Die Regierung war mit einem Programm angetreten,
welches das nationale Budget ausgleichen, staatliche Aufgaben an
Gemeinden und \enquote{die Gesellschaft} (verstanden vor allem als
Wohlfahrtsorganisationen) übertragen sollte sowie gleichzeitig die
Bedeutung ökonomischer Prinzipien in der öffentlichen Verwaltung
etablieren wollte. Dies führte dazu, dass unter anderem Bibliotheken
unter neuen Kriterien bewertet, geschlossen und an nicht-staatliche
Organisationen übergeben wurden. Nach dem Rücktritt David Camerons gab
es im Land rund 340 Öffentliche Bibliotheken und 8.000
Bibliothekar*innen weniger. Das Buch stellt diese Entwicklungen dar und
bietet auch den Protesten gegen diese Schliessungen -- die zahlreich und
von verschiedenen Gruppen getragen waren -- viel Platz. Insbesondere für
die Proteste werden zahlreiche Beiträge aus diesen Jahren direkt
zitiert.

Einerseits ist es richtig zu fragen, wie die britischen Public Libraries
in die Krise geraten sind. Andererseits hinterlässt das Buch doch den
Eindruck, einiges auszulassen. Es wird zwar diskutiert, dass in diesen
Jahren tatsächliche Veränderungen -- zum Beispiel technische und solche
in der Mediennutzung -- stattfanden, die nichts mit der Regierung zu tun
hatten. Aber am Ende wird doch der Eindruck erzeugt, als wäre es das
Government alleine gewesen, welches die Krise produziert hätte. Aber so
einfach wird die Situation nicht sein, auch vorher gab es
gesellschaftliche Probleme -- ansonsten wäre die vorhergehende Regierung
ja nicht abgewählt worden --, eine Sinnsuche bei britischen Bibliotheken
und zahlreiche Aufforderungen, zum Beispiel von Seiten der
Blair-Regierung, an die Bibliotheken, sich zu ändern. Diese
Vorgeschichte wird überhaupt nicht diskutiert. Nicht zuletzt legt das
Buch Wert darauf, die ganzen Gruppen, die sich gegen die Schliessung von
Bibliotheken engagierten (Parteien, Zivilgesellschaft, Autor*innen und
andere) und deren Argumente ausführlich darzustellen. Aber am Ende haben
diese zwar Teilerfolge vorweisen können, doch auf breiter Strecke
verloren. Was ist dann der Erkenntniswert dieser Darstellungen? Wäre es
nicht sinnvoller darüber nachzudenken, warum diese ganzen Aktivitäten
wenig Erfolg hatten? (ks)

\begin{center}\rule{0.5\linewidth}{0.5pt}\end{center}

Grailles, Bénédicte; Marcilloux, Patrice; Neveu, Valérie; Sarrazin,
Véronique (dir.) (2018). \emph{Les dons d'archives et de bibliothèques:
XIX\textsuperscript{e}-XXI\textsuperscript{e} siècle. De l'intention à
la contrepartie}. (Collection \enquote{Histoire}) Rennes Cedex: Presse
universitaires de Rennes, 2018 {[}gedruckt{]}

Warum übergeben Menschen Sammlungen an Archive und Bibliotheken? Die die
konkreten Beiträge dieses Sammelwerks umschliessenden Texte (Vorwort,
Einleitung, Nachwort) betonen eine ethnographische Perspektive,
insbesondere mit Bezug auf Marcel Mauss, der in den 1920ern zur
Bedeutung von Geschenken geforscht hatte (allerdings nicht in
europäischen Gesellschaften). Geschenke hätten immer Bedeutung für die,
die geben; für die, die nehmen und für den Zusammenhalt von
Gemeinschaften. Durch sie werden Beziehungen hergestellt,
Verpflichtungen generiert und auch eingelöst. Im Buch soll diese
Erkenntnis auf heutige Archive und Bibliotheken (vor allem in
Frankreich, mit je einem Beispiel aus der Schweiz und den USA)
angewendet werden. Das findet leider kaum statt, obwohl es interessant
wäre.

Vielmehr finden sich hier Beiträge, in denen vor allem Archivar*innen
und einige Bibliothekar*innen darüber berichten, wie in ihren
Einrichtungen mit solchen Sammlungen umgegangen, wie mit Spender*innen
kommuniziert und wie Zugang zu den gespendeten Sammlungen ermöglicht
wird. Dabei geht es unter anderem auch um nicht traditionelle
Institutionen oder Themen. Das ist alles interessant und rückt vor allem
in den Blick, dass es sich nicht einfach nur um Dokumente handelt, die
in archivalische oder bibliothekarische Arbeitsgänge integriert werden,
sondern um teilweise komplexe Beziehungen. Der Blick der Spender*innen
wird aber nur von einer Person vertreten (Hélène Mouchard-Zay). Die
eigentliche Frage, was Spender*innen von ihren Spenden erwarten, wie sie
von ihnen profitieren und welche Beziehungen zwischen Institution und
Spender*in etabliert werden, werden so aber nicht beantwortet, sondern
es werden eher einseitig die Eindrücke der Institutionen dazu gesammelt.
(ks)

\begin{center}\rule{0.5\linewidth}{0.5pt}\end{center}

Haucap, Justus; Moshgbar, Nima; Schmal, Wolfgang Benedikt (2021).
\emph{The Impact of the German \enquote{DEAL} on Competition in the
Academic Publishing Market.} {[}CESifo Working Paper No.~8963{]}
München, Munich Society for the Promotion of Economic Research, 2021,
\url{https://ssrn.com/abstract=3815451}.

Die Autoren untersuchen das Publikationsverhalten deutscher Autorinnen
und Autoren im Fach Chemie im Zeitraum 2016--2020. Da in
Fachzeitschriften der Chemie vergleichsweise schnelle
Publikationsprozesse vorherrschen, wird dieses Fach als geeignet
angesehen, um frühe Effekte der DEAL-Vereinbarungen auf den
Zeitschriftenmarkt sichtbar zu machen. Im Ergebnis wird ein kleiner,
aber statistisch signifikanter Effekt festgestellt: Nach Inkrafttreten
der DEAL-Verträge publizieren die Angehörigen deutscher
Forschungseinrichtungen etwas häufiger bei den Verlagen Springer und
Wiley -- und entsprechend etwas weniger bei anderen Verlagen. Die
Autoren empfehlen, auch mit kleineren Verlagen deutschlandweite
Read-and-Publish-Verträge abzuschließen, da sonst die Marktmacht der
Großverlage und die Konzentration auf dem Zeitschriftenmarkt auf Kosten
der Publikationsvielfalt und schließlich auch zu Lasten der Bibliotheken
weiter voranschreiten dürfte. (eb)

\begin{center}\rule{0.5\linewidth}{0.5pt}\end{center}

Barlösius, Eva (2019). \emph{Infrastrukturen als soziale
Ordnungsdienste: Ein Beitrag zur Gesellschaftsdiagnose}. Frankfurt am
Main: Campus Verlag, 2019 {[}gedruckt{]}

Die Autorin untersucht in diesem Buch, wie sich Infrastrukturen
entwickeln und ob diese Entwicklungen als Werkzeug für die Analyse
gesellschaftlicher Veränderungen eingesetzt werden können. Die Frage ist
also soziologisch, der Text selber verarbeitet Ergebnisse aus
vorhergehenden Studien unter einem theoretischen Blickwinkel. Die Frage
drängt sich aber auf, weil viele Infrastrukturen mit der Entstehung der
Nationalstaaten im 19. Jahrhundert verbunden waren (beispielsweise
Eisenbahnen, allgemeines Bildungswesen, staatliche Normungen), sich
diese Staaten aber seitdem massiv verändert haben.

Für das Bibliothekswesen relevant ist, dass einer der vier Fälle, die
Barlösius als Beispiel untersucht, die Forschungsinfrastrukturen sind,
welche in den letzten Jahren etabliert wurden beziehungsweise weiter
werden. Sie beschreibt dabei die Veränderungen aus soziologischer Sicht
als \enquote{Infrastrukturierung} von bislang als Teil der
wissenschaftlichen Arbeit angesehenen Tätigkeiten. Diese würden als
Infrastruktur anderen, klarer formulierten Regeln und Zielen
unterworfen, zudem öffentlicher gemacht. In gewisser Weise würde hier im
Bereich Wissenschaft Aufgaben \enquote{verstaatlicht}, während in
anderen Bereichen Infrastruktur \enquote{entstaatlicht} würde. Es ist
ein analytischer Blick auf Veränderungen, die im Bibliothekswesen sonst
eher mit einem praxisorientierten Fokus bearbeitet werden. (Die
Ergebnisse der Studien, auf die sich die Autorin bei diesem Fall stützt,
hat sie auch gesondert in der bibliothekarischen Literatur publiziert,
aber ohne diesen expliziten Fokus.) (ks)

\hypertarget{rassismus-und-dekolonisierung}{%
\subsection{3.2 Rassismus und
Dekolonisierung}\label{rassismus-und-dekolonisierung}}

Mätschke, Jens (2017): \emph{Rassismus in Kinderbüchern. Lerne, welchen
Wert deine soziale Positionierung hat!} In: Karim Fereidooni und Meral
El (Hg.): Rassismuskritik und Widerstandsformen. Wiesbaden: Springer
Fachmedien Wiesbaden, S. 249--268.
\url{https://doi.org/10.1007/978-3-658-14721-1_1} {[}Paywall{]}

Jens Mätschke analysiert Kinder- und Jugendbücher auf rassistische
Inhalte und klassifiziert diese mittels eines eigens erstellten
Kategoriensystems, das er im Rahmen seiner Bachelorarbeit zum Thema
\enquote{Rassismus in Comics der DDR} entwickelt hat. Es werden
Beispiele aus einer Vielzahl von Kinderbüchern genannt, bei denen die
Möglichkeit besteht, dass Kinder durch die vermeintlich rassifizierte
Darstellung der Geschichte ein hierarchisches Wertemodell erwerben, das
eine gleichberechtigte Begegnung zwischen \enquote{schwarz} und
\enquote{weiß} im Kopf erschwert.

Der Beitrag vermittelt einen Einblick in das dreizehnteilige
Kategorie-System, mit dem Jens Mätschke rassistische Zuschreibungen für
\enquote{Schwarze} benennt und analysiert. Er erläutert diesen Ansatz
anhand des Klassikers \enquote{Robinson Crusoe} (2002) und eines
aktuelleren Kinderbuchs mit dem Titel \enquote{Vimala gehört zu uns}
(2002). Die Vergleiche sind nachvollziehbar und begründet. Die
Konfrontationen regen zum Nachdenken darüber an, wie Geschichten mit
rassifizierten Darstellungen in Kinderbüchern wirken und welche Rolle
AutorInnen, PädagogInnen, Eltern, Verlage und die Gesellschaft
übernehmen sollten. (gs)

\begin{center}\rule{0.5\linewidth}{0.5pt}\end{center}

Garcês da Silva, Franciéle Carneiro (2019): \emph{Com a palavra, as/os
professoras/es. A formação da/o docente em biblioteconomia para a
inclusão das temáticas africana e afro-brasileira na prática docente.}
In: Danielle Barroso, Elisângela Gomes, Erinaldo Dias Valério, Garcês da
Silva, Franciéle Carneiro und Graziela dos Santos Lima (Hg.):
Epistemologias Negras: relações raciais na biblioteconomia.
Florianópolis: Rocha Gráfica e Editora, S. 139--174. {[}gedruckt{]}

Für ihren Beitrag interviewte Franciéle Carneiro Garcês da Silva 13
brasilianische Dozent:innen der Bibliothek- und Informationswissenschaft
(BIW). Thema der Gespräche war die (mangelnde) Inklusion von
afrikanischen und afro-brasilianischen Themen sowie die Nutzung von
Quellen von afro-diasporischen Wissenschaftler:innen an Universitäten.

Franciéle stellt heraus, wie die unzureichende Ausbildung von
Bibliothekar:innen und Informationswissenschaftler:innen in Brasilien zu
diesen Themen einen Mangel an afrikanischen und afro-brasilianischen
Themen und Inhalten nach sich zieht. Die Gründe dafür sind unter anderem
fehlende Sensibilisierung und fehlendes Wissen über Antirassismus und
Dekolonialität auf Seiten der Dozent:innen. Für Franciéle kann das unter
anderem daran liegen, dass unkritisch das US-amerikanische Lehrprogramm
und damit entsprechende Perspektiven von Whiteness und ihrer impliziten
Normativität übernommen werden. Sie betont, dass das Lehrprogramm der
BIW betont wertneutral und technisch ausgerichtet wird und zugleich
Wissensformen, die nicht dieser Norm der Whiteness entsprechen, nicht
berücksichtigt.

In den Interviews merken viele Dozent:innen an, dass sie zu wenig Zeit
für eine Aktualisierung der Kurse haben, dass das Lehrprogramm für
Einbindung anderer Wissensformen zu unflexibel ist oder das Fach zu
technisch sei. Dem stellt Franciéle vergleichsweise leichte Lösungen
gegenüber. Zum Beispiel listet sie schwarze Bibliotheks- und
Informationswissenschaftler*innen auf, die auch die sogenannten
technischen Themen der BIW erforschen, die es nur sichtbar zu machen
gilt.

Übergreifend betont Franciéle, wie wichtig und dringend es ist, dass
sich Dozent:innen der Vielfalt der Studierenden anpassen sollten, vor
allem in dem sie deren Geschichten und Kulturen aufnehmen und
respektieren. (vt)

\begin{center}\rule{0.5\linewidth}{0.5pt}\end{center}

Garcês da Silva, Franciéle Carneiro (2020). \emph{Perspectivas Críticas
e Epistemologias Negras na Biblioteconomia.} In: Natalia Duque Cardona
und Garcês da Silva, Franciéle Carneiro (Hg.): Epistemologias
Latino-Americanas na Biblioteconomia e Ciência da Informação.
Contribuições da Colômbia e do Brasil. Florianópolis: Rocha Gráfica e
Editora, S. 73--117. {[}gedruckt{]}

In ihrem Kapitel betont Franciéle Carneiro Garcês da Silva die Bedeutung
der Inklusion vielfältiger Erkenntnistheorien in die Bibliotheks- und
Informationswissenschaft (BIW). Zugleich kritisiert sie die Hegemonie
des Wissens nach kolonialen Standards. Sie stellt dafür unter anderem
kritische Erkenntnistheorien und Bewegungen des Feldes vor, unter
anderen Guerrilla Librarianship, Radical Librarians Collective, Critical
Whiteness Studies in Library and Information Science, Black
Librarianship und Biblioteconomia Negra Brasileira (übersetzt:
Schwarz-Brasilianische Bibliothekswissenschaft). Zudem verweist sie auf
schwarze sowie indigene brasilianische Bibliothekar:innen, die in
unterschiedlichen Feldern der Bibliotheks- und Informationswissenschaft
forschen. Die Beforschung dekolonialer Erkenntnistheorien ist für die
Dekonstruktion der von ihr kritisierten Hegemonien essentiell. Eine
Zusammenfassung, wie sie der vorliegende Text liefert, bildet daher eine
notwendige Voraussetzung sowohl für das kritische Hinterfragen dieser
Hegemonien als auch eine Orientierung zum Thema für Dozent:innen und
Bibliothekar:innen unverzichtbar, die in diesem Bereich arbeiten. (vt)

\hypertarget{einfuxfchrungen-und-handbuxfccher}{%
\subsection{3.3 Einführungen und
Handbücher}\label{einfuxfchrungen-und-handbuxfccher}}

Ferguson, Lea Maria; Pampel, Heinz; Bruch, Christoph; Bertelmann,
Roland; Weisweiler, Nina; Schrader, Antonia C.; Messerschmidt, Reinhard;
Faensen, Katja (2020). \emph{Gute (digitale) wissenschaftliche Praxis
und Open Science: Support und Best Practices zur Umsetzung des DFG-Kodex
``Leitlinien zur Sicherung guter wissenschaftlicher Praxis.} (Helmholtz
Open Science Briefing), Potsdam : Helmholtz Open Science Office.
\url{https://doi.org/10.2312/os.helmholtz.012}.

Was haben die Sicherung guter wissenschaftlicher Praxis und Open Science
miteinander zu tun? Dieser Frage geht eine Handreichung des Helmholtz
Open Science Office nach. Dabei werden einzelne Leitlinien des DFG-Kodex
systematisch beleuchtet, der jeweilige Bezug zu Open Science hergestellt
und Empfehlungen für den Forschungsalltag ausgesprochen, die sich
zuvorderst an Helmholtz-Zentren richten, aber für andere
Wissenschaftseinrichtungen ebenso relevant sein dürften. Ergänzend liegt
auch eine Checkliste vor, die die insgesamt 16 Empfehlungen im Überblick
zusammen stellt. (mv)

\begin{center}\rule{0.5\linewidth}{0.5pt}\end{center}

Lenstra, Noah (2020). \emph{Healthy Living at the Library: Programs for
all Ages}. Santa Barbara: Libraries Unlimited, 2020 {[}gedruckt{]}

Libraries Unlimited ist einer der Verlage der American Library
Association, in ihm wird vor allem praxisorientierte bibliothekarische
Fachliteratur verlegt. In den letzten Jahren ist der Verlag dazu
übergegangen, eine Reihe von Büchern zu verlegen, die zwar nicht
offiziell als Reihe geführt werden, aber es praktisch sind: Jeweils wird
im Titel ein interessantes, zeitgemässes Thema als eines versprochen,
welches auch (Öffentliche) Bibliotheken angehen würden. Die Publikation
ist immer 180-220 Seiten lang und von jemandem aus der Praxis
geschrieben. Auch der Aufbau ist stets ähnlich: Kapitel werden zum
Beispiel oft mit Fragen abgeschlossen, die man sich als praktizierende*r
Bibliothekar*in stellen soll. Viele Beispiele zum Thema sind in Kästen
im Text dargestellt. Die ordnende Hand des Verlages ist in diesen
Büchern unübersehbar.

Inhaltlich gibt es in Büchern dieser Reihe durchgehend die gleiche
Argumentation: (1) Zuerst wird postuliert, dass jeweilige Thema sei
wichtig für die Gesellschaft und wenn sich Bibliotheken ihm annehmen
würden, würden sie Communities bilden und neue Nutzer*innen erreichen,
(2) dann wird oft gezeigt, dass Bibliotheken in der Vergangenheit sich
schon mit diesem Thema auseinandergesetzt hätten, (3) drittens wird
gesagt, Bibliotheken müssten sich auf das jeweilige Thema einlassen,
einen Plan für neue Angebote in diesem Bereich aufstellen und
Kooperationen eingehen, (4) im längsten Teil des jeweiligen Buches
werden dann zahllose Beispiele aus Bibliotheken zusammengetragen, die
sich schon mit dem Thema beschäftigen und (5) am Ende betont, dass das
jeweilige Thema nachhaltig abgesichert werden muss, damit es sich
wirklich in einer Bibliothek etablieren kann, beispielsweise in
Bibliotheks- oder anderen Strategien integriert.

Das genannte Buch ist nur eines der neueren Publikationen, das genau
diesem Aufbau folgt. Es hinterlässt den gleichen schlechten Geschmack
wie andere Bücher in der Reihe: Auf den ersten Blick scheint das Thema
sympathisch (wer hat schon etwas dagegen, wenn die Gesundheit anderer
gefördert wird?), ebenso die einzelnen Beispiele aus Bibliotheken (Urban
Gardens, Yoga-Klassen, Kochkurse und vieles mehr -- warum nicht?). Aber
in seiner Masse wirkt das alles nicht mehr: Wenn immer wieder neue
Themen als überaus wichtig bezeichnet und das -- wie auch hier -- mit
heranzitierten Beispielen einzelner Studien und Aussagen herausgehobener
Einzelpersonen untermauert wird, wirkt das schnell so, als könnte
einfach jedes beliebige Thema so verkauft werden. Auch die Ansammlung
der Beispiele in solchen Büchern vermittelt schnell den Eindruck, dass
es eigentlich egal ist: Alle sollen halt tun, was sie selber als wichtig
ansehen. Es gibt noch nicht einmal eine Gewichtung der Beispiele. Noch
weniger gibt es wirklich Hilfestellungen für die Adaption in einen
lokalen Kontext, vielmehr wird -- in diesem Buch und in praktisch allen
anderen dieser Reihe -- schnell betont, dass es keine allgemeingültige
Lösung gäbe, sondern dass Bibliotheken jeweils lokal entscheiden
müssten, was sie tun wollen. Die Hinweise zur Umsetzung sind fast immer
oberflächlich.

Es stellt sich die Frage, für wen diese Bücher eigentlich geschrieben
werden. Beispiele aus Bibliotheken zu den unterschiedlichen Themen
lassen sich auch so zusammensuchen, wenn eine Bibliothek dies anstrebt.
Wirkliche Hilfestellung beim Entwickeln neuer Angebote liefern die
Bücher nicht, weil sie so beliebig argumentieren und eher Beispiele
anhäufen. Vielmehr scheint es sich um reine Verlagsentwicklungen zu
handeln, bei der die Autor*innen je einen schon fertigen Rahmen
auszufüllen haben. Sicherlich sind viele Autor*innen persönlich an den
jeweiligen Themen interessiert (auch bei diesem Buch), aber in dieser
Form nutzt das leider wenig. Von allen Büchern dieser inoffiziellen
Reihe ist abzuraten. (ks)

\begin{center}\rule{0.5\linewidth}{0.5pt}\end{center}

Armstrong, Alison M.; Dinkle, Lisa (2020). \emph{The library liaison's
training guide to collection management}. Chicago: ALA Editions.
\url{https://www.alastore.ala.org/content/library-liaisons-training-guide-collection-management}
{[}gedruckt{]}

Bestandsmanagement gehört ja zu den \enquote{klassischen}
Fachreferatsaufgaben, aber nicht nur Quereinsteiger:innen fragen sich zu
Beginn ihrer Tätigkeit in diesem Bereich vermutlich manchmal, wie das
eigentlich genau funktioniert und was alles dazugehört. Dieser kleine
Band kann -- auch wenn er auf amerikanische Bibliotheken ausgelegt ist
-- Einsteiger:innen Ideen zur Orientierung geben und \enquote{alten
Hasen} eine Anregung sein, mal darüber nachzudenken, was sie im Rahmen
ihrer Bestandsaktivitäten auch/nicht/nicht mehr machen (und warum).

Die zahlreichen Fallbeispiele inklusive Lessons Learned sowie
\enquote{local practice questions} am Ende jedes Kapitels (und nochmal
zusammengefasst am Ende des Buches) sind dazu hilfreich.
\enquote{Touristisch interessant} sind amerikanische Zuständigkeiten und
Verfahren, die es an deutschen Bibliotheken {[}zumindest an denen, die
ich kenne, V.V.{]} nicht gibt, wie zum Beispiel den \enquote{Collection
Development Librarian} (die/der nicht mit der Leitung der
Erwerbungsabteilung und auch nicht mit den für die einzelnen Fächer
zuständigen Subject Librarians identisch zu sein scheint) oder die
Beteiligung von Bibliotheken an \enquote{New Course Proposals} und
Akkreditierungsverfahren der Institute/Fachbereiche.

Erstaunlicherweise wird das Thema \enquote{Open Access} mit keiner Silbe
erwähnt; nur Open Educational Ressources wird im Zusammenhang mit
Lehrbüchern kurz angerissen. Für eine zweite Auflage würde ich mir zudem
ein Kapitel zum Thema \enquote{Bestandsbewerbung} wünschen. (vv)

\begin{center}\rule{0.5\linewidth}{0.5pt}\end{center}

Melo, Maggie; Nichols, Jennifer T. (edit) (2020). \emph{Re-Making the
Library Makerspace: Critical Theories, Reflections, and Practices}.
Sacramento: Library Juice Press, 2020 {[}gedruckt{]}

In gewisser Weise gilt für dieses Buch tatsächlich einmal, dass die
Entwicklung in den USA weiter ist als im DACH-Raum. Während hierzulande
Bibliotheken immer noch überlegen, ob und wie sie Makerspaces einrichten
werden, wird in diesem Buch schon -- auf der Basis vorhandener
Erfahrungen mit, vor allem, Makerspaces in Wissenschaftlichen
Bibliotheken -- zur Kritik und Verbesserung übergegangen. Die
Autor*innen des Bandes verweisen dabei immer wieder neu darauf, dass
Makerspaces -- entgegen ihrem Anspruch, für alle offen zu sein und neue
Lernformen zu motivieren -- auch in Bibliotheken immer wieder vor allem
von den gleichen Personen (weiss, männlich, gut ausgebildet und gut
verdienend) besucht werden. Aber sie bleiben dabei nicht stehen, sondern
versuchen zu klären, wie dies anders werden kann, wie diese Orte also
von mehr diversen Personengruppen genutzt werden können. Klar wird
dabei, dass es nicht darum geht, Personen einfach anders anzusprechen
oder einfach zu behaupten, dass mit einem Makerspace alle Menschen
angesprochen werden, sondern darum, die Makerspaces selber zu ändern.
Barrieren müssen benannt, wahrgenommen und dann auch abgebaut werden
(und nicht abgestritten). Oft geht es dabei darum, von Technik und
unüberprüften Versprechen (wie dem, dass Makerspaces neue Formen des
Lernens motivieren würden) wegzukommen. Auch der Anspruch, Innovation zu
fördern, muss dazu oft fallengelassen werden. Es geht darum, Räume und
Strukturen zu schaffen, in denen Menschen sich nicht erst beweisen
müssen, um diese zu nutzen. Und es geht auch um Angebote, die Personen
tatsächlich interessieren -- was öfter mit Basteln und viel weniger mit
Hightech zu tun hat.

Während das Buch stark mit Beiträgen startet, die teilweise umfangreiche
Forschung schildern, werden sie je weiter hinten sie im Buch stehen,
beliebiger. Nicht bei allen ist der Bezug zum eigentlichen Thema des
Buches immer klar. Trotzdem ist die Lektüre zumindest ausgewählter
Beiträge für alle zu empfehlen, die Makerspaces in Bibliotheken nicht
als Prestigeprojekt betreiben wollen (oder weil es andere auch tun),
sondern damit diese eine tatsächliche positive Wirkung für eine
möglichst grosse Anzahl von Menschen haben können. (ks)

\begin{center}\rule{0.5\linewidth}{0.5pt}\end{center}

Byrant, Tatiana; Cain, Jonathan O. (edit.) (2020). \emph{Libraries and
Nonprofits: Collaboration for the Public Good}. Sacramento: Library
Juice Press, 2020 {[}gedruckt{]}

Das Buch versammelt Beispiele für die Zusammenarbeit von
US-amerikanischen Öffentlichen Bibliotheken und Non-Profit Einrichtungen
wie Stiftungen oder politischen Gruppen. Auf der einen Seite erhält man
hier -- wieder einmal -- einen Einblick darin, wie anders die
Bibliotheksarbeit in den USA aufgestellt ist. Zwar gibt es in der
Einleitung und dann den einzelnen Beiträgen übertriebene Aussagen zur
sozialen Funktion von Bibliotheken, die mit solchen Kooperation noch
mehr gestärkt werden könnten (was nicht gezeigt, sondern behauptet
wird). Aber gleichzeitig hat man auch den Eindruck, dass es für die
Bibliotheken eigentlich nicht weiter begründet werden müsste: Es scheint
normal, dass es solche Kooperationen, die über die eigene Institution
und den Ort Bibliothek hinausgehen, gibt. Die Frage scheint nur, welche
Kooperation von welcher Bibliothek konkret eingegangen wird. Allerdings
sind die geschilderten Kooperationen oft auch so spezifisch lokal, dass
nicht klar ist, was andere Bibliotheken von ihnen lernen sollen -- oder
gar, ob Bibliotheken aus dem DACH-Raum mit ihnen etwas anfangen können.

Auf der anderen Seite ist das Buch gerade deshalb auch ein Muster dafür,
dass eine solche Beispielsammlung nicht sehr hilfreich ist. Alle
Beiträge bewegen sich sehr an der Oberfläche, die konkrete Arbeit bei
den Kooperationen -- wer verhandelt, was wird geklärt, wie wird es
evaluiert, weiterentwickelt und ähnliches -- wird praktisch nicht
thematisiert. Insoweit können andere Bibliotheken auch wenig mit den
Beispielen anfangen. Gleichzeitig gibt es keine richtige theoretische
Einordnung oder Entwicklung von verallgemeinernden Aussagen. Aus diesem
Grund liefert die Sammlung auch keinen Erkenntnisfortschritt. Das Buch
hinterlässt die Frage, für wen es dann eigentlich da ist. Auffällig ist
das vor allem, weil es aus dem restlichen Verlagsprogramm von Library
Juice Press, wo ansonsten kritische Werke verlegt werden, heraussticht.
(ks)

\hypertarget{bibliotheksgeschichte-geschichte}{%
\subsection{3.4 Bibliotheksgeschichte /
Geschichte}\label{bibliotheksgeschichte-geschichte}}

Jochum, Uwe; Lübbers, Bernhard; Schlechter, Armin; Wagner, Bettina
(Hrsg.) (2020). \emph{Jahrbuch für Buch- und Bibliotheksgeschichte, Band
5}. Heidelberg: Universitätsverlag Winter, 2020 {[}gedruckt{]}

Das Jahrbuch für Buch- und Bibliotheksgeschichte folgt auch in seiner
fünften Ausgabe einer Vorstellung von Geschichtsschreibung, bei der es
vor allem um kleinteilige Darstellungen geht und wiederum von den
Lesenden tiefgehende Kenntnisse bestimmter historischer Zusammenhänge
vorausgesetzt wird. (Wie schon in der Vorstellung der ersten vier Bände
in der Ausgabe \#5 dieser Kolumne dargestellt wurde
\url{https://libreas.eu/ausgabe36/dldl/}). In dieser Ausgabe werden
beispielsweise Wissen zur bayerischen Landesgeschichte des späten 18.
Jahrhunderts, inklusive des Illuminatenordens, Kenntnisse der
nationalsozialistischen Besatzpolitik in Lothringen sowie Kenntnisse der
französischen Sprache vorausgesetzt.

Nichtsdestotrotz sind mehrere Artikel auch mit grossem Gewinn für nicht
unbedingt an den speziellen Themen Interessierte zu lesen: Fabian Waßer
{[}35--77{]} stellt, basierend auf seiner Bachelorarbeit, die
Bibliotheken und Lesegesellschaften in Bayern im besagten späten 18.
Jahrhundert vor. Der Artikel ist sehr kenntnisreich, teilweise etwas
langatmig und zeigt nicht nur, wie diese Einrichtungen für eine
wachsende Zahl von Menschen Zugang zu Literatur boten, sondern auch, wie
sich dann schnell Lesegesellschaften zu sozialen Clubs und
Freizeiteinrichtungen wandelten, bei denen das Lesen nur einen
Teilbereich darstellte. Er thematisiert auch die Ergebnisse der
Verfolgung aller irgendwie als revolutionär (also demokratisch oder
aufklärerisch) verstanden Einrichtungen nach dem Verbot des genannten
Illuminatenordens in Bayern. Wolfgang Freud's {[}111--130{]} Darstellung
der \enquote{Westraumbibliothek} in Metz, die vom dortigen
nationalsozialistischen Machthaber als wissenschaftliche
\enquote{Grenzlandbibliothek} über Frankreich gedacht war, in welcher
nach dem zweiten Weltkrieg das notwendige Wissen zur
\enquote{Auseinandersetzung} mit und Beherrschung von Frankreich durch
Deutschland produziert werden sollte, ergänzt das vorhandene Wissen über
die Wissenschafts- und Bibliothekspolitik im Nationalsozialismus. Am
spannendsten ist der Artikel von Birgit Schaper und Michael Herkenhoff
{[}131--190{]}. Sie schildern, wie die Universitätsbibliothek Bonn durch
das Auktionshaus Sotheby darüber informiert wurde, dass diesem
Inkunabeln und alte Drucke zur Versteigerung angeboten wurden, welche
zum Bestand der Universitätsbibliothek gehören könnten -- und wie diesem
Hinweis folgend ein grosser Bestand solcher Stücke in belgischem
Privatbesitz aufgefunden und einvernehmlich wieder an die Bibliothek
zurückgeführt wurde. Ein viel länger Teil des Artikels ist der Recherche
gewidmet, wie das Fehlen dieser Bestände so lange übersehen werden
konnte und wie sie überhaupt abhanden kamen. Dabei gibt es einige
Stellen, die für die Recherche gewiss notwendig waren, aber den Text
etwas lang machen, insbesondere die Schilderungen der Auslagerung der
Bestände am Ende des zweiten Weltkrieges und deren Rückführung bis in
die 1950er Jahre in den Bestand. Letztlich zeigte die Recherche nämlich,
dass die Medien dabei nicht abhanden kamen, sondern als sie über einige
Jahre in einem Bunker zwischengelagert wurden. Insgesamt ist der Artikel
auch eine Aufforderung an andere Bibliotheken (in Deutschland und
Österreich) mit historischen Beständen, sich die Vorgänge der späten
1940er und 1950er-Jahre in der eigenen Institution einmal genauer
anzuschauen. (ks)

% \begin{center}\rule{0.5\linewidth}{0.5pt}\end{center}

Peiss, Kathy (2020). \emph{Information Hunters: When Librarians,
Soldiers, and Spies Banded Together in World War II Europe.}New York:
Oxford Library Press, 2020 {[}gedruckt{]}

Zufällig ist die Autorin dieses Buches, eine Professorin für
amerikanische Geschichte, mit einem früh verstorbenen Bibliothekar
verwandt, welcher während und kurz nach dem Zweiten Weltkrieg für den
amerikanischen Nachrichtendienst und die Library of Congress in Europa
Bücher und andere Medien der \enquote{Achsenmächte} erwarb. Diese
Verwandtschaft trieb die Autorin dazu, zu erforschen, wie
US-amerikanische Bibliothekar*innen in dieser Zeit erst
Informationsservices für Regierung und Armee aufbauten, gleichzeitig für
US-amerikanische Bibliotheken Forschungsbestände aus Medien, die während
des Krieges erschienen waren, anlegten und anschliessend zum Beispiel
damit beschäftigt waren, die Medien, welche vom nationalsozialistischen
Deutschland geraubt wurden, wieder zu restituieren. Sie stellt dabei
auch dar, wie diese Aktivitäten zu Entwicklungen im Bibliothekswesen und
der Bibliothekswissenschaft führten, beispielsweise dadurch, dass
US-amerikanischen Bibliotheken das erste Mal zu einer landesweiten
Kooperation zusammengebracht wurden, dass intensiv Mikroverfilmung
eingesetzt und neue Techniken zur schnellen Auswertung von Medien für
die Nutzung durch Nachrichtendienst und Armee entwickelt wurden. Sie
zeigt auch, dass diese Arbeiten wegen der besonderen Situation oft im
moralischen Graubereich stattfanden.

Die Profession der Autorin ist ein grosser Gewinn: Das Buch ist
faktenreich, stellt Kontext dar und ist trotzdem in einem gut zu
lesenden, erzählenden Modus geschrieben. Gleichzeitig ist es keine
\enquote{Heldengeschichte}, wie sie manchmal von historisch
interessierten Bibliothekar*innen geschrieben werden. Es ist auch
deshalb zu empfehlen, weil sich in der Literatur des DACH-Raumes oft auf
die Geschichte der Bibliotheken in diesem Raum konzentriert wird und
beispielsweise viel Platz für die Geschichte der Bestandssicherung durch
Verschickung in abgelegene Orte und die Rückholung nach dem Zweiten
Weltkrieg verwendet wird. Dieses Buch stellt die andere Seite dar. (ks)

\begin{center}\rule{0.5\linewidth}{0.5pt}\end{center}

Black, Alistair (2018). \emph{Libraries of Light: British public library
design in the long 1960s}. London; New York: Routledge, 2018
{[}gedruckt{]}

In weiten Teilen ist dieses Buch Architekturgeschichte, inklusive vieler
Bilder aus Beispielbauten. Der Autor stellt hier -- anschliessend an
sein Buch \enquote{\emph{Books, buildings and social engineering : early
public libraries in britain from past to present}} (London : Routledge,
2016), in welchem er die Bibliotheksbauten in Grossbritannien bis 1939
untersuchte -- die Gebäude Öffentlicher Bibliotheken vor, welche
zwischen den späten 1950ern und frühen 1970ern in Grossbritannien
errichtet wurden. Er argumentiert, dass diese vor dem Hintergrund des
Wohlstandsstaates, welcher in diesen Jahren ausgeweitet wurde, inklusive
der Vorstellung, ein neues Grossbritannien zu errichten, in welchem
Planung, Demokratisierung, Wohlstand und Marktwirtschaft zusammenspielen
würden, verstanden werden müssen. Architektonisch seien die Neubauten
allesamt der Moderne zuzuordnen. Das \enquote{Licht} im Titel bezieht
sich dabei sowohl auf die Vorstellung einer gesellschaftlichen Öffnung
als auch des geplanten Einsatzes von Tages- und künstlichem Licht in den
Bibliotheken (und anderen öffentlichen Bauten dieser Zeit).

Interessant sind bei dieser Vorstellung aber auch Kontinuitäten, die er
im Anschluss an sein vorhergehendes Buch herausstellt. In diesem hatte
er gezeigt, dass die Architekten und Bibliotheksplaner der ersten
Jahrzehnte des Öffentlichen Bibliothekswesens in Grossbritannien
praktisch davon besessen waren, funktionale Gebäude zu planen, die eine
möglichst effektive Nutzung als Bibliothek ermöglichten. In den
\enquote{langen 1960ern} allerdings wurden diese Bibliotheken dann vor
allem als veraltet, rückwärtsgewandt, historisierend wahrgenommen.
Deshalb wurden neue, offene, funktionale Gebäude als notwendig
angesehen. Anhand der Zentralbibliothek in Birmingham zeigt Black, wie
sehr diese Einschätzung sich wiederholt. Die 1974 als modern und mutig
beschriebene neue Bibliothek, welche eine als veraltet angesehene
Bibliothek vom Ende des 19. Jahrhunderts ablöste, wurde 2017 abgerissen,
weil sie selber als veraltet, unpraktisch und rückwärtsgewandt
verstanden wurde. Für die heute gebauten und gelobten neuen
Bibliotheksgebäude zeigt das nur, wie sehr sie jeweils von aktuellen
Diskursen geprägt sind und wie sehr sie selber Gefahr laufen, in wenigen
Jahrzehnten als vollkommen überholt zu gelten. (Wobei einige Gebäude aus
den 1960ern, wie Black auch zeigt, weiterhin gut funktionieren.) (ks)

\begin{center}\rule{0.5\linewidth}{0.5pt}\end{center}

Atkin, Lara; Comyn, Sarah; Fermanis, Porscha; Garvey, Nathan (2019).
\emph{Early Public Libraries and Colonial Citizenship in the British
Southern Hemisphere.} (New Directions in Book History) Cham: Palgrave
Macmillan, 2019, \url{https://doi.org/10.1007/978-3-030-20426-6}

In der Einleitung zu diesem Buch stellen die Autor*innen als Ziel ihrer
Geschichte auf, die Entwicklung Öffentlicher Bibliotheken in britischen
Kolonien auf der Südhalbkugel (heutiges Südafrika, Singapur, Australien
und Aotearoa Neuseeland) im 19. Jahrhundert als Geschichte des
Entstehens von Citizenship und lokalen Identitäten zu erzählen, welche
die Grundlage späterer nationaler Identitäten legten. Diesen Anspruch
können sie nicht ganz einhalten. Sie berichten über die Geschichte von
acht Bibliotheken in den genannten Kolonien, gehen auf die Debatten in
und über diese ein, über die -- aus den gedruckten Katalogen und
Berichten der Bibliotheken -- abzulesende Bestandsentwicklung und auch
die Entwicklung der Frage, was für diese Bibliotheken als
\enquote{öffentlich} galt. Was sie dabei zeigen ist folgendes: (1) Die
Entwicklung ist jeweils gleichzeitig lokal und imperial -- Basis sind
immer die lokale Situation und die Interessen der lokalen
Kolonialgesellschaft (zum Beispiel stets, als besonderer Ort gegenüber
dem imperialen Zentrum in London und den anderen Kolonien zu gelten und
gleichzeitig ständig wieder die gleichen Debatten zu führen, wie sie
auch aus der Geschichte britischer Bibliothek bekannt sind), (2) die
Entwicklung der Bibliotheken ist immer im \enquote{aufgeklärten}
britischen Diskurs des 19. Jahrhundert verankert (zum Beispiel geht es
konstant darum, die Arbeitenden anzusprechen und ihnen einen Aufstieg
durch Bildung zu ermöglichen, aber gleichzeitig den Zugang zu
Bibliotheken einzuschränken gegenüber Personen, denen vorgeworfen wird,
gar nicht wirklich lesen zu wollen oder aber -- wie gebildeten Frauen --
das falsche zu lesen), (3) die Bibliotheken agieren immer in der
kolonialen Situation, also für die \enquote{europäische} Gesellschaft
und die Teile der \enquote{nicht-europäischen} Gesellschaft, die als
\enquote{an die Zivilisation herangeführt} betrachtet wurden. Der Rest
der Bevölkerung wurde praktisch nicht wahrgenommen. All das war zu
erwarten und wird hier noch einmal bestätigt. Wirklich neue Erkenntnisse
bringt das Buch dagegen leider nicht. (ks)

\begin{center}\rule{0.5\linewidth}{0.5pt}\end{center}

Towsey, Mark ; Roberts, Kyle B. (edit.) (2018). \emph{Before the Public
Library. Reading, Community, and Identity in the Atlantic World,
1650-1850} (Library of the Written Word ; 61, The Handpress World ; 46).
Leiden ; Boston: Brill, 2018 {[}gedruckt{]}

Die hier versammelten Beiträge sind grösstenteils Forschungsbeiträge zu
einzelnen Bibliotheken aus Grossbritannien und seinen Kolonien, plus
einer Privatbibliothek im kolonialen Brasilien. Gemeinsam haben sie,
dass es \enquote{noch} keine Öffentlichen Bibliotheken sind, wie sie
heute in Grossbritannien oder den USA zu finden sind. Allerdings wird
davon mit den letzten beiden Beiträgen auch wieder abgewichen, indem die
frühe Geschichte der Public Libraries in diesen beiden Ländern
besprochen wird.

Beeindruckend ist die Vielfalt der unterschiedlichen Bibliotheken, die
vorgestellt werden. Es geht zum Beispiel um private Büchersammlungen und
die soziale Funktion des Verleihens von Büchern innerhalb der kolonialen
oder urbanen Elite in verschiedenen (heutigen) Ländern. Ebenso werden
verschiedene Formen von Subscription Libraries besprochen, die teilweise
auch wieder als Ort von Eliten wirkten, beispielsweise vor allem als
Clubhaus und weniger als Büchersammlung, teilweise aber auch breit
Literatur zur Verfügung stellten. In einem unterhaltsamen Beitrag (von
Markman Ellis) geht es um Bibliotheken in Kaffeehäusern und ihre
Benutzung. Sichtbar wird, dass Bibliotheken verschiedene Funktionen
übernahmen und auch sehr unterschiedlich wirkten, dass aber schon seit
der frühen Neuzeit immer wieder neue Bibliotheken eingerichtet wurden.
Die Quellen, welche in den Beiträgen ausgewertet werden, sind
unterschiedlich: Teilweise liegen Kataloge und Ausleihverzeichnisse vor,
teilweise werden Tagebücher genutzt. Der Beitrag zu Kaffeehäusern
basiert auf Büchern, die über Ex-Libris und ähnlichen solchen Sammlungen
zugeordnet werden können. Teilweise führt das zu langen Listen von
Büchern oder kleinteiligen Schilderungen von Entscheidungen in Board
Meetings. Das ist, für sich alleine, dann nicht unbedingt erhellend.
Zusammengenommen aber zeigt sich, dass Medien offenbar schon seit
Jahrhunderten immer im Umlauf waren, dass an die Einrichtung
\enquote{Bibliothek} -- nicht nur als Öffentliche Bibliotheken -- immer
wieder neu Hoffnungen gebunden werden, Wissen zu verbreiten und dass
sich trotzdem immer wieder das Freizeitinteresse der Nutzer*innen
durchsetzt. (ks)

\begin{center}\rule{0.5\linewidth}{0.5pt}\end{center}

Hanbury, Dallas (2020). \emph{The Development of Southern Public
Libraries and the African American Quest for Library Access,
1898--1963}. (New Studies in Southern History) Lanham, Boulder, New
York, London: Lexington Books, 2020 {[}gedruckt{]}

Thema dieser Arbeit ist die Entwicklung des Zugangs von
Afroamerikaner*innen zu Leistungen von Öffentlichen Bibliotheken in den
Südstaaten der USA, vor allem während der Jim Crow Era und der Zeit der
Bürgerrechtsbewegung. In einem Teil des Buches geht es um den
historischen Kontext, im anderen Teil um drei Case Studies (Atlanta,
Nashville und Birmingham). Es wird gezeigt, dass sich im Süden der USA
erst spät überhaupt Öffentliche Bibliotheken entwickelten (oft unter
Mitwirkung von Frauenvereinen, die so eine Möglichkeit fanden, in die
öffentliche Sphäre zu treten), dass diese dann geprägt waren von
viktorianischen Vorstellungen (vor allem, dass Öffentliche Bibliotheken
auch Erziehungseinrichtungen seien, die zu Ordnung und Fleiss anregen
würden) und selbstverständlich von Rassismus und Segregation. Immer war
die Situation für Afroamerikaner*innen schlechter, auch im Bezug auf
Öffentliche Bibliotheken, als für Weisse. Der Autor zeigt, dass die
Entwicklung durchgehend lokal unterschiedlich und von Widersprüchen
geprägt war. So gab es zum Beispiel immer die Frage für Aktivist*innen,
ob sie sich für die Desegregation der Öffentlichen Bibliotheken (also
den gleichen Zugang für alle) einsetzen sollten oder dafür, dass die für
Afroamerikaner*innen unterhalten Bibliotheken (die neben denen für
Weisse existierten) gleich gut ausgestattet werden sollten. Oder, dass
es immer einige weisse Bibliothekarinnen gab, die versuchten, im Rahmen
der Möglichkeiten die Situation für Afroamerikaner*innen zu verbessern,
die dann aber bei der Desegregation um ihre Arbeitsplätze in den
Bibliotheken fürchteten. Zudem zeigt der Autor mit den drei Case Studies
auch, dass es immer Unterschiede bei den konkreten lokalen Situationen
gab, obgleich es sich gleichbleibendum rassistische Strukturen handelte.

Der Autor ist Archivar und hat eine historische Ausbildung genossen. Das
ist dem Buch auch anzumerken. Zum einen interessiert ihn, die von ihm
beschriebenen Entwicklungen in die gesamte Geschichtsschreibung zum
US-amerikanischen Süden einzugliedern. Er betont zum Beispiel, dass sie
teilweise gegen gängige Periodisierungen stehen. Gleichzeitig bleibt er
ganz eng an den Quellen und Archivalien, die er heranzieht. Das
hinterlässt oft eine erstaunlich distanzierte Beschreibung. Obgleich er
gewalttätige und spannende Geschichten beschreibt, verliert er sich
teilweise in der Schilderung von Protokollen von Board Meetings und
ähnlichen Quellen. Die Desegregation der drei von ihm beschriebenen
Öffentlichen Bibliothekssysteme wird fast als reiner Verwaltungsakt
dargestellt. (ks)

\begin{center}\rule{0.5\linewidth}{0.5pt}\end{center}

van Sluis, Jacob (2020). \emph{The Library of Franeker University in
Context, 1585-1843} (Library of the Written Word ; 81, The Handpress
World ; 62). Leiden ; Boston: Brill, 2020 {[}gedruckt{]}

Die Universität in Franeker, einer Kleinstadt in der niederländischen
Provinz Friesland / Fryslân, wurde 1585 gegründet, um für den damals
neuen Staat benötigte Priester, Lehrer und Anwälte auszubilden. Sie
entwickelte sich eher langsam, sowohl inhaltlich als auch personell.
Geschlossen wurde sie 1811 von der napoleonischen Regierung. 1815 wurde
sie noch einmal als Art \enquote{Vor-Universität} gegründet, dies aber
ohne grossen Erfolg. Endgültig geschlossen wurde sie dann 1843. Heraus
sticht, dass im Laufe der Zeit elf Kataloge der Bibliothek gedruckt
wurden und ein Grossteil des letzten Bestandes dieser Bibliothek weiter
vorhanden ist, allerdings nicht in Franeker sondern in Leeuwarden und
Delft. Der Autor, Bibliothekar und Forscher an der nahegelegenen
Universität Groningen, nutzte diesen Umstand, um eine detaillierte
Studie zur Geschichte von Universität und Bibliothek durchzuführen.
(Obwohl nicht ausdrücklich gesagt, geschah dies offenbar im Zusammenhang
mit einer Ausstellung in Franeker selber, bei der unter anderem die
Bibliothek im Zustand zur Zeit des ersten Katalogs 1626 rekonstruiert
wurde {[}vergleiche
\url{https://mainzerbeobachter.com/2018/07/20/land-van-latijn/}{]}.)

Der im Titel genannte Kontext ist die gesamte Hochschulgeschichte der
Niederlande in der im Buch behandelten Zeit. Gleichzeitig gibt es auch
Verweise auf die Entwicklung anderer Bibliotheken. Teilweise wird dieser
Kontext recht breit dargestellt. Die gesamte erste Hälfte des Buches
beschäftigt sich nur damit und erst die zweite Hälfte mit der Bibliothek
selber. Bei der Bibliothek werden nicht nur alle elf Kataloge besprochen
und ausgewertet, sondern auch weitere Dokumente hinzugezogen.
Beispielsweise erfährt man über die langsame Veränderung der Nutzung der
Bibliothek, über einen Diebstahl von Büchern im 17. Jahrhundert -- der
im Endeffekt zu einer Verbesserung der Organisation der Bibliothek
führte -- oder den heute noch vorhandenen Schilderungen zum
Bibliotheksraum.

Alles das ist erstaunlich tief recherchiert. Es wird dadurch aber
teilweise sehr langatmig, auch weil sich in der Bibliothek und der
Universität wenig veränderte. Die Nutzungsbedingungen änderten sich zum
Beispiel während 250 Jahren praktisch nicht. Zur Wissenschaft selber
trug die Universität kaum bei. Innovationen gingen von ihr praktisch
nicht aus. Von den Entwicklungen, welche die ganzen Niederlande prägten,
abgesehen, war die Entwicklung der Bibliothek eher langsam und
gleichmässig. Der erwähnte Diebstahl und die darauf folgende
Neuorganisation der Bibliothek stechen schon als Ereignis heraus. (ks)

\begin{center}\rule{0.5\linewidth}{0.5pt}\end{center}

Klosterberg, Brigitte (Hrsg.) (2021). \emph{Historische
Schulbibliotheken: Eine Annäherung.} (Hallesche Forschungen, 56). Halle:
Verlag der Franckeschen Stiftungen Halle, Harrassowitz Verlag in
Kommission, 2021 {[}gedruckt{]}

Dieser Band vereint die Beiträge zu einem schon 2017 durchgeführten
Workshop zu Historischen Schulbibliotheken (in Deutschland), wobei
historisch hier grösstenteil 17. bis 19. Jahrhundert heisst. In der
Einleitung wird postuliert, dass dieses Feld noch kaum bearbeitet ist
und deshalb ein grosses Forschungspotential bietet. Dies stimmt vor
allem für die Frage, was eigentlich erforscht werden sollte.

Die ersten drei Beiträge (von Axel E. Walter, Stefan Ehrenpreis und
Sebastian Schmideler) machen Vorschläge, die vor allem darauf hinzielen,
die Geschichte der Bibliotheken in Schulen in dieser Zeit in der
allgemeinen Schul- und Bildungsgeschichte zu verorten. Das wird in den
folgenden Beiträgen dann nur zum Teil eingelöst. In diesen werden vor
allem einzelne Schulbibliotheken und deren Bestandsgeschichten
rekonstruiert. Dabei wird sich sehr oft eng an den vorliegenden
Dokumenten und bis heute überlieferten Sammlungen orientiert, was
selbstverständlich ein nachvollziehbares, aber wenig
erkenntnisförderndes Vorgehen ist. Der Beitrag von Juliane Jacobi zu
Bibliotheken in Ritterakademien zeigt dies exemplarisch auf: Sie
schildert drei dieser Ausbildungsstätten für den Adel und das, was über
den Bestand ihrer Bibliotheken bekannt ist. Am Ende ihres Artikels
stellt sie fünf Fragen, die zukünftig bearbeitet werden sollten, zum
Beispiel wie Bibliotheken von Schüler*innen genutzt wurden oder wie die
Anschaffungspolitik von den Aufgaben der Akademien bestimmt wurde --
also die weit interessanteren Fragen, die man sich wünschen würde, dass
sie (schon) von ihr angegangen worden wären.

Viele Sammelwerke zu historischen Themen nennen in ihren Vorworten den
Anspruch, ein Thema erst einmal umreissen zu wollen. Bei diesem Werk
stimmt es: Es ist in weiten Teilen Darstellung der vorhandenen Dokumente
und Sammlungen, die für Forschungen über die Nutzung der Bibliotheken
und der Bildungsvorstellungen, die sie repräsentierten, als Grundlage
dienen können. (ks)

\begin{center}\rule{0.5\linewidth}{0.5pt}\end{center}

Tygör, Lutz; Friebe, Reiner (2019). \emph{Potsdamer städtische
Volksbücherei: Vorgeschichte und Gründung}. Leipzig: Engelsdorfer
Verlag, 2019 {[}gedruckt{]}

In dieser Broschüre tragen die Autoren Quellen zur Gründung der im Titel
genannten Volksbücherei in Potsdam zusammen. Dabei können sie darauf
verweisen, dass die bisherige Literatur dazu unterschiedliche Daten
geliefert hat. Mittels Dokumenten aus dem Stadtarchiv Potsdam und
Verlautbarungen in der damaligen städtischen Presse können sie zeigen,
dass die tatsächliche Gründung 1899 stattfand und dass es gleichzeitig
eine Vorgeschichte dieser Gründung gab. Zudem liefern sie Angaben zu an
dieser Gründung beteiligten Personen. Der Text liest sich, wie viele
Lokalgeschichten sonst auch: Engagiert in der konkreten lokalen Frage,
aber kaum darum bemüht, einen über die lokalen Kontext hinausgehenden
Zusammenhang herzustellen. Letztlich zeigt sich nämlich, dass die
Entwicklung in Potsdam -- zuerst die Gründung von Vereinsbibliotheken
und (kommerziellen) Leihbibliotheken, dann von Vereinen getragene
Bibliotheken für die Öffentlichkeit, zuletzt die Gründung einer
Bibliothek und Lesehalle durch die Gemeinde -- sich so vollzog, wie auch
in vielen anderen deutschen Gemeinden auch. Aber dies muss den
Leser*innen selber geschlossen werden, die Autoren tun dies nicht. (ks)

\begin{center}\rule{0.5\linewidth}{0.5pt}\end{center}

Mandel, Birgit; Wolf, Birgit (2020). \emph{Staatsauftrag »Kultur für
alle«. Ziele, Programme und Wirkungen kultureller Teilhabe und
Kulturvermittlung in der DDR}. (Schriften zum Kultur- und
Museumsmanagement) Bielefeld: transcript Verlag, 2020 {[}gedruckt{]}

Dieses Buch will untersuchen, wie in der DDR die Vermittlung von Kultur
organisiert war und mit welcher Wirkung dies passierte. Grundthese ist,
dass dies grundsätzlich anders organisiert war, als heute, dass darüber
aber wenig bekannt sei und dass es möglich wäre, aus dieser Geschichte
etwas für die heutige Kulturvermittlung zu lernen. Es ist aber keine
Untersuchung, sondern eher eine Materialsammlung, welcher der
theoretische Rahmen oder Abstand zum Thema fehlt: In einem Kapitel
werden vor allem Ausschnitte offizieller Dokumente zitiert, dann in
einem weiteren Interviews mit in der Kulturvermittlung in der DDR
Tätigen inhaltlich zusammengefasst, gefolgt von Interviews mit Personen,
die in der DDR gelebt haben und zuletzt werden die Ergebnisse einer
Archivrecherche zu drei Kultureinrichtungen präsentiert. Teilweise sind
das Ergebnisse von Studierendenarbeiten, durchgängig wird mit langen
Zitaten gearbeitet. Das Fazit ist gegenüber dem (im Vorwort
formulierten) eigenen Anspruch dann sehr kurz.

Das Buch zeigt, dass Kulturvermittlung in der DDR grundsätzlich von der
Idee ausging, dass eine \enquote{entwickelte sozialistische
Persönlichkeit} sich immer auch kulturell betätigen würde und dass
deshalb daraufhin gearbeitet wurde, dies auch zu ermöglichen: Mit früher
Kulturbildung, mit massiven Investitionen in die Öffnung von
\enquote{Hochkultur} für alle und in die Entwicklung einer
\enquote{Breitenkultur}, beispielsweise durch ein enges Netz von
Kulturhäusern, von kulturellen Zirkeln in den Betrieben oder
Verpflichtungen von kulturellen Einrichtungen zur Kooperation mit
Betrieben, Schulen und so weiter. Grundsätzlich sei der Kulturbegriff in
der DDR sehr breit gewesen und hätte zum Beispiel auch den Sport oder
Tourismus umfasst. Zudem sei die Kulturvermittlung von einer Mischung
aus Zwang und Freiwilligkeit geprägt gewesen: Immer sei es vordergründig
um das Ziel einer \enquote{sozialistisch entwickelten Persönlichkeit}
gegangen, aber in der Realität hätte es auch Freiräume gegeben. Zudem
sei Kultur und Geselligkeit eng miteinander verbunden gewesen. Die
Erfolge dieser Politik seien unterschiedlich zu bewerten: Die
anvisierten Schichten (\enquote{Arbeiter und Bauern}) seien nur bedingt
für die \enquote{Hochkultur} gewonnen worden, dafür aber seien viele
Menschen breitenkulturell tätig gewesen und Kultur in allen Formen eher
Teil des normalen Alltags gewesen. Dennoch hätte die Politisierung und
der Zwang auch zum Entstehen von subversiven Gegenkulturen geführt.

Erwähnt wird das Buch hier aber, weil (selbstverständlich) auch
Bibliotheken in diesem immer wieder vorkommen. Ihnen wird nicht die
Aufmerksamkeit wie den Kulturhäusern (die als Eigenheit der DDR
beschrieben werden) geschenkt, aber in den Interviews werden sie immer
wieder einmal angesprochen. Mit Roswitha Kuhnert ist unter den
interviewten Kulturvermittler*innen eine Vertreterin der
(Kinder-)Bibliotheken. Sie schätzt die Arbeit der Bibliotheken in der
DDR als breitenwirksam ein und betont, dass die politischen Vorgaben
offiziell erfüllt, aber in der Praxis auch immer umgangen wurden. Die
gesetzlichen Grundlagen der (eigenständigen) Kinderbibliotheken im
Jugendfördergesetz werden ebenso kurz vorgestellt. Öffentliche
Bibliotheken, so der Eindruck, den die versammelten Materialien
vermitteln, waren fester Teil der Kulturvermittlung in der DDR. (ks)

\begin{center}\rule{0.5\linewidth}{0.5pt}\end{center}

Stühlinger, Harald R. (Hrsg.) (2020). \emph{Rotes Wien Publiziert.
Architektur in Medien und Kampagnen}. Wien, Berlin: Mandelbaum Verlag,
2020 {[}gedruckt{]}

Die heutige Wienbibliothek im Rathaus -- ehemals Stadtbibliothek, zuvor
Städtische Sammlungen -- hat als Landesbibliothek den Auftrag, eine
möglichst vollständige Sammlung aller Publikationen von Stadt und Land
Wien zu betreiben und setzt dies seit über 100 Jahren konsequent um. Auf
der Basis dieses Fundus werden regelmässig Ausstellungen im Wiener
Rathaus durchgeführt. Diese Publikation zur Bautätigkeit im
\enquote{Roten Wien} -- also der Zeit der ersten sozialdemokratischen
Alleinregierung in der Stadt von 1918 bis 1934 -- mit ihren sozialen
Einrichtungen und massiven Gemeindebauten sowie deren Repräsentation in
Bildern, Filmen, Plakaten, Büchern und Broschüren, stellt den
erweiterten Katalog einer dieser Ausstellung von Ende 2019 bis Anfang
2020 dar. Erstaunlicherweise werden zwar die unterschiedlichen Medien
der sich (damals) als eine der fortschrittlichsten verstehenden
Stadtverwaltung der kapitalistischen Welt vorgestellt (und teilweise mit
zeitgleichen Publikationen aus Amsterdam verglichen), aber Öffentliche
Bibliotheken -- die selbstverständlich auch in grosser Zahl gebaut
wurden, obgleich das Netz der Arbeiterbibliotheken in Wien in diesen
Jahren auch ausgebaut wurde -- kommen nicht explizit vor. Was sich
findet, sind aber Verweise darauf, dass die Ausstellung getragen wird
von der langjährigen Sammlungstätigkeit der Wienbibliothek. Zudem gibt
es den kurzen Verweis, dass diese Bibliothek in den 1920er und 1930er
Jahren die Aufgabe hatte, biographische Auskünfte für die Benennung der
Gemeindebauten zu liefern. (ks)

\hypertarget{buch--und-bibliothekskultur-zur-zeit-des-europuxe4ischen-mittelalters}{%
\subsection{3.5 Buch- und Bibliothekskultur zur Zeit des (europäischen)
Mittelalters}\label{buch--und-bibliothekskultur-zur-zeit-des-europuxe4ischen-mittelalters}}

Livesey, Steven J. (2020). \emph{Science in the Monastery: Textes,
Manuscripts and Learning at Saint-Bertin} (Bibliologia ; 55). Turnhout:
Brepols, 2020 {[}gedruckt{]}

Wilkins, Nigel (2018). \emph{Le Bibliothèque de l'Abbaye disparue de
Notre-Dame de Lyre.} (Selbstverlag) {[}Bruges{]}: {[}Aquiprint{]}, 2018
{[}gedruckt{]}

Zäh, Helmut ; Pfändtner, Karl-Georg ; Raueiser, Stefan ; Weber, Petra
(Hrgs.) (2018). \emph{Abtransportiert, verschwunden und wieder sichtbar
gemacht: Die Bibliothek Kloster Irsee in der Staats- und Stadtbibliothek
Augsburg.} (Cimeliensaal ; 3). Luzern: Quaternio Verlag, 2018
{[}gedruckt{]}

Alle drei Werke tragen retrospektiv die Bibliothek je eines
dominikanischen Klosters zusammen, welche während der französischen
Revolution (Notre-Dame de Lyre, Saint-Bertin) beziehungsweise der
Säkularisierung in Bayern (Irsee) vom jeweiligen Staat übernommen
wurden. Die Bibliotheken wurden aufgelöst und zum Teil in staatliche
Bibliotheken überführt, zum Teil in private Hände oder den
Antiquariatshandel übergeben oder auch vernichtet. In diesen drei
Büchern wird die jeweilige Bibliothek, soweit das aufgrund von Quellen
oder Funden in Bibliotheken möglich ist, rekonstruiert. Für Notre-Dame
de Lyre und Saint-Bertin liegen dabei hilfreich je eine Liste der Bücher
vor, welche bei der Übernahme der Klosterbibliothek in die designierten
staatlichen Bibliotheken erstellt wurde.

Die beide Bücher über Notre-Dame de Lyre und Irsee beginnen je mit einem
Aufsatz über die Geschichte der jeweiligen Bibliotheken, wobei bei
beiden aufgrund der mangelhaften Quellenlage vieles im Dunkeln
verbleibt. Bei Wilkins, dessen Buch wie ein Manuskript für den
Privatgebrauch wirkt, folgt dann eine lange Liste mit allen bekannten
Werken der Bibliothek aus Lyre, inklusive Nachweisen der vorhandenen
Exemplare in heutigen Bibliotheken und Sammlungen. Abgeschlossen wird
das Buch dann von Bildern dieser Werke. Das Buch von Zäh et al., welches
die Begleitpublikation zu zwei Ausstellungen zur Bibliothek in Augsburg
war, ist aufwendiger gestaltet. Hier werden 48 Publikationen in Bild und
Text (je ein bis zwei Textseiten) vorgestellt und kontextualisiert sowie
auf die Exemplare und, wenn vorhanden, Digitalisate hingewiesen. Livesey
liefert auch einen kommentierten Katalog ausgewählter Bücher aus der
Bibliothek in Saint-Bertin, welche heute noch zugänglich sind. Der
einleitende Text interpretiert diese Bücher, inklusive der enthaltenen
handschriftlichen Kommentare, als Arbeitsmittel für Bildung und
Wissenschaft, wie sie im Kloster betrieben wurde. Er interpretiert
ausgewählte Bücher, indem er sehr tief in die Anmerkungen und
Nutzungsspuren eingeht und verortet sie im Wissen darüber, wie Bücher in
Klostern des Spätmittelalters und speziell in Saint-Bertin genutzt
wurden.

Alle drei Werke gehören in den offensichtlich vorhandenen Trend von
Publikationen, welche versuchen, mittelalterliche und frühneuzeitliche
Klosterbibliotheken wieder sichtbar zu machen. (ks)

\begin{center}\rule{0.5\linewidth}{0.5pt}\end{center}

Behrens-Abouseif, Doris (2019). \emph{The Book in Mamluk Egypt and Syria
(1250-1517): Scribes, Libraries and MarketI.} (Islamic History and
Civilization: Studies and Texts, 162) Leiden; Boston: Brill, 2019
{[}gedruckt{]}

Hirschler, Konrad (2020). \emph{A Monument to Medieval Syrian Book
Culture: The Library of Ibn 'Abd al-Hādī}. (Edinburgh Studies in
Classical Islamic History and Culture). Edinburgh: Edinburgh University
Press, 2020 {[}gedruckt{]}

Diese beiden Werke bieten Einblick in die intellektuelle Kultur,
inklusive Buch- und Bibliothekskultur, im heutigen Ägypten und Syrien
zur Zeit der Herrschaft der Mamluken. (Wie im Titel des Werkes von
Behrens-Abouseif angegeben als 1250--1517 europäischer Zeitrechnung.)
Sie zeigen dabei eine Kultur, die intensiv mit Texten, dem Schreiben,
Lesen, Sammeln und Verbreiten von Wissen beschäftigt war, entsprechende
Strukturen ausbildete und gleichzeitig an diese Epoche und ihre
Interessen -- in einer Zeit der Konsolidierung muslimischer Herrschaft
und eines religiösen Kanons -- gebunden war.

Das Buch von Behrens-Abouseif liefert eine Zusammenschau des vorhandenen
Wissens zum Themengebiet. Es bietet eine gute Übersicht, um in das Thema
einzusteigen. Sie beschreibt unter anderem Bibliotheken und
Büchersammlungen, aber auch das heute vorhandene Wissen um damalige
Möglichkeiten des Verleihs und Kopierens von Büchern. Sichtbar wird, wie
sehr dieses Sammeln und Verleihen Teil des intellektuellen und
religiösen Lebens war. Gleichzeitig zeigt das Buch auch, dass im Bezug
auf Medien andere Schwerpunkte als heute gesetzt wurden, beispielsweise
in den beiden langen Kapiteln über Kalligraphie und
Kalligraphie-Schulen.

Hirschler untersucht eine massive Sammlung von Büchern aus der Endzeit
der Herrschaft der Mamluken, die zu einem grossen Teil noch bis zum
aktuellen Bürgerkrieg in Syrien in Damaskus zugänglich war. Diese wurden
von einem Gelehrten angelegt, der vor allem lokale Bedeutung hatte und
mit ihr auch die Kultur einer verblassenden religiösen Tradition
dokumentierte. Die Sammlung überlebte bis in heutige Zeit, weil sie
keine handwerklichen Meisterwerke beinhaltete, sondern als Ansammlung
gebundener, selbstgeschriebener Broschüren daherkommt. Dennoch gibt es
hinter ihr einen Plan. Hirschler rekonstruiert anhand von Nutzungsspuren
in den Dokumenten und einer historischen Kontextualisierung die
dahinterstehende intellektuelle Nutzung, beispielsweise die Tradition
des wörtlichen Vortrags von Textes, die anschliessend in den Dokumenten
selber vermerkt wurde. Dem eigentlichen inhaltlichen Teil von 170 Seiten
schliessen sich einige hundert Seiten an Dokumentation des Katalogs
dieser Sammlung und Digitalisaten an.

Zusammen zeigen beide Bücher auch, wie ähnlich und doch anders diese
Buch- und Bibliothekskultur zur damaligen (oder heutigen) in Europa ist.
Wie immer ist daraus auch zu lernen, wie spezifisch \enquote{unsere}
Buch- und Bibliothekskultur zu unserer Zeit und unserem Ort in der Welt
ist. (ks)

\hypertarget{social-media}{%
\section{4. Social Media}\label{social-media}}

Dobusch, Leonhard (2020): \emph{Kein Open-Access-Deal, dafür mit Spyware
gegen Schattenbibliotheken?} In: netzpolitik.org.
\url{https://netzpolitik.org/2020/neues-vom-grossverlag-elsevier-kein-open-access-deal-dafuer-mit-spyware-gegen-schattenbibliotheken/},
zuletzt geprüft am 17.02.2021

Im Sommer 2018 brach Elsevier die Verhandlungen über mögliche neue
Open-Access-Publika\-tionsmodelle im Rahmen des DEAL-Projektes ab und
weigert sich seitdem, diese wieder aufzunehmen. Mitglieder von
wissenschaftlichen Einrichtungen umgehen das Problem, aktuell keinen
Zugang zu den Publikationen des Verlags zu haben, indem sie auf
Schattenbibliotheken, wie zum Beispiel Sci-Hub, zurückgreifen. Für
wissenschaftliche Einrichtungen bieten Schattenbibliotheken eine gewisse
Kompensation fehlender oder zu teurer Zugänge, wenn es um den Zugang zu
Publikationen von Großverlagen geht. Anstatt sich mit Hochschulen und
Forschungseinrichtungen auf neue Open-Access-Modelle zu konzentrieren
und mögliche Verträge auszuhandeln, will Elsevier gegen
\enquote{Cybercrime} vorgehen und dazu zählt der Verlag auch
Schattenbibliotheken. Für diesen Zweck wurde mit anderen Verlagen die
\enquote{Scholarly Networks Security Initiative (SNSI)} gegründet. (kfk)

\begin{center}\rule{0.5\linewidth}{0.5pt}\end{center}

Yannick Paulsen (2020): \emph{Der aktuelle Stand deutscher
Hochschulbibliographien an Universitäten}. In: Gemeinsamer Blog der DINI
AGs FIS \& EPUB, 16.11.2020,
\url{https://blogs.tib.eu/wp/dini-ag-blog/2020/11/16/stand-hochschulbibliographien-universitaeten/}

Yannick Paulsen stellt in einem Blogbeitrag die Ergebnisse eines
studentischen Projekts an der Hochschule für Technik, Wirtschaft und
Kultur Leipzig vor, welches sich mit Hochschulbibliographien an
deutschen Universitäten beschäftigte. Die Untersuchung, welche sich
methodisch auf eine Kombination von Befragung und der Auswertung von
Webseiten beziehungsweise Systemen stützte, beleuchtet Fragen der Art
des Zugangs zur Bibliographie, der organisatorischen Zuständigkeiten und
Abläufe sowie der Einbettung in den universitäten Kontext. (mv)

\begin{center}\rule{0.5\linewidth}{0.5pt}\end{center}

Twitter: Uni of York Library (@UoYLibrary)
\href{https://twitter.com/UoYLibrary/status/1389508509339996166?s=20}{https://twitter.com/UoYLibrary/status/13895085 09339996166?s=20}

Eine Frage, die sich (nicht nur) in Pandemie-Zeiten stellt: Warum gibt
es das Buch123 nicht als E-Book über die Bibliothek? Die Bibliothek der
Universität York gibt einen Überblick über die Gründe, die in der
Fachcommunity wohl hinlänglich bekannt, aber dadurch nicht weniger
verstörend sind: Neben fehlenden Lizenzangeboten für Bibliotheken (im
Gegensatz zu Angeboten für Privatpersonen -- eine Geschäftsentscheidung
der Verlage) ist es vor allem die eklatante Preispolitik (häufig bei
gleichzeitiger Beschränkung der zulässigen Anzahl paralleler
Nutzer*innen), welche einen Medienerwerb verunmöglichen, der dem 21.
Jahrhundert angemessen wäre. (mv)

\begin{center}\rule{0.5\linewidth}{0.5pt}\end{center}

Wissen, Dirk. \emph{Technische Hochschule Wildau: RoboticLab
(Studiengang Telematik): Auf einen Espresso mit Wilma: Interview von
Dirk Wissen mit dem humanoiden Roboter Wilma.} aufgerufen am 24.11.2020.
\url{https://icampus.th-wildau.de/icampus/home/de/roboticlab}.

Dirk Wissen, Herausgeber der Fachzeitschrift BuB: Forum Bibliothek und
Information, führt in der Hochschulbibliothek der TH Wildau ein circa
neunminütiges Interview mit dem humanoiden Roboter Wilma. Wilma gibt
dabei ausführlich Auskunft zu wesentlichen Fragen über Aufgaben, Wesen
und Zukunft von humanoiden Robotern in Bibliotheken. Eingesetzt als
Assistenzsystem soll sie die Mitarbeitenden der Bibliothek unterstützen
und entlasten. Derzeit führt sie schon ganz autonom Bibliotheksführungen
durch oder erzählt gestressten Nutzenden zur Entspannung kurze Witze.
Perspektivisch soll Wilma weitere Aufgaben übernehmen: Als
Informationssystem für die Bibliothek könnte sie in den Abend- und
Morgenstunden das Fehlen von Fachpersonal kompensieren, so dass
Studierende und Mitarbeitende der Hochschule die Bibliothek an sieben
Tagen 24 Stunden nutzen können. (dg)

\hypertarget{konferenzen-konferenzberichte}{%
\section{5. Konferenzen,
Konferenzberichte}\label{konferenzen-konferenzberichte}}

{[}diesmal keine Beiträge{]}

\hypertarget{populuxe4re-medien-zeitungen-radio-tv-etc.}{%
\section{6. Populäre Medien (Zeitungen, Radio, TV
etc.)}\label{populuxe4re-medien-zeitungen-radio-tv-etc.}}

Copper, Quintin (2021): \emph{Eine Wiederaneignung} -- \emph{Die
Soziologin Nina Degele bekennt sich offensiv zu Politischer
Korrektheit.} In: Neues Deutschland, 23.01.2021,
\url{https://www.neues-deutschland.de/artikel/1147320.political-correctness-eine-wiederaneignung.html}

Der Artikel gibt dem Lesenden Aufschluss darüber, wann und in welchem
Kontext der Begriff \enquote{Political Correctness} entstanden ist und
darüber hinaus, welches Gewicht er in der jetzigen Debatte hat.
Inhaltlich entstand der Begriff in den USA zusammen mit den
Bürgerrechtsbewegungen in den 1960er Jahren. Prägend waren in den
meisten Fällen Schwarze Menschen und später Frauen. In den populären
Wortschatz ging der Begriff durch eine Rede von Georg Bush über, der
damit linke Bewegungen abwerten wollte. Heute gilt \enquote{Political
Correctness} als Begriff einer Bewegung, wird zur Selbst- und
Fremdpositionierung verwendet und steht im Kern zahlreicher öffentlicher
Diskussionen. Generell hat der Begriff \enquote{Political Correctness}
das Potential für eine Grundlage, um über Veränderung debattieren zu
können -- Stichwort \enquote{Wiederaneignung}. Mit einer Debatte
vollzieht sich im besten Fall auch diese Veränderung. Nina Degele sagt
dazu: \enquote{Das Reden {[}über dieses Thema{]} bewirkt, dass überhaupt
Denkräume geöffnet werden, dass Dinge selbstverständlicher werden, die
dann auch einfacher umgesetzt werden können.}

Die Verwendung \enquote{Political Correctness} als Streitbegriff findet
sie richtig, jedoch nur im Sinne dieser Wiederaneignung, um das
emanzipatorische Potenzial und die Reflexion der eigenen Privilegien zu
fördern. (gs)

\begin{center}\rule{0.5\linewidth}{0.5pt}\end{center}

Clark, Anthony (2021). \emph{Will There Be a Trump Presidential Library?
Don't Count On It.} In: Politico, 22.01.2021,
\url{https://www.politico.com/news/magazine/2021/01/22/trump-presidential-library-plans-461383}.

Schon 2016 stellten erschreckt viele Bibliothekar*innen fest, dass es
dann wohl irgendwann auch eine \enquote{Trump Presidential Library}
geben wird, welche die Bedeutung der Präsidentschaft Donald Trumps in
weite Zukunft positiv darstellen würde. Einfach, weil es Tradition ist,
dass US-amerikanische Präsidenten solche Einrichtungen etablieren.

Anthony Clark, der als Experte für \enquote{Presidential Libraries}
vorgestellt wird (er hat ein Buch zu ihnen geschrieben) sagt etwas
anderes voraus: Trump wird es nicht schaffen, eine solche Bibliothek
einzurichten und schon gar nicht, sie in die Verantwortung des National
Archives zu übergeben. Es ist zu teuer, mit zu viel Aufwand verbunden
und bedarf Durchhaltevermögen. Im Artikel erfährt man auch, dass schon
Präsident Obama von der Tradition solcher Bibliotheken abgewichen ist
und es wohl zukünftig für Präsident Biden und seiner Nachfolger*innen
noch schwieriger sein wird, diese fortzusetzen. (ks)

%\begin{center}\rule{0.5\linewidth}{0.5pt}\end{center}
\pagebreak

o.\,A. (2021). \emph{Gutes Konzept in Krisenzeiten: Stadtbibliothek
bekommt 750 Euro}. In: Wolfsburger Allgemeine, 09.02.2021, S. 15.\\
Die Stadtbibliothek Wolfsburg erhielt als eine von zehn Bibliotheken in
Niedersachsen zusätzliche Fördermittel in Höhe von 750 Euro. Diese
werden vom Deutschen Bibliotheksverband zur \enquote{Förderung von
Bibliotheken in Krisenzeiten} bereitgestellt und sollen das Engagement
der Bibliothek bei der \enquote{Bewältigung der Corona-Pandemie}
würdigen. Die Aktivitäten zu dieser Bewältigung zählten ein frühzeitig
eingerichteter Abholservice für Medien und ein Ausbau des Bestands an
Online-Medien. Dass die Nachfrage hoch ist, zeigen die Steigerungszahlen
bei den Zugriffen auf digitale Angebote: das Angebot von Pressreader
verzeichnete eine Steigerung von 50\,\%, die NAXOS-Library 30 \% und
Munzinger immerhin 5\,\%. Den höchsten Zuwachs gab es bei den
BROCKHAUS-Daten mit 130\,\%. Sehr offensichtlich schlägt also der Bedarf
nach entsprechenden Informationen durch das Home-Schooling durch. Die
Fördermittel möchte die Bibliothek in Angebote zur Unterstützung von
Medienkompetenz von Kindern und Jugendlichen investieren. Zudem wird die
App \enquote{Tigerbooks} für diese Zielgruppen lizensiert. (bk)

\begin{center}\rule{0.5\linewidth}{0.5pt}\end{center}

Engelmark, Siv (2021). \emph{Vi vill få ner kostnaderna för
publiceringar}. In: Curie / \url{https://www.tidningencurie.se/},
3.5.2021,
\url{https://www.tidningencurie.se/nyheter/2021/05/03/vi-vill-fa-ner-kostnaderna-for-publiceringar/}.

Der schwedische Hochschulverbund SUHF hat eine Arbeitsgruppe eingesetzt,
die eine Strategie zur Frage erarbeiten soll, wie das
Open-Access-Publizieren nach 2024 sichergestellt werden kann -- und dass
ohne die Transformationsverträge fortzuführen, die schwedische
Forschungseinrichtungen im Rahmen des nationalen Bibsam-Konsortiums
abgeschlossen haben. Hintergrund ist die Befürchtung, dass die Verlage
an den jetzt befindlichen Modellen festhalten wollen und ergo
Forschungseinrichtungen dauerhaft für den Zugang und das OA-Publizieren
zahlen, während auch weiterhin ein Teil der Artikel (in den hybriden
Zeitschriften) closed access bleibt. Ziel ist also de facto die
Transformation des Publikationsmarktes hin zu Open Access -- konkret
will man \enquote{100\,\% Open Access, zu niedrigeren Kosten und mit
einem transparenten Preismodell}. Wichtig ist wohl vor allem der Hinweis
\enquote{zu niedrigeren Kosten}. Die vorhandenen Transformationsverträge
werden als Werkzeug für den Übergang eingeschätzt (sie haben die
Forschungseinrichtungen zunächst in die Lage versetzt, einen Überblick
über die Gesamtkosten zu erhalten -- zuvor wurden Open-Access-Gebühren
weitestgehend dezentral finanziert); nun aber wird es Zeit für den
nächsten Schritt (denn Teil der Transformationsverträge ist eine
vertraglich vereinbarte Preissteigerung von bis zu 10\,\% pro Jahr --
was über längere Zeit nicht finanzierbar ist).\\
2024 ist vor allem mit Blick auf Plan S ein wichtiges Datum; nur bis
dahin gelten Übergangsfristen, so dass Artikel in hybriden Zeitschriften
noch compliant sind, sofern sie im Rahmen eines Transformationsvertrages
Open Access publiziert werden. Die schwedischen Forschungseinrichtungen
haben also die Ziele der coalition S fest im Blick. Die anvisierte
Strategie soll dann als Fahrplan bei der Verhandlung der nächsten
Verträge mit Verlagen dienen; es wurde aber auch angekündigt, mit
einigen Verlagen schon dafür erste vorbereitende Gespräche für
Folgeverträge führen zu wollen.

Die Arbeitsgruppe soll spätestens im Herbst 2023 einen Vorschlag für die
neue Strategie präsentieren; dieser dürfte dann auch für andere Länder
von Interesse sein. (mv)

%\begin{center}\rule{0.5\linewidth}{0.5pt}\end{center}
\pagebreak

Anna-Seghers-Bibliothek (2020). \emph{Makerspace \enquote{Robo und
Faden} -- Textiles Gestalten in der Anna-Seghers-Bibliothek}.
\url{https://www.youtube.com/watch?v=_C5-SUCohdM}.

In dem etwa sechsminütigen YouTube-Video wird der Makerspace
\enquote{Robo und Faden} vorgestellt. Er befindet sich in der
Anna-Seghers-Bibliothek in Berlin Neu-Hohenschönhausen. Im Makerspace
kann man mit diversen Stick- und Nähmaschinen, einem 3D-Drucker und
anderen Geräten und Werkzeugen rund um das Thema \enquote{textiles
Gestalten} arbeiten. Gedacht ist der Makerspace für alle, denen die
Mittel fehlen, um eigene Ideen und Projekte umzusetzen. Aber auch Leute
ohne konkrete Projektideen sind eingeladen im Makerspace kreativ zu
werden. Es gibt sowohl für Anfänger als auch Fortgeschrittene die
Möglichkeit sich auszuprobieren.

Um die Maschinen für eigenständige Projekte nutzen zu können, benötigt
man einen gültigen Bibliotheksausweis des VOEBB, muss an einer
kostenlosen Einweisung teilnehmen und die Nutzungsbedingungen
unterschreiben. Danach steht der Makerspace frei zur Verfügung. (pf)

\begin{center}\rule{0.5\linewidth}{0.5pt}\end{center}

dpa (2020). \emph{Lesen unterm Sonnenschirm. Bibliotheken ziehen erste
Bilanz.} In: Berliner Morgenpost, 14.12.2020.
\url{https://www.morgenpost.de/berlin/article231135594/Bibliotheken-im-Krisenjahr-Plus-bei-Online-Angeboten.html}.

Für den Artikel \enquote{Lesen unterm Sonnenschirm: Bibliotheken ziehen
erste Bilanz} wurde die Sprecherin der Zentral- und Landesbibliothek
Berlin (ZLB), Anna Jacobi, dazu befragt, wie sich die Corona-Pandemie
auf die Bibliothek auswirkte. Jacobi blickt positiv auf das Jahr 2020
zurück. Die Herausforderungen, die das letzte Jahr gebracht hat, konnten
mit kreativen Ansätzen bewältigt werden. Zwar konnten Besucher*innen
nicht in den Bibliotheken arbeiten. Stattdessen wurden vor der
Bibliothek Liegestühle und andere Sitzgelegenheiten bereitgestellt, die
die Besucher*innen dankend benutzten. Die Dienste, die nicht vor Ort
umgesetzt werden konnten, wurden so gut es ging in digitaler Form
angeboten. Alles in allem, so ihr Fazit, haben sich alle Betroffenen mit
der aktuellen Lage arrangiert und das Beste aus ihr gemacht. (pf)

\begin{center}\rule{0.5\linewidth}{0.5pt}\end{center}

Savage, Sam (2009): \emph{Firmin. Adventures of a Metropolitan Lowlife.}
London: Phoenix, 2009. {[}gedruckt{]}

In einer Buchhandlung zu leben und dort nach Herzenslust lesen zu
dürfen, ist für viele Bücherliebende sicherlich eine traumhafte
Vorstellung. Für Firmin ist dieser Zustand die Wirklichkeit, denn er ist
eine buchstäbliche Leseratte -- also tatsächlich eine Ratte, die liest
-- und lebt in einem alten Buchladen in Boston. Ausgestattet mit einem
enormen Lesehunger verschlingt Firmin wortwörtlich viele Werke der
Weltliteratur und beschäftigt sich intensiv mit der fiktionalen Welt.
Trotzdem fühlt er sich einsam, da er seine Leidenschaft mit niemandem
teilen kann. Seine Familie verlässt ihn wegen seiner für eine Ratte sehr
ungewöhnlichen Literaturbegeisterung und so hofft Firmin, sich eines
Tages wenigstens mit dem Buchhändler Norman Shine verständigen und
anfreunden zu können. Doch dann droht die alteingesessene Buchhandlung
ein Opfer der Gentrifizierung zu werden. \enquote{Firmin} ist eine
charmante Verneigung vor der Literatur und zeigt, wie sie das Leben
bereichern kann. (ne)

%\begin{center}\rule{0.5\linewidth}{0.5pt}\end{center}
\pagebreak

Sielmann, Lara (2021) \emph{\enquote{Baynatna} ist die erste arabische
Bibliothek in Berlin}. In: Tagesspiegel am 08.01.2021. URL:
\url{https://www.tagesspiegel.de/berlin/zwischenklassik-undmoderne-baynatna-ist-die-erste-arabische-bibliothek-in-berlin/26773926.html}.

Seit 2018 ist im Ribbeck-Haus in Berlin-Mitte unmittelbar neben der
Zentral- und Landesbibliothek eine arabische Bibliothek angesiedelt. Sie
heißt \enquote{Baynatna}, was aus dem Arabischen übersetzt
\enquote{Unter uns} bedeutet. Die Bibliothek entstand im Rahmen eines
Programms namens \enquote{Zusammenkunft} des Zentrums für Kunst und
Urbanistik (ZK/U) und dem Wunsch nach einem Ort für arabisch stämmige
Berliner*innen, an dem frei diskutiert und sich speziell hinsichtlich
arabischer Literatur und Literatur auf Arabisch informiert werden kann.
Der Bestand generierte sich hauptsächlich durch Bücherspenden und
umfasst mittlerweile über 3500 Exemplare. Gleichzeitig versteht sich die
Bibliothek als Veranstaltungsort. Er bietet ein vielseitiges,
interkulturelles Angebot und war bereits zweimal Veranstaltungsort für
die Arabisch-deutschen Literaturtage. 2020 wurde die Bibliothek zudem
mehrfach für ihr soziales und kulturelles Engagement ausgezeichnet.
\enquote{Baynatna} ist eine vielversprechende, aufstrebende Bibliothek,
die einen wichtigen Beitrag für Kultur und Vielfältigkeit in Berlin
leistet. (ak)

\begin{center}\rule{0.5\linewidth}{0.5pt}\end{center}

\emph{Denkmalstürze? Black Lives Matter und Postkolonialismus in der
Popkultur} (2020).
\url{https://www.arte.tv/de/videos/093211-019-A/tracks}.

Artes Popkulturmagazin Tracks widmet sich dem Thema des Dekolonisierens
öffentlicher Räume. Blinde Flecken deutscher Geschichte werden durch die
Aktivist*innen für die Öffentlichkeit sichtbar. Ein Denkmal Bismarcks in
Kamerun wird künstlerisch so inszeniert, dass sein Geist aus dem Land
vertrieben wird. Bismarck, so die Kuratorin Pascale Obolo, zog die
Grenzen Kameruns zur Zeit der Kolonisierung. In Deutschland hingegen
politisiert sich ein Teil der Graffitiszene. Illegal wird ein Berliner
U-Bahnhof zu einem Ort des Erinnerns der deutschen Kolonialgeschichte
umgestaltet. Weitere Akteure des Beitrags sind das aktivistische
Künstlerkollektiv \enquote{Rocco und seine Brüder}. Ein Plakat im
Berliner Stadtraum lenkt die Aufmerksamkeit auf den Genozid an den Nana
und Hereros. Der Versuch einer Wiedergutmachung durch die
Bundesregierung wird ironisch inszeniert. Die Regierung in Windhuk
lehnte die Zahlung von zehn Millionen Euro für den Mord an über 80.000
Menschen ab. Das Format zeigt, wie geschichtliche Aufarbeitung und
Dekolonisierung auch popkulturell aufgegriffen werden. (fj)

\begin{center}\rule{0.5\linewidth}{0.5pt}\end{center}

Wiedemann, Charlotte (2019). \emph{Der lange Abschied von der weißen
Dominanz.} ttt -- titel, thesen, temperamente. Dokumentarfilm.
\url{https://www.daserste.de/information/wissen-kultur/ttt/videos/Der-lange-Abschied-von-der-weissen-Dominanz-wiedemann-senudng-vom-17112019-video-100.html}.

Die Sachbuchautorin Charlotte Wiedemann stellt die Mechanismen des
\enquote{Weiß-seins} als Machtposition dar. Aus diesem Blickwinkel wird
die Darstellung und Aufarbeitung der Geschichte reflektiert.
\enquote{Weißes Erbe gilt es aufzuarbeiten und neu einzuordnen} und
dieser Aufarbeitungsprozess schafft ein Bewusstsein für tief verankerte
\enquote{Machtverhältnisse}. Ein Beispiel ist die Verteilung der
weltweiten Finanzen. Trotz vorhandener Ressourcen im globalen Süden ist
der Wohlstand im globalen Norden zu finden. Die weiße
Dominanzgesellschaft ist auch eine von Unterdrückern und Herrschenden.
Das Übernehmen von Verantwortung und eine Distanzierung von einer primär
eurozentrischen Sichtweise wäre ein Anfang. (fj)

\hypertarget{abschlussarbeiten}{%
\section{7. Abschlussarbeiten}\label{abschlussarbeiten}}

Haupka, Nick (2021). Analyse der Entwicklung des Open
Access-Discovery-Services Unpaywall seit 2018.
\url{https://doi.org/10.25968/OPUS-1899}.\\
Hobert, A., Haupka, N., \& Najko, N. (2021). Entwicklung und Typologie
des Datendiensts Unpaywall. Preprints 2021 der Zeitschrift BIBLIOTHEK --
Forschung und Praxis, \url{https://doi.org/10.18452/22728}.

In seiner Bachelorarbeit an der Hochschule Hannover beschäftigt sich
Nick Haupka mit dem Datendienst Unpaywall, der seit dem Start 2017
(damals noch unter dem Namen oaDOI) quasi unerlässlich für alle geworden
ist, die den Anteil und die Art des Open Access für eine bestimmte Menge
an Publikationen untersuchen wollen, oder Daten zu OA-Status und Links
zu OA-Versionen in eigenen Such- beziehungsweise Discovery-Systemen
einbinden wollen. (In der LIBREAS nutzen wir Unpaywall unter anderem für
die vorliegende Kolumne, um gezielt frei zugängliche Versionen direkt
verlinken zu können, sollte die Publikation nicht über den Verlags
selbst Open Access sein.) Nick Haupka wertete Unpaywall-Daten aus
verschiedenen Zeiträumen aus; entsprechend kann er zu verschiedenen
inhaltlichen Entwicklungen (zum Beispiel wie sich die Anteile der
verschiedenen \enquote{OA-Farben} über die Zeit entwickelt haben) wie
auch methodischen Aspekten (Veränderung des Unpaywall-Datenschemas, die
Behandlung sogenannter Paratexte) Schlüsse ziehen.\\
Die Ergebnisse der Untersuchung, welche dieser sehr lesenswerten
Abschlussarbeit vorausgegangen ist, werden auch in der Zeitschrift
Bibliothek: Forschung und Praxis präsentiert (aktuell als Preprint
verfügbar). (mv)

\hypertarget{weitere-medien}{%
\section{8. Weitere Medien}\label{weitere-medien}}

Colovini, Leo (studiogiochi); Streese, Folko; Kreativbunker (2020).
\emph{Die verlassene Bibliothek. Ein Escape-Spiel.} Kempen: moses.
Verlag GmbH {[}Brettspiel{]}.

Bei Escape- oder Exit-Spielen für zuhause handelt es sich zumeist um
Brett- oder Kartenspiele, welche auf Rätseln basieren. Diese Rätsel
müssen gelöst werden, um, meist innerhalb einer vorgegebenen Zeit, aus
einem Raum oder einer Situation zu entkommen und somit das Spiel zu
\enquote{gewinnen}.

Im Fall von \enquote{Die verlassene Bibliothek} muss man in einem Team
von bis zu vier Personen aus einer plötzlich verschlossenen Bibliothek
entkommen. Es müssen alle 18 Aufgaben gelöst werden, um den Code für die
Ausgangstür des Gebäudes zu finden und somit entkommen zu können. Das
Setting der Bibliothek scheint zunächst sehr ansprechend, jedoch sind
die Aufgaben eher mathematischer Natur, was für manche das Spielerlebnis
trüben könnte. (jlb)

\begin{center}\rule{0.5\linewidth}{0.5pt}\end{center}

Amer, Karim ; Noujaim, Jehane (2019). \emph{The Great Hack. Cambridge
Analytica großer Hack}. {[}Dokumentarfilm{]}

Sind wir übers Netz manipulierbar? Im Frühjahr 2018 kommt es zu einem
großen Datenskandal von Facebook und dem Datenanalyse-Unternehmen
Cambridge Analytica, bei dem die persönlichen Daten von mehr als 85
Millionen Facebook-Nutzern für politische Zwecke ausgewertet und
missbraucht werden. Der wahre Einfluss der Firma bleibt zwar umstritten,
fest steht jedoch, dass sie zu Zeiten der Brexit-Abstimmung 2016 und des
US-Wahlkampfes 2016 große Datenmengen dazu nutzte, um mit Hilfe von
Online-Kampagnen und Anzeigen die Meinung der Menschen zu beeinflussen
und zu manipulieren. Der Film ist als Weckruf für alle gedacht, die noch
nicht begriffen haben, dass die eigenen Daten in den falschen Händen zu
einer Waffe werden können. \enquote{Wer von Ihnen hat schon mal eine
Werbeanzeige gesehen, bei der er gedacht hat, das Mikrofon seines
Smartphones hat gerade mitgehört? -- Sie glauben, wir werden belauscht?
Tatsächlich ist es aber so, dass ihr Verhalten richtig kalkuliert wird},
sagt der Medienwissenschaftler David Carroll. Eine denkbare Rolle der
Bibliothek besteht darin, die Medienkompetenz der Menschen dahingehend
zu schulen, dass sie um die Manipulierbarkeit von Social Media wissen
und sich davor schützen. (lf)

\begin{center}\rule{0.5\linewidth}{0.5pt}\end{center}

\emph{Die Comicbibliothek \enquote{Renate}}, Tucholskystr. 32, 10117
Berlin. \url{http://www.renatecomics.de/sites/kurse.htm}.

Wer Comics mag, kennt wahrscheinlich auch Renate, die Comicbibliothek.
Diese spezielle Bibliothek ist im Herzen von Berlin zu finden, wird mit
viel Liebe von Ehrenamtlichen geführt und hat über die Jahre über 15.000
Bände angesammelt. Im Jahr 1989 wurde Renate als Fanzine von
Comicinteressierten gegründet. Sie ist bis heute die einzige reine
Comicbibliothek in Deutschland. Inzwischen gehört zur Comicbibliothek
auch ein kleiner Shop, ein Magazin, das einmal im Jahr erscheint, ein
Kurs, der einmal im Monat angeboten wird und ein Blog, der regelmäßige
und interessante Einträge über die Comicwelt veröffentlicht. Der Laden
zur Bibliothek bietet viele weniger bekannte Comics auf Französisch,
Spanisch aber auch auf Deutsch und Englisch an. Diese selteneren Stücke
stammen vor allem von Midi- und Miniverlagen. Durch diesen Verkauf wird
der Erhalt von der Bibliothek Renate gesichert. Der monatliche Kurs gibt
den Teilnehmer:innen die Möglichkeit, mehr über die Comic-Kunst zu
lernen. Hier werden Techniken zur Figurenentwicklung und Spannungsbogen
vermittelt. Zum Schluss entsteht immer ein neues Comicheft. Diese sehr
besondere Bibliothek arbeitet oft mit Schulen zusammen, damit die
Comicwelt auch für die nächste Generation interessant bleibt. (kfk)

\begin{center}\rule{0.5\linewidth}{0.5pt}\end{center}

Schuldt, Karsten (2020). \emph{Katastrophenplanung für Öffentliche
Bibliotheken. Eine Literaturübersicht.} Schweizerisches Institut für
Informationswissenschaft, 21.06.2020.
\url{https://nbn-resolving.org/urn:nbn:de:0290-opus4-173908}.

In seinem Vortrag präsentiert Karsten Schuldt eine Literaturübersicht
zur Katastrophenplanung für öffentliche Bibliotheken. Seine Recherche
ergab, dass Katastrophenplanung und ein Training für öffentliche
Bibliotheken zwingend notwendig sind, um im Ernstfall einen geregelten
Ablauf sicherzustellen. Auch sollen größere Bibliotheken kleinere
unterstützen, so dass im Ernstfall auch diese reagieren können und
wissen, wer im Ernstfall welche Aufgaben übernimmt. Unterstützungen
können in den Bereichen der Community, Bildung und des Trainings
erfolgen. Während des Vortrags zählt Karsten Schuldt einige
Katastrophenszenarien auf: Naturkatastrophen, gesellschaftliche
Auseinandersetzungen, Pandemien aber auch Wasserschäden, die in der
Einrichtung selbst auftreten. Zudem spielen der Faktor Mensch und sein
Verhalten während einer solchen Katastrophe eine entscheidende Rolle.
Eine Planung sollte sich an zwei Zielen ausrichten: 1. die
Wahrscheinlichkeit des Überstehens einer Krise zu steigern, 2. mögliche
Schäden klein halten, damit eigentliche Funktion eines Betriebs
schnellstmöglich wiederhergestellt werden kann. Hierfür kann das
Verfahren der Modellierung von Katastrophenszenarien herangezogen. (kmg)

\begin{center}\rule{0.5\linewidth}{0.5pt}\end{center}

Nguyen-Kim, Mai Thi (maiLab) (2020). \emph{Corona hat meine Meinung
geändert}. \url{https://www.youtube.com/watch?v=Nn2rJrKwENI}.

Mai Thi Nguyen-Kim ist enttäuscht. Vor Corona war sie der festen
Überzeugung für mehr wissenschaftliche Aufklärung bräuchte es mehr
Wissenschaftler*innen in den Medien. Corona hätte sie eines besseren
belehrt. Ein Problem ist, dass die Expertise von Wissenschaftler*innen
häufig unangefochten bleibt, doch diese auch nur Menschen sind und somit
menschlichen Fehlern unterliegen. Daher ist es immer wieder nötig
Kontrollinstrumente einzuführen um sichere und neutrale Wissenschaft zu
erzeugen. Zu dieser gehören die Überprüfung der wissenschaftlichen
Methoden, die kritischen Kolleg*innen und die Kontextualisierung des
eigenen wissenschaftlichen Beitrags. Das funktioniere innerhalb des
Wissenschaftssystem recht gut, in der breiten Öffentlichkeit jedoch
weniger. Hier erfahren diese \enquote{Expert*innen}, ohne weitere
Kontrollen für ihre These meist mehr Glaubwürdigkeit als andere
Akteur*innen, was für die Durchsetzung persönlicher Meinungen und
Ideologien genutzt werden kann. Dies wird durch die Medien verstärkt,
die häufig ausschnittshaft kontroverse Meinungen öffentlichkeitswirksam
inszenieren. Wie man am Beispiel des Diskurses um COVID-19 sehen kann,
sorgt dieses unhinterfragte Vertrauen für eine zusätzliche Verwirrung.
Deshalb braucht es eine Qualitätskontrolle der
Wissenschaftskommunikation.

Überzeugend ist für mich das Feingefühl, das die
Wissenschaftsjournalistin bei der Kombination aus wissenschaftlicher
Genauigkeit und unterhaltender Erzählung anschlägt. Mai scheut sich
nicht vor umgangssprachlichen Begriffen wie \enquote{Bullshit} oder
\enquote{Knalltüte}. Auch ihre Beispiele, etwa die detaillierten
Überlegungen über eine Studie zur Wirksamkeit von Stangenlauch zur
\enquote{Coronaheilung}, wirken erst einmal absurd. Die Journalistin
schafft es jedes dieser Beispiele wissenschaftlich Kontrollmechanismen
sauber zu analysieren und anschließend zu übertragen, so dass es ihr
gelingt auch vermeintlich komplizierte Sachverhalte verständlich zu
erklären. Gestützt wird ihre Argumentation durch eine umfassende
Quellenliste unter ihren Videos. Zudem überzeugt Mai immer durch
kontinuierliche Selbstreflektion. In einer ganzen Sequenz ordnet sie
ihre eigene journalistische Arbeit ein und findet immer wieder elegant
Lösungen bei Kontroversen nicht den Verursacher*innen selbst, sondern
den wissenschaftlichen Kontexten eine Bühne zu geben. (vs)

%autor

\end{document}
