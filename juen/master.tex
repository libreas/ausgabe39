\documentclass[a4paper,
fontsize=11pt,
%headings=small,
oneside,
numbers=noperiodatend,
parskip=half-,
bibliography=totoc,
final
]{scrartcl}

\usepackage[babel]{csquotes}
\usepackage{synttree}
\usepackage{graphicx}
\setkeys{Gin}{width=.4\textwidth} %default pics size

\graphicspath{{./plots/}}
\usepackage[ngerman]{babel}
\usepackage[T1]{fontenc}
%\usepackage{amsmath}
\usepackage[utf8x]{inputenc}
\usepackage [hyphens]{url}
\usepackage{booktabs} 
\usepackage[left=2.4cm,right=2.4cm,top=2.3cm,bottom=2cm,includeheadfoot]{geometry}
\usepackage{eurosym}
\usepackage{multirow}
\usepackage[ngerman]{varioref}
\setcapindent{1em}
\renewcommand{\labelitemi}{--}
\usepackage{paralist}
\usepackage{pdfpages}
\usepackage{lscape}
\usepackage{float}
\usepackage{acronym}
\usepackage{eurosym}
\usepackage{longtable,lscape}
\usepackage{mathpazo}
\usepackage[normalem]{ulem} %emphasize weiterhin kursiv
\usepackage[flushmargin,ragged]{footmisc} % left align footnote
\usepackage{ccicons} 
\setcapindent{0pt} % no indentation in captions

%%%% fancy LIBREAS URL color 
\usepackage{xcolor}
\definecolor{libreas}{RGB}{112,0,0}

\usepackage{listings}

\urlstyle{same}  % don't use monospace font for urls

\usepackage[fleqn]{amsmath}

%adjust fontsize for part

\usepackage{sectsty}
\partfont{\large}

%Das BibTeX-Zeichen mit \BibTeX setzen:
\def\symbol#1{\char #1\relax}
\def\bsl{{\tt\symbol{'134}}}
\def\BibTeX{{\rm B\kern-.05em{\sc i\kern-.025em b}\kern-.08em
    T\kern-.1667em\lower.7ex\hbox{E}\kern-.125emX}}

\usepackage{fancyhdr}
\fancyhf{}
\pagestyle{fancyplain}
\fancyhead[R]{\thepage}

% make sure bookmarks are created eventough sections are not numbered!
% uncommend if sections are numbered (bookmarks created by default)
\makeatletter
\renewcommand\@seccntformat[1]{}
\makeatother

% typo setup
\clubpenalty = 10000
\widowpenalty = 10000
\displaywidowpenalty = 10000

\usepackage{hyperxmp}
\usepackage[colorlinks, linkcolor=black,citecolor=black, urlcolor=libreas,
breaklinks= true,bookmarks=true,bookmarksopen=true]{hyperref}
\usepackage{breakurl}

%meta
%meta

\fancyhead[L]{S. Juen\\ %author
LIBREAS. Library Ideas, 39 (2021). % journal, issue, volume.
\href{http://nbn-resolving.de/}
{}} % urn 
% recommended use
%\href{http://nbn-resolving.de/}{\color{black}{urn:nbn:de...}}
\fancyhead[R]{\thepage} %page number
\fancyfoot[L] {\ccLogo \ccAttribution\ \href{https://creativecommons.org/licenses/by/4.0/}{\color{black}Creative Commons BY 4.0}}  %licence
\fancyfoot[R] {ISSN: 1860-7950}

\title{\LARGE{Feminismus, Algorithmen, Gender-Data-Gap und was das alles mit Bibliotheks- und Informationswissenschaft zu tun hat}}% title
\author{Sara Juen} % author

\setcounter{page}{1}

\hypersetup{%
      pdftitle={Feminismus, Algorithmen, Gender-Data-Gap und was das alles mit Bibliotheks- und Informationswissenschaft zu tun hat},
      pdfauthor={Sara Juen},
      pdfcopyright={CC BY 4.0 International},
      pdfsubject={LIBREAS. Library Ideas, 39 (2021).},
      pdfkeywords={Algorithmen, Bibliotheksethik, Big Data, Feminismus},
      pdflicenseurl={https://creativecommons.org/licenses/by/4.0/},
      pdfcontacturl={http://libreas.eu},
      baseurl={http://libreas.eu},
      pdflang={de},
      pdfmetalang={de}
     }



\date{}
\begin{document}

\maketitle
\thispagestyle{fancyplain} 

%abstracts
\begin{abstract}
\noindent
\textbf{Kurzfassung}: In den letzten Jahren gab es vermehrt Berichte
über unethisches Handeln in Technologiekonzernen und Hinweise darauf,
dass Algorithmen nicht neutral handeln, sondern gesellschaftlich
bestehende Ungerechtigkeiten weiterführen oder diese sogar verschärfen.
Der vorliegende Artikel gibt einen Einblick in diese Thematik und zeigt
auf, warum dieses Thema für die Bibliotheks- und
Informationswissenschaft relevant ist.

\begin{center}\rule{0.5\linewidth}{0.5pt}\end{center}

\noindent\textbf{Abstract}: In recent years, there have been increasing reports
about unethical actions in technology corporations and indications that
algorithms do not act neutrally, but rather continue existing social
injustices or even intensify them. This article provides insight into
this issue and shows why this topic is relevant to library and
information science.
\end{abstract}

%body
Wir sitzen an unserem Computer oder Smartphone und scrollen durch
vorgeschlagene Angebote. Wir hören uns Musik an, die uns, basierend auf
unserer Playlist, empfohlen wird und lassen uns von einem
Navigationssystem durch den Großstadtdschungel führen. Wenn wir eine
Auskunft zu unserer Kreditwürdigkeit machen müssen, geben wir einen
Score weiter, der von einem Algorithmus berechnet wurde. Mittlerweile
ist es für uns normal, dass wir von maschinell lernenden Algorithmen
(auch "schwache Künstliche Intelligenz" genannt) umgeben sind. Sie
schlagen uns Produkte und Dienstleistungen vor, die wir eventuell kaufen
wollen, die Sozialen Medien buhlen um unsere Aufmerksamkeit und
Bildschirmzeit und Alexa reagiert auf unsere Anfragen. Und das alles
basierend auf Daten, die über uns gesammelt wurden oder aus abgeleiteten
Daten von Anderen. Im Allgemeinen wird angenommen, dass datenbasierte
Software neutral und unvoreingenommen ist und somit gewisse
Entscheidungen \enquote{besser} treffen kann als der Mensch.

Seit einigen Jahren gibt es einen stetig größer werdenden Diskurs und
zunehmend Untersuchungen darüber, ob diese Annahme der Neutralität
tatsächlich zutrifft. Mittlerweile wurde die Vermutung, dass Algorithmen
neutrale(re) Entscheidungen treffen, vielfach widerlegt (vgl. O'Neil
2017; Zweig 2019). Tatsächlich verstärken datenbasierte
Entscheidungssysteme bestehende Diskriminierungen und Benachteiligungen
eher noch als sie zu verringern (vgl. Zweig 2019, S. 211).
Funktionstüchtige Entscheidungssysteme müssen nicht nur programmiert
werden, sondern auch trainiert. Damit ein Algorithmus arbeiten kann,
braucht er eine enorme Menge an Trainingsdaten, anhand derer er
\enquote{lernt} Entscheidungen zu treffen. Die Trainingsdaten, die der
Algorithmus bekommt, sind bereits vorhandene Daten aus der
Vergangenheit. Diese Datengrundlage kann schon an sich problematisch
sein, da die bereits bestehenden Daten sehr wahrscheinlich die
Wertvorstellungen und Diskriminierungen repräsentieren, die in der
Gesellschaft herrschen, in der sie erhoben wurden. So werden bereits
vorbelastete Daten durch den Algorithmus zu vermeintlich neutralen
Datenströmen. In Wirklichkeit werden aber die eingebetteten
Benachteiligungen reproduziert und verstärkt. Das folgende Beispiel aus
den USA macht diesen Prozess deutlich.

In den USA wird ein datenbasiertes Bewertungssystem bei Strafverfahren
verwendet. Dieses soll den Richter\_innen dabei helfen, anhand eines
berechneten Risiko-Scores für die angeklagte Person die angemessene
Strafe zu fällen. Diesem algorithmischen Bewertungssystem liegen Daten
zugrunde, die seit 1995 anhand eines Fragebogens erhoben werden. Der
Fragebogen, der von den festgenommenen Personen selbst ausgefüllt werden
muss, stellt allgemeine Fragen zu der Person, aber auch sehr
persönliche. Fragen, wie oft die angeklagte Person schon mit der Polizei
zu tun hatte oder in welchem Alter das erste Mal Kontakt mit der Polizei
bestand, können schon eine Diskriminierung auslösen, bedenkt man dabei,
dass BIPoC (Black, Indigenous and People of Color) sehr viel häufiger
von der Polizei angehalten und kontrolliert werden als weiße Personen.
Des Weiteren enthält der Fragebogen auch Fragen zum persönlichen Leben,
zum Beispiel wie man aufgewachsen ist, wie die Familienkonstellation war
und Fragen zu Freunden. Fragen, die bei einer Straftat keine Rolle
spielen sollten, sich aber negativ auf die Bewertung auswirken können.
Mehrere Studien haben bewiesen, dass dieses System BIPoC benachteiligt
und sie höhere Strafen bekommen als weiße Menschen mit vergleichbaren
Vergehen (vgl. O'Neil 2017, S. 25 ff).

In dem Dokumentarfilm \enquote{Coded Bias} von Shalini Kantayya erzählt
die MIT Computerwissenschaftlerin Joy Buolamwini, wie ihr während der
Arbeit im Computer Lab aufgefallen ist, dass eine
Gesichtserkennungssoftware ihr Gesicht nur erkannte, wenn sie sich eine
weiße Maske aufsetzte. Aufgrund dieses Erlebnisses unternahm Joy
Buolamwini weitere Untersuchungen dazu und fand heraus, dass auch andere
Gesichtserkennungssoftware dunklere Haut schlechter erkannte als hellere
Haut und dass die Systeme bei Frauen fehleranfälliger waren als bei
Männern. Diese Defizite der Software konnten auf die unausgewogenen
Trainingsdaten, mit denen der Algorithmus gefüttert wurde, zurückgeführt
werden. Circa zwei Drittel der Trainingsdaten bestanden aus Fotos von
hellhäutigen Männern, nur ein geringer Anteil waren Frauengesichter und
noch weniger waren dunkelhäutige Frauen (vgl. Kantayya 2020). Diese
Beispiele zeigen deutlich, dass datenbasierte Systeme nicht automatisch
neutrale oder gerechtere Entscheidungen treffen als Menschen. Die
Systeme werden mit Daten trainiert, welche die bereits vorhandenen
Benachteiligungen und Unterrepräsentationen weiterführen.

In ihrem Buch \enquote{Ein Algorithmus hat kein Taktgefühl -- Wo
künstliche Intelligenz sich irrt, warum uns das betrifft und was wir
dagegen tun können} beschreibt die Informatikprofessorin Katharina Zweig
anschaulich, wie Algorithmen und datenbasierte Systeme funktionieren,
was sie gut können und wo ihre Grenzen liegen (vgl. Zweig 2019). Laut
Zweig können algorithmische Entscheidungssysteme keine \enquote{Lösung}
von gesellschaftlichen Diskrepanzen sein. Sie macht dies unter anderem
an dem Beispiel von Amazon deutlich. Amazon arbeitete in 2014 mit einem
datenbasierten Bewerbungsbewertungssystem mit dem Ziel, mögliche
Diskriminierung durch Menschen im Bewerbungsprozess auszuschließen. Als
Trainingsdaten wurden die Bewerbungsunterlagen der vergangenen Jahre
verwendet. Das Geschlecht wurde nicht in die Trainingsdaten eingespeist.
Dabei wurde allerdings vergessen, dass in der Vergangenheit vor allem
Männer erfolgreich eingestellt wurden und dass Frauen in großen
Tech-Firmen unterrepräsentiert sind. So begann der Algorithmus damit,
Frauen auszusortieren, wenn zum Beispiel aus ihren Unterlagen
hervorging, dass sie an einem Frauen-College studiert hatten. Der
Algorithmus hatte aufgrund seiner Trainingsdaten gelernt, dass dieses
Merkmal nicht oft zu einer erfolgreichen Einstellung geführt hat und
sortierte die Bewerbung deswegen aus (vgl. ebd. S. 212).

Die Beispiele von Joy Buolamwini und Katharina Zweig führen uns zu einem
weiteren Punkt, der bei der Arbeit mit Daten oft zu wenig Aufmerksamkeit
bekommt. Es handelt sich um die Leerstellen, die Vergessenen, die
blinden Flecken. Die, die nicht mitgedacht, nicht mitgezählt oder
schlichtweg vergessen werden. Sind keine Daten über sie vorhanden,
existieren sie nicht. In der Vergangenheit bis hin in die Gegenwart
betrifft dies in vielen Bereichen minorisierte Gruppen und die größte
Minderheit der Welt: die Frauen. Momentan machen Frauen 49,5 \% der
Weltbevölkerung aus (vgl. Country Meters 2021, o. S.); trotzdem werden
sie in vielen Bereichen, unter anderem in der Stadtplanung, der Medizin
oder im Design, nicht (genügend) berücksichtigt. Dies kann, abgesehen
von ärgerlichen Zwischenfällen im Alltag, auch schwerwiegende und
lebensgefährliche Folgen haben. In ihrem Buch \enquote{Unsichtbare
Frauen -- Wie eine von Daten beherrschte Welt die Hälfte der Bevölkerung
ignoriert} (vgl. Criado-Perez 2020) geht Caroline
Criado-Perez\footnote{Über die Autorin wird in den sozialen Medien
  diskutiert. Auf Grund von Aussagen aus der Vergangenheit und ihrem
  festhalten am binären Geschlechtersystem, wird ihr vorgeworfen
  Trans-Menschen nicht mit einzubeziehen.} ausführlich auf dieses Thema
ein und schildert die Probleme, die durch das Phänomen der
Gender-Data-Gap ausgelöst werden. Im Vorwort ihres Buches erklärt sie,
was sich hinter diesem Begriff verbirgt. In der Geschichtsschreibung
klafft eine große Lücke der Sichtbarkeit von Frauen. Die Frauen nehmen
\enquote{weder in kultureller noch in biologischer Hinsicht viel Platz
ein. Stattdessen galten männliche Lebensläufe als repräsentativ für alle
Menschen} (Criado-Perez 2020, S. 11). Dies betrifft aber nicht nur die
Menschheitsgeschichte, sondern zeigt sich auch in vielen anderen
Bereichen des Lebens. Diese Lücke nennt Criado-Perez die
Gender-Data-Gap. Um zu verdeutlichen, wie weit die Gender-Data-Gap
reicht, wird hier ein Beispiel vorgestellt. Die Schlangen, die sich in
den Pausen von Veranstaltungen oder an anderen öffentlichen Orten vor
Frauentoiletten bilden, sind wahrscheinlich allen schon aufgefallen,
entweder weil sie daran vorbei gegangen sind oder sie selbst in der
Schlange standen. Der Platz für Toiletten wird für Männer und Frauen in
gleicher Quadratmeterzahl zur Verfügung gestellt. Dies erscheint gerecht
und logisch. Allerdings gibt es in Männertoiletten Urinale, die weniger
Platz einnehmen als Kabinen, weshalb mehr Menschen gleichzeitig auf die
Toilette gehen können. Darüber hinaus benötigen Frauen beim Gang zur
Toilette mehr Zeit, weil sie vermehrt von Kindern, älteren oder
behinderten Personen begleitet werden. Auch die Periode oder
Schwangerschaft beeinflusst die Dauer und Anzahl von Toilettenbesuchen.
Bei der Planung der meisten öffentlichen Toiletten werden diese
Bedürfnisse allerdings nicht berücksichtigt und so entstehen am Ende
Warteschlangen vor WCs für Frauen (Criado-Perez 2020, S. 75 f).

Von immer mehr Wissenschaftler\_innen wird diese scheinbar
\enquote{neutrale} und \enquote{objektive} Sichtweise grundsätzlich in
Frage gestellt (vgl. Criado-Perez 2020; D'Ignazio, Klein 2020; Roig
2021). Die Welt, wie wir sie kennen, wurde fast vollständig von weißen
Männern mit einer eurozentristischen Sicht erschaffen. Da diese
Sichtweise sehr tief in der Gesellschaft verankert ist, wird sie
fälschlicherweise als \enquote{neutral} und \enquote{objektiv}
wahrgenommen. Daran anschließend kann auch das Beispiel des generischen
Maskulinums als \enquote{geschlechtsneutraler} Schreibweise genannt
werden.

Ein weiteres Buch, das sich mit der Datendiskrepanz auseinandersetzt,
ist \enquote{Data Feminism} der Autorinnen Catherine D'Ignazio und
Lauren F. Klein (vgl. D'Ignazio, Klein 2020). Anhand von zahlreichen
Beispielen aus den unterschiedlichsten Wissenschaftsbereichen zeigen die
beiden Autorinnen auf, wie die Ungleichheit in den Daten entstanden ist
und wie dies in Zukunft vermieden werden könnte. Entlang der sieben
Prinzipien des Data Feminism, welche die Autorinnen erarbeitet haben,
führen sie durch ihr Buch und die einzelnen Themenkomplexe. Im Folgenden
werden die sieben Prinzipien kurz vorgestellt, da sie sehr deutlich
zeigen, wie umfassend die feministische Sichtweise auf Forschungsarbeit
sein kann.

1. Prinzip -- Examine Power \emph{(Die Macht
untersuchen}\footnote{Die kursiv geschriebenen deutschen Überschriften
  sind eigene Übersetzungen.}\emph{)}

Beim ersten Prinzip geht es darum, die Wer-Fragen zu stellen. Wer
profitiert von den erhobenen Daten? Wer nicht? Wer macht die Arbeit? Wer
wird außen vorgelassen? Wer sammelt die Daten und wessen Daten werden
gesammelt? Wessen Prioritäten werden schlussendlich in einem Produkt
oder den Ergebnissen der wissenschaftlichen Arbeit berücksichtigt? Und
wessen nicht? (vgl. ebd. S. 47)

2. Prinzip -- Challenge Power \emph{(Die Macht herausfordern)}

Aufbauend auf dem ersten Prinzip, wird hier danach gefragt, wie
ungleiche Machtstrukturen angegangen werden können. Außerdem, wie
Systeme, Studien und Forschungsergebnisse auf Ungleichheit untersucht
und wie diese gerechter gestaltet werden können (vgl. ebd. S. 72).

3. Prinzip -- Elevate emotion and embodiment \emph{(Beachten von
Gefühlen und Ausdrucksformen)}

Hier stellen sich die Autorinnen die Frage, wie Daten visualisiert
werden. Von Statistiken und Datenvisualisierungen wird besonders
angenommen, dass sie neutral und objektiv seien. Auch wird immer wieder
betont, wie wichtig eine nüchterne Präsentation von Daten ist. Emotionen
und Menschlichkeit haben, in der Regel, in statistischen, visuellen
Darstellungen keinen Platz. Hier weisen die Autorinnen darauf hin, dass
jede Datenvisualisierung aus einer bestimmten Perspektive vorgenommen
wird und somit nie ganz neutral oder objektiv sein kann (vgl. ebd. S.
95 f).

4. Prinzip -- Rethink binaries and hierarchies \emph{(Binaritäten und
Hierarchien überdenken)}

\enquote{Was gezählt wird, zählt}, sagte die feministische Geografin
Joni Seager (vgl. ebd. S. 97). Bei Prinzip 4 regen die Autorinnen an,
Klassifikationssysteme zu hinterfragen. Klassifikationen spiegeln die
Werte, aber genauso die Vorurteile einer Gesellschaft wider (vgl. ebd.
S. 123). Ein allgegenwärtiges Beispiel dafür sind Anmeldeformulare, in
denen das Geschlecht angegeben werden muss, die Auswahl sich aber auf
die weibliche und männliche Option beschränkt. In diese Formulare können
sich Intersexuelle oder Non-Binäre Menschen nicht wahrheitsgetreu
eintragen, werden somit auch nicht mitgezählt und in den Ergebnissen der
Untersuchungen oder in offiziellen Statistiken der Länder auch nicht
repräsentiert.

5. Prinzip -- Embrace pluralism \emph{(Diversität fördern)}

Unsere Gesellschaft ist vielschichtig und divers. Wenn Studien
durchgeführt werden, sollte darauf geachtet werden, dass alle
Betroffenen der Untersuchung \emph{und} der Ergebnisse mit einbezogen
und mitgedacht werden. Die Personen, die sich am besten mit der
untersuchten Angelegenheit auskennen, sind die Personen deren
Angelegenheit untersucht wird. Die lokale Zusammenarbeit soll dabei
berücksichtigt und gefördert werden, genauso wie die Diversität in der
Forschung selbst (vgl. ebd. S. 147 f).

6. Prinzip -- Consider context \emph{(Den Kontext beachten)}

Die Autorinnen weisen bei diesem Prinzip darauf hin, dass Daten nicht
neutral und objektiv sind, sondern immer in einem bestimmten Kontext
entstehen und auch in diesem gelesen werden sollten. Zahlen können nicht
für sich selbst sprechen. Der Kontext sollte stets miteinbezogen,
dargestellt und sichtbar gemacht werden. Außerdem sollte die Frage
danach gestellt werden, wie der ursprüngliche Kontext war, wenn Daten
nachgenutzt werden (vgl. ebd. S. 171 f).

7. Prinzip -- Make labor visible \emph{(Die Arbeit sichtbar machen)}

Im siebten und somit letzten Prinzip des Data Feminism, erinnern die
Autorinnen daran, dass fertiggestellte Arbeit, die Arbeit vieler ist.
Sie regen an, sich die Fragen zu stellen: Durch wie viele Hände geht ein
Produkt/eine Untersuchung bis es/sie fertig ist? Die Arbeit von welchen
Menschen wird womöglich vergessen/nicht gezeigt? Die Arbeit Anderer
sichtbar zu machen, bedeutet andere Menschen zu unterstützen, in den
Fokus zu rücken und ihren Anteil an der Arbeit zu würdigen (vgl. ebd. S.
200 f).

Zusammenfassend arbeiten die sieben Prinzipien des Data Feminism von
Catherine D'Ignazio und Lauren F. Klein einen breiten Rahmen aus, wie
Forschung gerechter werden kann. Das Buch ist Open Access
erhältlich\footnote{data-feminism.mitpress.mit.edu/}.

Was hat das nun alles mit der Bibliotheks- und Informationswissenschaft
zu tun?

Nun, unsere Disziplin ist darauf spezialisiert, mit Informationen zu
arbeiten, die mittlerweile meistens digital sind. Informationen sind
unser Forschungsgegenstand, unser Hauptaugenmerk. Ob wir nun
untersuchen, wie Menschen, Informationen und Technologien interagieren
oder wie Informationen gefunden, aufbereitet und nutzbar gemacht werden
können. Oder ob wir über die Relevanz von Informationssystemen und den
Aufbau von Datenbanken nachdenken. Bis hin zu der Nutzung von
Klassifikationen und Bibliothekskatalogen in der eigenen Forschung, aber
auch als Endprodukt für Nutzer\_innen. Seit jeher sind Bibliotheken
Meisterinnen der Katalogisierung und der Klassifikationen. Diesen
Systemen liegen Wertvorstellungen zugrunde, und zwar nicht irgendwelche,
sondern die Werte, die die Erschaffenden dieser Systeme hatten. Wie
verschiedene Klassifikationssysteme zeigen (z. B. die
Dewey-Dezimalklassifikation), repräsentieren sie eine gewisse
Weltanschauung und verweisen auf den Zeitgeist, in welchem diese Systeme
erstellt wurden. Erneuerungen und Anpassungen an die sich verändernden
Werte werden zwar vorgenommen, meistens aber sehr verzögert und nicht
umfassend. Zum Beispiel wurde erst in der 22. Fassung der
Dewey-Dezimalklassifikation von 2006 der Begriff \enquote{rassisch}
geändert, sowie die Ausschließung der Notationen für Rassen vorgenommen
(vgl. Chan, Mitchel 2006, S. 186).

Abgesehen von den ethischen Fragen, die sich unsere Disziplin in Bezug
auf Daten und Informationen stellen muss, sollte auch nicht vergessen
werden, dass wir uns in einem System mit Machtstrukturen bewegen. Wir
sollten uns vergegenwärtigen, dass es auch im Kontext des
Bibliothekswesens und der Bibliotheks- und Informationswissenschaft
Personen oder Gruppen gibt, die wenig bis keine Repräsentation erfahren
und diese in Zukunft in unsere Forschung und in die Entwicklung von
neuen Konzepten in Bibliotheken mit einbeziehen (Stichwort: Diversity
Management).

Abschließend kann gesagt werden, dass die feministische oder die
ganzheitliche Sicht auf die Arbeit mit Daten im ersten Moment vielleicht
komplizierter und aufwändiger erscheint als altbewährte Strategien,
allerdings die so erarbeitete Forschung und Konzepte umfassender,
gerechter und repräsentativer sein werden als bisher. Und dies ist doch
ein erstrebenswertes Ziel.

Literaturnachweise:

Chan, Lois Mai; Mitchell, Joan S.: \emph{Dewey Dezimalklassifikation --
Theorie und Praxis. Lehrbuch zur DDC 22}. München, 2006.

Country Meters (abgerufen am 05.06.21),
https://countrymeters.info/de/World.

Criado-Perez, Caroline: \emph{Unsichtbare Frauen -- Wie eine von Daten
beherrschte Welt die Hälfte der Bevölkerung ignoriert}. München, 2020.

D'Ignazio, Catherine; Klein, Lauren F.: \emph{Data Feminism}.
Massachusetts, 2020, \linebreak https://data-feminism.mitpress.mit.edu/.

Kantayya, Shalini: \emph{Coded Bias}. United States: 7th Empire Media,
2020.

O'Neil, Cathy: \emph{Weapons of Math Destruction -- How Big Data
increases Inequality and threatens Democracy}. London, 2017.

Roig, Emilia: \emph{Why we matter -- Das Ende der Unterdrückung}.
Berlin, 2021.

Zweig, Katharina: \emph{Ein Algorithmus hat kein Taktgefühl -- Wo
künstliche Intelligenz sich irrt, warum uns das betrifft und was wir
dagegen tun können.} München, 2019.

%autor
\begin{center}\rule{0.5\linewidth}{0.5pt}\end{center}

\textbf{Sara Juen} , *1983, Sortimentsbuchhändlerin, von 2016--2019
Bachelorstudium in Bibliotheks- und Informationswissenschaft an der
Humboldt Universität zu Berlin, seit Oktober 2019 Masterstudentin für
Informationswissenschaft (HU). Seit 2021 Mitglied der LIBREAS Redaktion.
ORCID: https://orcid.org/0000-0003-0725-8592

\end{document}