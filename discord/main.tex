\documentclass[a4paper,
fontsize=11pt,
%headings=small,
oneside,
numbers=noperiodatend,
parskip=half-,
bibliography=totoc,
final
]{scrartcl}

\usepackage[babel]{csquotes}
\usepackage{synttree}
\usepackage{graphicx}
\setkeys{Gin}{width=.4\textwidth} %default pics size

\graphicspath{{./plots/}}
\usepackage[ngerman]{babel}
\usepackage[T1]{fontenc}
%\usepackage{amsmath}
\usepackage[utf8x]{inputenc}
\usepackage [hyphens]{url}
\usepackage{booktabs} 
\usepackage[left=2.4cm,right=2.4cm,top=2.3cm,bottom=2cm,includeheadfoot]{geometry}
\usepackage{eurosym}
\usepackage{multirow}
\usepackage[ngerman]{varioref}
\setcapindent{1em}
\renewcommand{\labelitemi}{--}
\usepackage{paralist}
\usepackage{pdfpages}
\usepackage{lscape}
\usepackage{float}
\usepackage{acronym}
\usepackage{eurosym}
\usepackage{longtable,lscape}
\usepackage{mathpazo}
\usepackage[normalem]{ulem} %emphasize weiterhin kursiv
\usepackage[flushmargin,ragged]{footmisc} % left align footnote
\usepackage{ccicons} 
\setcapindent{0pt} % no indentation in captions

%%%% fancy LIBREAS URL color 
\usepackage{xcolor}
\definecolor{libreas}{RGB}{112,0,0}

\usepackage{listings}

\urlstyle{same}  % don't use monospace font for urls

\usepackage[fleqn]{amsmath}

%adjust fontsize for part

\usepackage{sectsty}
\partfont{\large}

%Das BibTeX-Zeichen mit \BibTeX setzen:
\def\symbol#1{\char #1\relax}
\def\bsl{{\tt\symbol{'134}}}
\def\BibTeX{{\rm B\kern-.05em{\sc i\kern-.025em b}\kern-.08em
    T\kern-.1667em\lower.7ex\hbox{E}\kern-.125emX}}

\usepackage{fancyhdr}
\fancyhf{}
\pagestyle{fancyplain}
\fancyhead[R]{\thepage}

% make sure bookmarks are created eventough sections are not numbered!
% uncommend if sections are numbered (bookmarks created by default)
\makeatletter
\renewcommand\@seccntformat[1]{}
\makeatother

% typo setup
\clubpenalty = 10000
\widowpenalty = 10000
\displaywidowpenalty = 10000

\usepackage{hyperxmp}
\usepackage[colorlinks, linkcolor=black,citecolor=black, urlcolor=libreas,
breaklinks= true,bookmarks=true,bookmarksopen=true]{hyperref}
\usepackage{breakurl}

%meta
%meta

\fancyhead[L]{V. Geske, Y. Paulsen, H. Wiesenmüller, D. Brenn, B. Mattmann\\ %author
LIBREAS. Library Ideas, 39 (2021). % journal, issue, volume.
\href{http://nbn-resolving.de/}
{}} % urn 
% recommended use
%\href{http://nbn-resolving.de/}{\color{black}{urn:nbn:de...}}
\fancyhead[R]{\thepage} %page number
\fancyfoot[L] {\ccLogo \ccAttribution\ \href{https://creativecommons.org/licenses/by/4.0/}{\color{black}Creative Commons BY 4.0}}  %licence
\fancyfoot[R] {ISSN: 1860-7950}

\title{\LARGE{Communitybuilding über Discord – der DACH-Bibliothekswesen-Server}}% title
\author{Victoria Geske \and Yannick Paulsen \and Heidrun Wiesenmüller \and Daniel Brenn \and Beat Mattmann} % author

\setcounter{page}{1}

\hypersetup{%
      pdftitle={Communitybuilding über Discord – der DACH-Bibliothekswesen-Server},
      pdfauthor={Victoria Geske, Yannick Paulsen, Heidrun Wiesenmüller, Daniel Brenn, Beat Mattmann},
      pdfcopyright={CC BY 4.0 International},
      pdfsubject={LIBREAS. Library Ideas, 39 (2021).},
      pdfkeywords={Bibliothek, Community, Kommunikation, digitale Werkzeuge},
      pdflicenseurl={https://creativecommons.org/licenses/by/4.0/},
      pdfcontacturl={http://libreas.eu},
      baseurl={http://libreas.eu},
      pdflang={de},
      pdfmetalang={de}
     }



\date{}
\begin{document}

\maketitle
\thispagestyle{fancyplain} 

%abstracts
\begin{abstract}
\noindent
\textbf{Kurzfassung}: Discord ist ein Dienst, der ursprünglich aus der
Gaming-Szene kommt, mit dem man als Gruppe online miteinander
kommunizieren kann. Selbst erstellte Server können gestaltet werden, wie
es zum jeweiligen Freundeskreis oder der Community passt, und umfassen
sowohl schriftlichen als auch audiovisuellen Austausch. Den hier
thematisierten Server für Bibliotheksmenschen gibt es seit Februar 2021.
Das Interview fand am 22.06.21 über Videochat statt. Zu diesem Zeitpunkt
waren knapp 1200 Nutzer*innen registriert.
\end{abstract}

%body
\textbf{V. Geske: Wir sprechen heute über den Discord-Server
\enquote{DACH-Bibliothekswesen} und Sie sind drei der Admins, die diesen
versorgen und pflegen. Wir möchten erst einmal mit einer allgemeinen
Frage einsteigen: Was machen Sie und Ihre User auf diesem Server?}

D. Brenn: Die Idee war, eine Plattform zum fachlichen Austausch für das
Bibliothekswesen im DACH-Raum zu schaffen, und das ist im Grunde auch
das, was wir auf dem Server machen. Das heißt, wir versuchen eine
Diskussionskultur zu schaffen und eine Plattform zu bieten, auf der man
sich zu fachlichen Fragen auf einem niedrigschwelligen Niveau
austauschen kann.

H. Wiesenmüller: Auf dem Server gibt es verschiedene fachliche Kanäle.
Je nach Interesse für bestimmte Themen kann man regelmäßig im jeweiligen
Kanal nachschauen oder einstellen, dass man bei Aktivitäten eine
Benachrichtigung bekommt. Der fachliche Austausch ist auch ein
Instrument zum Community-Building: Es können auch mal Leute miteinander
sprechen, die das sonst vielleicht nicht täten. Und ein Anlass, den
Discord-Server zu erstellen, war natürlich die Pandemie-Situation, in
der viele normale Bereiche zur Begegnung von Bibliothekarinnen und
Bibliothekaren über Institutionen hinweg weggefallen sind. Das war auch
verbunden mit der Diskussion über virtuelle Konferenzen, die ganz
unterschiedlich aufgezogen werden. Manche sind so wie Fernsehen und man
sitzt passiv davor. Ein sehr gutes Gegenbeispiel war die SWIB-Konferenz
im letzten Herbst, bei der ein ähnliches Tool wie Discord (Mattermost)
parallel für Diskussionen genutzt wurde. Das war so der Ausgangspunkt,
den Discord-Server aufzusetzen.

\textbf{V. Geske: Also eine Plattform, die schriftliche und auch
mündliche Kommunikation vereint?}

D. Brenn: Das war auch ein wichtiger Faktor, auf jeden Fall.

B. Mattmann: Gewissermaßen die Fusion von Forum und Zoom. Ursprünglich
hatten wir auch die Idee, dass man sich spontan in der Mittagspause oder
bei Tagungen in der Pause treffen kann, um die Beiträge zu diskutieren.

D. Brenn: Es gab beispielsweise beim Bibliothekstag einen Punkt, an dem
die Plattform abstürzte und wir uns kurzfristig in einem der
Discord-Channels sammelten und warteten, bis alles wieder funktionierte.

H. Wiesenmüller: Wir haben auch Audiochannels, zum Beispiel einen, der
\enquote{Großraumbüro} heißt und so intendiert ist, dass man eintreten
und arbeiten kann und sich dann vielleicht auch zwischendurch mit
anderen unterhält - das wird aber nicht so häufig genutzt. Darüber
hinaus gibt es den Direktchat (auch als Audio- oder Video-Chat). Also
mag es viele Begegnungen geben, die wir nicht mitbekommen, weil sie
nicht im öffentlichen Raum stattfinden.

D. Brenn: Und ich stelle, auch bei anderen Discord-Servern, auf denen
ich unterwegs bin, fest, dass die meisten Leute eher still mitlesen oder
nur mal reinlesen. Es ist auch normal, dass nicht die ganze Zeit Leute
in Audiochannels hängen, sondern eher sporadisch oder seltener. Bei den
Gamebrarians\footnote{Discord-Server für videospielende
  Bibliotheksmenschen, eröffnet im Herbst 2020. Gemeinsame Events zur
  Freizeitgestaltung.} ist das auch ähnlich und beim
DHall-Server\footnote{Ursprünglich Hallenser
  Digital-Humanities-Stammtisch, Discord-Server eröffnet im März 2020.
  Darauffolgend überregionale Ausbreitung in der DH-Community.}, den ich
auch mit administriere, ist das exakt das Gleiche. Also das schätze ich
nicht als ungewöhnlich ein.

\textbf{V. Geske: Das hält mit Sicherheit die Kommunikation am Laufen,
auch seitens der Mitglieder. Vielleicht fühlen sich stumme Leser bei
einem der Themen mehr angesprochen. Ist denn der Server neben
Bibliotheks- und Informationswissenschaftler*innen auch für fachlich
Außenstehende offen?}

B. Mattmann: Der Fokus liegt natürlich auf Bibliotheken und
Bibliotheksmitarbeiter*innen, wobei wir nicht zwischen
wissenschaftlichen und öffentlichen Bibliotheken unterscheiden, oder
Menschen, die einfach in diesem Sektor arbeiten. Der Server per se ist
offen, das heißt wir geben den Einladungslink heraus. Teilweise sind
auch schon Links unterwegs, die endlos gültig sind. Insofern können sich
auch Interessierte dazugesellen.

H. Wiesenmüller: In einigen Fällen sind interessierte Menschen aus
angrenzenden Bereichen mit auf dem Server, zum Beispiel eine Kollegin,
die früher im Bibliotheksbereich beschäftigt war und nun im Museum tätig
ist -- da sind wir ganz offen. Studierende sind übrigens auch gerne
gesehen. Inzwischen sind schon ziemlich viele aus verschiedenen
Hochschulen mit dabei.

\textbf{Y. Paulsen: Es gab im Februar sehr schnell sehr große
Beitrittszahlen. Wie haben Sie diesen starken Zuspruch wahrgenommen.
Haben Sie überhaupt damit gerechnet?}

D. Brenn: Nach der Ankündigung über Inetbib ging es relativ flott, dass
die Nutzungszahlen so anstiegen. Wir hatten den Server bereits ein oder
zwei Wochen vorher, um im ganz kleinen Kreis zu testen und dann hat
Heidrun es mit großem Trommelwirbel über Inetbib verbreitet. Wir haben
auch ein Begrüßungsevent veranstaltet, bei dem etwa 80 Personen auf dem
Server waren, die ihn ausprobiert haben. Gerechnet haben wir damit nicht
unbedingt.

H. Wiesenmüller: Interessanterweise sind auch Leute gekommen, für die
das wirklich ein Erstversuch war. Ich kann aus meiner eigenen
Entwicklungsgeschichte sagen, dass ich Discord auch erst seit letztem
Jahr kenne, weil wir es zur internen Kommunikation in meinem Studiengang
unter den Professor*innen und Mitarbeitenden nutzen. Es gab viel
Neugierde auf diese Art von Tool -- Discord ist ja eines von
verschiedenen dieser Art. Wir haben hier sicherlich auch ein bisschen
Entwicklungshilfe bei Leuten höheren Alters geleistet.

B. Mattmann: Für diese Leute war der erste Abend eingeplant, an dem wir
Vieles erklärt haben. Wir haben auch einen Channel eingerichtet, in dem
wir die FAQ haben, etwa welche Einstellungen relevant sind. Aber die
Beitritte kamen in den ersten Tagen schubweise. Heidrun, Daniel und ich
haben es beispielsweise über Twitter angekündigt und dann etwas später
über Inetbib, die Schweizer Mailingliste swiss-lib und ForumÖB geteilt.
In Wellen traten mehrere Hundert Personen in kurzer Zeit bei und
anschließend wurde der Server zusätzlich gezielt in Arbeitsgruppen
verbreitet.

\textbf{Y. Paulsen: Auch wenn diese hohen Beitrittszahlen nicht erwartet
wurden, war es schon eine bewusste Entscheidung zu sagen \enquote{Wir wollen
jetzt nicht in diesem privaten Kreis bleiben, sondern es sollen mehr
Leute zu uns kommen}?}

B. Mattmann: Ja, der Ursprungskreis bestand wirklich nur aus ein paar
Admins. Wir hatten aber von Beginn an diese große Vision und wir fanden
uns zuerst nur in dieser kleinen Runde zusammen, um unsere Idee im
Kleinen zu testen. Dazu zählte auch, die ganzen Einstellungen mit den
Benachrichtigungen und Rechten zu prüfen, bevor wir den Server dann auf
die große Masse losließen.

\textbf{Y. Paulsen: Bibliotheksmenschen muss man nicht wirklich in den
Austausch zwingen, da gibt es natürlich die Mailinglisten, aber auch
eine sehr aktive Twitter-Community und man trifft sich regelmäßig auf
Konferenzen. Was macht Discord aus, was man vorher nicht schon konnte,
was jetzt der Anreiz für Leute ist, auf eine neue Plattform zu
wechseln?}

H. Wiesenmüller: Es ist schon auffällig, dass viele, die auf Twitter
aktiv sind, jetzt auch aktiv auf Discord sind. Insofern sieht man
vielleicht schon eine gewisse Affinität zu dieser Art von Kommunikation,
wobei Twitter öffentlich ist. Der Discord-Server ist zwar auch offen,
aber er ist trotzdem ein gewisser geschützter Raum. Es gab zu Beginn
Fragen, ob wir da irgendetwas für die Ewigkeit speichern. Das haben wir
verneint, Discord hat einen anderen Charakter. Es ist eben mehr eine
Unterhaltung, man hört zu, schaltet sich ein. Da ist eine Mailingliste
doch sehr viel sperriger.

D. Brenn: Und wir haben sozusagen den Vater der Inetbib-Mailingliste mit
in unserem Adminkreis, den Michael Schaarwächter. Wir würden uns für die
Zukunft wünschen, dass einige dieser Mailinglisten-Diskussionen
stattdessen auf Discord stattfinden würden, weil man da eben doch anders
diskutieren kann. Und wenn man feststellt, dass man sich im Text
missversteht, kann man auf einen Voice-Channel ausweichen, um das ein
bisschen präziser durchzusprechen.

H. Wiesenmüller: Es muss nichts anderes komplett ablösen, aber es kann
sich gut ergänzen. Und um nochmal auf den Punkt Konferenzen
zurückzukommen: Es wird ja gerade sehr viel darüber diskutiert,
insbesondere was die Anteile von realer und virtueller Konferenz sein
sollen. Für mich ist etwa besonders attraktiv, auch während einer realen
Konferenz, nicht nur mit meiner Nebenfrau oder meinem Nebenmann oder wem
ich zufällig begegne im Vortrag ein bisschen zu schwätzen, sondern auch
gezielt jemanden kontaktieren und während eines Vortrags in einem
geschützten Raum chatten zu können. Es gab öfter Konferenzsysteme, die
das ermöglichen wollten, beispielsweise Konferenzapps, aber so richtig
eingeschlagen sind die eigentlich nie. Und da könnte so eine
Parallelität von realer und virtueller Begegnung schon ganz interessant
sein - ein zusätzlicher Kanal, eine zusätzliche Dimension von
Kommunikation und Begegnung.

D. Brenn: Ein Vorteil gegenüber solchen Konferenzapps ist auch, dass man
im Idealfall auf dem Server bereits angemeldet ist. Wir hatten Fälle,
bei denen auf Discord gefragt wurde, ob da gerade jemand im selben Panel
ist oder ob sich jemand mit dem jeweiligen Thema beschäftigt, um
Rückfragen zu stellen. Gerade der Bibtag hat gezeigt, wo die Vorteile
liegen. Mir hat dieser Austausch auf dem Discord-Server und auf anderen
Kanälen einen guten Teil dieses Konferenzfeelings gegeben, dass man mit
Kolleg*innen auf der Konferenz unterwegs ist. Wohingegen man bei
Videokonferenzen ohne kollegialen Austausch nur fernsehmäßig davor sitzt
und konsumiert.

\textbf{Y. Paulsen: Das ist auf jeden Fall ein sehr attraktives
Gesamtsystem, was sich da anbietet. Denken Sie, dass sich eine Art
Discord-Trend entwickelt? Herr Brenn hat zum Beispiel letzten Donnerstag
auf dem Bibliothekartag den Digital-Humanities-Server vorgestellt. Wir
haben jetzt nicht im Blick, wie das bei anderen Fachdisziplinen
aussieht, ob sich da auch Discord-Server entwickeln, aber es könnten zum
Beispiel einzelne Bibliotheken versuchen diese für ihre Communities zu
nutzen. Gibt es da Potenzial?}

H. Wiesenmüller: Man kann schon sagen, dass es einen gewissen
Discord-Trend gibt. Es gibt die Gamebrarians, den großen
Digital-Humanities-Server und jetzt auch unseren DACH-Server. Für uns
war ein sehr großer amerikanischer bibliothekarischer Discord-Server
vorbildgebend, beispielsweise bei der \enquote{Frage der Woche}. Das
haben wir von den amerikanischen Kolleg*innen übernommen: Sie posten
einmal in der Woche eine Frage, wobei alle benachrichtigt werden und man
die ganze Woche lang darüber diskutieren kann. Das empfanden wir als
eine sehr gute Idee, dass die Menschen jede Woche etwas von uns hören
und sie dazu eingeladen sind, sich u. a. zum wöchentlichen Thema zu
äußern.

D. Brenn: Da kommen wir zu einem Teil, bei dem uns Kritik für den
Bibliothekswesen-Server entgegen gebracht wurde: \enquote{Das ist ja
datenschutzrechtlich alles hoch fragwürdig, warum könnt ihr das nicht
Open-Source machen? Schon wieder so eine geschlossene Lösung!} Ich würde
einer Bibliothek nicht empfehlen, das als offizielles Angebot zu nutzen,
eben weil der Datenschutz da durchaus verbesserungswürdig ist. Und wir
würden uns auch für den Bibliothekswesen-Server eine institutionelle
gehostete Lösung wünschen, die einerseits Open-Source ist, aber
andererseits diese Funktionalitäten mitbringt. Leider gibt es keine
Open-Source-Lösung, die das out-of-the-box mitbringt.

B. Mattmann: Wir Admins haben mit \enquote{Element}{[}\^{}3{]} vor etwa
zwei Monaten einen Testlauf gemacht, aber gemerkt, dass damit viele
Funktionalitäten einfach wegfallen würden. Bezüglich des Trends ist es
eine Mischung aus der Pandemie und dass sich viele Menschen virtuelle
Ersatzwelten gesucht haben. Discord selbst ist in der Gaming-Community
sehr verbreitet, aber auch in der Streaming-Community. Im Zuge der
Pandemie sind sehr viele Twitch-Kanäle entstanden, die sich nicht nur
mit Gaming beschäftigen, sondern mit vielen unterschiedlichen Themen.
Mit diesen Communities ist meist ein Discord-Server für den Austausch
verbunden und so verbreitet sich das Tool natürlich in sämtliche Kreise.
Wenn man dann ein Community-Tool sucht, stützt man sich auf das Bekannte
und da Discord so populär ist, wird es sich in vielen Bereichen
durchsetzen und präsent sein. Wir selbst haben bei der
Universitätsbibliothek Basel projektintern Discord als
Kommunikationsmittel verwendet, im Austausch mit mehreren Partnern, die
an einem Entwicklungsprojekt beteiligt waren. Weil wir einfach eine
Lösung brauchten, die nicht mit einer Authentifizierungshürde versehen
war, die wir für Externe lösen müssen. Solange man nicht sensible
Informationen teilt, ist Discord als leicht und offen zugänglicher
Dienst nützlich.

H. Wiesenmüller: Ich weiß nicht, ob dieser Server in fünf Jahren noch
auf Discord laufen wird. Es gibt jetzt schon verschiedene Produkte,
u. a. \enquote{Mattermost} als Open-Source-Lösung und Slack, was auch viele in
der Businesswelt nutzen. Der Trend zu einer ungefähr so aussehenden
Kommunikationsplattform ist m. E. nicht zu stoppen und wird auch für
diejenigen, die sowas im Moment noch nicht nutzen, in den nächsten
Jahren zum Standard werden. Wie es genau aussehen wird, weiß ich nicht,
aber diese grundsätzliche Art einer Plattform geht so schnell erstmal
nicht weg. Sie wird eher noch an Bedeutung gewinnen.

D. Brenn: Ein Vorteil, den Discord außerdem im Vergleich zu einer
Open-Source-Alternative hat: Wenn man einmal einen Discord-Account hat,
kann man damit auf jeden beliebigen Discord-Server joinen. Mit Open
Source ist das nicht so einfach hinzubekommen.

\textbf{V. Geske: Wir werden also sehr wahrscheinlich eine ganze Weile
mit Discord zu tun haben, wenn es mit dem Trend so weitergeht und jetzt
interessiert uns natürlich auch, wie es hinter den Kulissen abläuft. Wie
organisieren Sie sich als Admin-Team? Haben Sie bestimmte Aufgaben oder
Aufgabenbereiche, die aufgeteilt sind oder steuert so jeder seins bei?}

B. Mattmann: Wer sich zuerst regt, muss die Aufgabe übernehmen. (lacht)

H. Wiesenmüller: Das Geniale ist wirklich, dass es ein sich selbst
organisierendes System ist. Wir haben einen gemeinsamen Bereich, das
\enquote{Admin-Sofa}, das ist unser geschlossener Kanal und wir haben
eigentlich auch einen geschlossenen Voicechannel, den wir aber nie
ernsthaft nutzen, und da passiert alles. Regelmäßig zu bearbeiten ist
die Frage der Woche und da fragt immer irgendeine*r aus dem Admin-Team
am Tag vorher, ob jemand eine Idee für die neue Frage der Woche hat. Wir
sammeln natürlich auch schon einige Ideen in einem Dokument, aber es ist
eigentlich nicht abgestimmt. Einmal fängt der Eine und einmal fängt die
Andere an. Ähnlich ist es mit Wünschen nach neuen Kanälen. Wir haben
einen eigenen Kanal \enquote{Vorschläge und Wünsche}, wo dann manchmal etwas
kommt und irgendjemand reagiert zuerst. Aber letztlich diskutieren wir
in der Runde -- wer dabei ist, diskutiert mit. Wenn sich mal jemand
gerade ausgeklinkt hat für ein paar Tage, ist das auch nicht schlimm.
Wir sind genügend, sodass immer jemand da ist, um zu reagieren, ohne
dass wir genaue Zuständigkeiten verteilen müssten. Im Vergleich zu den
vielen oft streng durchorganisierten Sachen, die man so im normalen
Leben hat, finde ich das ausgesprochen angenehm.

B. Mattmann: Wir hatten zu Beginn etwa fünf Personen in der
Administrationsrunde, mit denen wir nach der Pilotphase starteten. Als
dann das Wachstum so stark war und die ÖBs im Zuge dessen mit
verschiedenen neuen Wunschthemen sehr präsent wurden, haben wir auch
festgestellt, dass wir eigentlich nur WB-Vertreter*innen sind. Daraufhin
haben wir gezielt Personen, die aus dem ÖB-Umfeld kommen, gesucht und
eingeladen, unsere Adminrunde zu vergrößern.

H. Wiesenmüller: Aber das Lustige ist, dass wir zu allen möglichen und
unmöglichen Zeiten kommunizieren. Heute hatten wir etwa um 6:30 Uhr
schon die ersten Chatanfragen zu unserer Frage der Woche.

\textbf{V. Geske: Wenn ich das richtig verstehe, ist das ein eher
lockerer Workflow. Mit wie viel Aufwand müssen Sie denn ungefähr
rechnen, schließlich leisten Sie das ehrenamtlich nebenbei?}

B. Mattmann: 15 Stunden die Woche (scherzhaft). Nein. Auf dem Server
sind knapp 1200 registrierte User, aber wie viele tatsächlich aktiv
sind, kann man nicht genau sagen. Man kann in Discord Aktivitätsabfragen
machen. Ich habe das mal getestet und die Hälfte der User war in den
letzten 30 Tagen mindestens einmal auf dem Server aktiv. Anfangs gab es
einen recht starken Peak, aus dem ganz viele Vorschläge zu Kanälen kamen
und Fragen offen waren. Da hatten wir Einiges zu tun, diese Fragen zu
beantworten und das Ganze in gute Bahnen zu lenken. Ich persönlich
schaue mindestens einmal täglich durch all meine Discord-Server durch
und gucke, was ansteht, bringe auch mal Anliegen ein. Für mich sind das
5-10 Minuten am Tag für den DACH-Server. Manchmal eine Stunde, je
nachdem, ob eine Diskussion aufkommt.

H. Wiesenmüller: Ja, wobei wir sicher noch mehr tun könnten. Den Server
am Laufen zu halten, nachdem es sich eingespielt hat, ist nicht sehr
aufwändig. Am Anfang hatten wir die Idee, öfter mal eine Art Event zu
veranstalten, zum Beispiel einen Stammtisch in einem Audiochannel. Das
haben wir, abgesehen von dem Einstiegsevent, bisher noch nicht gemacht.
Ich denke auch, dass es eine größere Gruppe gibt, die auf den Server
gekommen ist, um nur mal zu gucken, die vielleicht ein bisschen
überfordert war und mehr aktive Hilfestellung bräuchte. Da könnte man
noch Zeit investieren und eventuell eine ausführlichere Anleitung
schreiben oder eben im Voicechannel zu bestimmten Zeiten für
Einführungen und Erklärungen zur Verfügung stehen.

D. Brenn: Ein regelmäßiges Beisammensein wie bei einem Stammtisch ist
auf jeden Fall etwas, was in Zukunft auch kommen wird. Und bezüglich des
Zeitaufwandes: Eine gute Community managt sich selbst. Ich habe bereits
diverse Communities miterlebt und wenn nur einzelne Personen eine
Community am Leben halten und vorwärts ziehen, weil sie immer wieder
etwas posten, dann ergibt es irgendwann keinen Sinn mehr. Das trifft
hier nicht zu, aber was ich damit meine: Wenn nur die Admins diejenigen
sind, die da am Server ständig etwas anschieben, dann läuft etwas
falsch. Ich denke, wir sollten aktiver Communitymeetings veranstalten,
um eben auch diese Diskussionskultur mehr aufzubauen. Von der Sache her
läuft es aber.

\textbf{Y. Paulsen: Inzwischen gibt es ja auch einige permanente
Zutrittlinks. Es könnte deswegen durchaus sein, dass Sie zum Beispiel
Probleme mit irgendwelchen Trollen haben oder vielleicht auch Bots, die
versuchen, sich auf den Server zu stehlen. Ist sowas schon vorgekommen?}

D. Brenn: Auf diesem Server ist mir das erstaunlicherweise noch nicht
aufgefallen. Auf DHall ist uns das tatsächlich passiert. Aus diesem
Grund vergeben wir da keine permanenten Zugriffslinks, aber auf dem
Bibliothekswesen-Server war bisher nichts -- die Bibliothekswesen sind
alle fantastisch.

H. Wiesenmüller: Was ich mir noch wünsche, ist, dass, wie in unserer
Netiquette beschrieben, alle unter Klarnamen schreiben oder zumindest
mit einem Vornamen, wenn man sich nicht wirklich outen will. Das klappt
nicht hundertprozentig, aber wir haben es auch nicht forciert. Insgesamt
bevorzugen wir es, wenn wir uns, ähnlich wie wenn man sich bei einer
Konferenz begegnet, nicht anonym unterhalten.

B. Mattmann: Insbesondere, da Discord die Möglichkeit bietet, selbst
wenn wir auf mehreren Servern unterwegs sind, je nach Server den
Anzeigenamen anzupassen. Man kann in anderen Communities trotzdem anonym
unterwegs sein, auch wenn man sich auf dem DACH-Server mit dem Vornamen
sozusagen outet.

\textbf{Y. Paulsen: Ich würde jetzt einen Blick in die Zukunft werfen.
Der Discord-Server hat zum Beispiel seit April einen eigenen
Twitter-Account, deswegen interessiert uns, wo Sie vielleicht noch mit
dem Projekt hin möchten. Gibt es da irgendwelche Zukunftspläne oder
Ziele, abgesehen von dem regelmäßigen Stammtisch, über den wir schon
geredet haben?}

D. Brenn: Der Twitter-Account sollte auch noch ein bisschen aktiver
bespielt werden. Wir hatten da die Überlegung, die Frage der Woche und
ein paar Antworten öfter zu teilen. Das haben wir, glaube ich, einmal
gemacht.

H. Wiesenmüller: Ich würde unseren Server gar nicht als
\enquote{Projekt} bezeichnen. Er ist eine Graswurzelinitiative. Es war
ein konkretes Bedürfnis da und es haben sich Leute gefunden, die das
umgesetzt haben. Insbesondere Daniel Brenn, der die Sache einfach in die
Hand genommen und den Discord-Server aufgesetzt hat und dann ein paar
Leute ansprach, von denen er schon ahnte, dass diese Interesse haben
würden. Ich denke, wenn wir akute Bedürfnisse haben, wird es sich in die
entsprechende Richtung weiterentwickeln. Es ist aber nicht so, dass wir
uns in regelmäßigen Admin-Treffen den Kopf zerbrechen würden, welche
Ziele wir für die nächsten fünf Jahre haben.

B. Mattmann: Der Twitter-Account war eine spontane Maßnahme. Nicht aus
PR-Gründen, sondern weil wir uns fragten, wie man am leichtesten zu uns
auf den Server kommt. Im Zuge dessen haben wir kurz über eine Webseite
oder Ähnliches nachgedacht, aber schließlich einen Twitter-Account, über
den man sich melden und einen Link bekommen kann, bzw. ein Posting auf
Twitter mit einem Einladungslink als einfachste Variante empfunden. Das
war die Initialzündung für den Twitter-Account, den wir anschließend
noch etwas ausgebaut haben.

\textbf{Y. Paulsen: Die Stärke liegt aus Ihrer Sicht also eher in der
jetzigen Struktur des Servers und er soll nicht auf eine offiziellere
Ebene gehoben werden?}

H. Wiesenmüller: Es ist alles sehr, sehr spontan und das ist eigentlich
auch das Schöne. Gerade weil es so anders ist als das, was wir aus
unserem normalen Berufsalltag kennen, macht es uns auch so viel Spaß.

B. Mattmann: Das beste Beispiel ist der Bibliothekartag. Zwei Tage
vorher hatten wir die Idee, dass wir die Parallelkanäle zur Verfügung
hätten, um die Tagung zu begleiten. Wir haben nochmal über Twitter usw.
erinnert und das einfach kurz vorher rausgehauen. Und ich denke, so
funktioniert es im Moment wirklich am besten -- auch mit den Personen,
die Teil des Admin-Teams sind, weil wir doch sehr unterschiedliche
Belastungen haben. Manche sind in der Lehre tätig, manche in
Leitungsfunktionen, andere regulär angestellt. Auch unterscheiden sich
die Arbeitsverhältnisse in Teilzeit, Vollzeit, Dienste an Wochenenden
etc.

H. Wiesenmüller: Richtig. Und man sieht letztlich diesen
Wahnsinnsvorteil von so einer Chatplattform -- es geht eben
blitzschnell. Ein wöchentliches oder 14-tägliches Meeting ist nicht
nötig. Wenn jemand einen Gedanken hat, chattet er*sie das, und es
reagiert immer irgendjemand. Nicht alle äußern ihre Meinung, was aber
auch nicht sein muss, weil man eben auch mal offline ist. Zwei, drei
Leute melden sich eigentlich immer und teilen sich mit, sei es per
Daumen hoch oder sie geben einen Verbesserungsvorschlag. Aus diesem
Grund können wir so fluide Entscheidungen treffen.

B. Mattmann: Genau, so werden u. a. Ideen zur weiteren Entwicklung, mit
der Frage, ob das ok ist, einfach mal in die Runde geworfen. Entweder
fallen sie gerade auf fruchtbaren Boden und man diskutiert darüber oder
man schaut sich das später noch einmal an, wenn man mehr Zeit und einen
freien Kopf hat, um darüber zu sprechen.

\textbf{V. Geske: Damit sind wir auch schon fast am Ende angelangt.
Möchten Sie den Leserinnen und Lesern noch etwas mitgeben, was nicht
erwähnt wurde?}

D. Brenn: Wenn Sie Interesse haben, die Türen stehen immer offen.

H. Wiesenmüller: Genau. An jene, die es noch nicht probiert und eine
gewisse Scheu haben -- es kann nichts passieren. Man kann einfach mal
hereinschnuppern und wir helfen auch gerne weiter, wenn es anfangs
Schwierigkeiten geben sollte.

B. Mattmann: Man muss auch nicht mit offiziellen Mails an uns
herantreten, sondern kann jemandem von uns schreiben oder auf Twitter
kontaktieren mit der Frage, ob sich jemand Zeit nehmen und eine
Einführung geben könnte. Da sind wir ganz unkompliziert. Falls man sich
vielleicht fürchtet, sich in die große Community zu begeben, können in
der sicheren Umgebung von zwei Personen alle möglichen Anfängerfragen
gestellt werden. So kann man sich in Ruhe durch alles hindurchführen
lassen.

D. Brenn: Eine Sache, die mir noch einfällt: Das ist natürlich sehr
pathetisch und sehr beladen, aber: Ohne Kommunikation, ohne Vernetzung,
nicht nur unter Bibliothekar*innen, kommt man am Ende nicht weit. Es
hilft natürlich, Leute zu kennen, die was wissen. Es hilft aber auch,
den Ort zu kennen, wo man Leute antrifft, die man fragen kann, die was
wissen. Das ist im Grunde das, was wir mit dem Server erreichen wollten
-- einen Ort zur Vernetzung, zur Kommunikation bieten zu können. Auch
unabhängig von dem Server kann ich die Wichtigkeit von Kommunikation
nicht überbetonen.

\textbf{V. Geske: Ein tolles Schlusswort. Wir danken Ihnen noch einmal
herzlich für das Interview sowie für die Zeit, die Sie sich genommen
haben und wünschen Ihnen weiterhin viel Freude und Erfolg.}

%autor
\begin{center}\rule{0.5\linewidth}{0.5pt}\end{center}

\textbf{Victoria Geske und Yannick Paulsen}  studieren am
Institut für Bibliotheks- und Informationswissenschaft der
Humboldt-Universität zu Berlin und haben für LIBREAS das folgende
Interview mit drei Initiator*innen des Discord-Servers
``DACH-Bibliothekswesen'' geführt. \textbf{Heidrun Wiesenmüller, Daniel Brenn
und Beat Mattmann} waren bei diesem Discord-Projekt von Beginn an dabei
und sind Teil des vielköpfigen Adminteams.

\end{document}
